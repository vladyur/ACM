\documentclass[a4paper,12pt]{article}

%%% Работа с русским языком
\usepackage[cm]{fullpage}
\usepackage[T2A]{fontenc}
\usepackage[utf8]{inputenc}
\usepackage[english,russian]{babel}
\usepackage{footmisc}
\usepackage[document]{ragged2e}
\usepackage{amsmath,amsfonts,amssymb,mathtools}
\usepackage{framed}
\usepackage{pstricks}
%\usepackage[framed]{ntheorem}
\usepackage{tikz}
\usetikzlibrary{arrows,automata}
\usepackage{cmap}					% поиск в PDF
\usepackage{mathtext} 				% русские буквы в формулах
\usepackage{indentfirst}
\frenchspacing

\newcommand{\vyp}{\hookrightarrow}
\renewcommand{\epsilon}{\varepsilon}
\renewcommand{\phi}{\varphi}
\renewcommand{\kappa}{\varkappa}
\renewcommand{\le}{\leqslant}
\renewcommand{\leq}{\leqslant}
\renewcommand{\ge}{\geqslant}
\renewcommand{\geq}{\geqslant}
\renewcommand{\emptyset}{\varnothing}

%%% Дополнительная работа с математикой
\usepackage{amsmath,amsfonts,amssymb,amsthm,mathtools} % AMS
\usepackage{icomma} % "Умная" запятая: $0,2$ --- число, $0, 2$ --- перечисление

%% Номера формул
%\mathtoolsset{showonlyrefs=true} % Показывать номера только у тех формул, на которые есть \eqref{} в тексте.
%\usepackage{leqno} % Нумереация формул слева

%% Свои команды
\DeclareMathOperator{\sgn}{\mathop{sgn}}

%% Перенос знаков в формулах (по Львовскому)
\newcommand*{\hm}[1]{#1\nobreak\discretionary{}
{\hbox{$\mathsurround=0pt #1$}}{}}



%%% Работа с картинками
\usepackage{graphicx}  % Для вставки рисунков
\graphicspath{{images/}{images2/}}  % папки с картинками
\setlength\fboxsep{3pt} % Отступ рамки \fbox{} от рисунка
\setlength\fboxrule{1pt} % Толщина линий рамки \fbox{}
\usepackage{wrapfig} % Обтекание рисунков текстом

%%% Работа с таблицами
\usepackage{array,tabularx,tabulary,booktabs} % Дополнительная работа с таблицами
\usepackage{longtable}  % Длинные таблицы
\usepackage{multirow} % Слияние строк в таблице

%%% Теоремы
\theoremstyle{plain} % Это стиль по умолчанию, его можно не переопределять.
\newtheorem{theorem}{Теорема}[section]
\newtheorem{proposition}[theorem]{Утверждение}
 
\theoremstyle{definition} % "Определение"
\newtheorem{corollary}{Следствие}[theorem]
\newtheorem{problem}{Задача}[section]
 
\theoremstyle{remark} % "Примечание"
\newtheorem*{nonum}{Решение}

%%% Программирование
\usepackage{etoolbox} % логические операторы

%%% Страница
\usepackage{extsizes} % Возможность сделать 14-й шрифт
\usepackage{geometry} % Простой способ задавать поля
\geometry{top=20mm}
\geometry{bottom=20mm}
\geometry{left=10mm}
\geometry{right=20mm}
 %
\usepackage{fancyhdr} % Колонтитулы
 	\pagestyle{fancy}
 	\renewcommand{\headrulewidth}{0pt}  % Толщина линейки, отчеркивающей верхний колонтитул
\fancypagestyle{firstpage}{
	\rhead{\large{Рябых Владислав, Б05-905}}
}
% 	\lfoot{Нижний левый}
% 	\rfoot{Нижний правый}
% 	\rhead{Верхний правый]}
% 	\chead{Верхний в центре}
% 	\lhead{Верхний левый}
%	\cfoot{Нижний в центре} % По умолчанию здесь номер страницы

\usepackage{setspace} % Интерлиньяж
\onehalfspacing % Интерлиньяж 1.5
%\doublespacing % Интерлиньяж 2
%\singlespacing % Интерлиньяж 1

\usepackage{lastpage} % Узнать, сколько всего страниц в документе.

\usepackage{soul} % Модификаторы начертания

\usepackage{hyperref}
%\usepackage[usenames,dvipsnames,svgnames,table,rgb]{xcolor}
\hypersetup{				% Гиперссылки
    unicode=true,           % русские буквы в раздела PDF
    pdftitle={Заголовок},   % Заголовок
    pdfauthor={Автор},      % Автор
    pdfsubject={Тема},      % Тема
    pdfcreator={Создатель}, % Создатель
    pdfproducer={Производитель}, % Производитель
    pdfkeywords={keyword1} {key2} {key3}, % Ключевые слова
    colorlinks=true,       	% false: ссылки в рамках; true: цветные ссылки
    linkcolor=red,          % внутренние ссылки
    citecolor=black,        % на библиографию
    filecolor=magenta,      % на файлы
    urlcolor=blue           % на URL
}

\usepackage{csquotes} % Еще инструменты для ссылок

%\usepackage[style=authoryear,maxcitenames=2,backend=biber,sorting=nty]{biblatex}

\usepackage{multicol} % Несколько колонок

\usepackage{pgfplots}
\usepackage{pgfplotstable}
\newcommand{\tbf}{\textbf}


\usepackage[shortlabels]{enumitem}

\newtheorem{task}{\textbf{Задача}}

\newtheorem{innercustomthm}{\textbf{Задача}}
\newenvironment{tasknum}[1]
{\renewcommand\theinnercustomthm{#1}\innercustomthm}
{\endinnercustomthm}

\newtheorem*{solution}{\textbf{Решение}}
\newcommand{\Ra}{\Rightarrow}
\newcommand{\La}{\Leftarrow}
\newcommand{\ra}{\rightarrow}
\newcommand{\LRa}{\Leftrightarrow}
\newcommand{\n}{\mathbb}
\newcommand{\Le}{\leqslant}
\newcommand{\Ge}{\geqslant}


\renewcommand{\inf}{\infty}
\newcommand{\ol}{\overline}

\newcommand{\bigline}{\noindent\makebox[\linewidth]{\rule{\paperwidth}{0.4pt}}}

\usetikzlibrary{fit}
\newcommand{\dost}{\overset{*}{\vdash}}
\newcommand{\vyv}{\overset{*}{\Rightarrow}}


\usepackage{capt-of}
\usepackage{tikz-qtree}
\usepackage{systeme}

\newcommand{\polysv}{\leq_p}
\def\coNP{{\mathbf{\text{\textbf{co--}}\mathcal{NP}}}}
\newcommand{\NP}{\mathcal{NP}}
\renewcommand{\P}{\mathcal{P}}

\usepackage{algorithm}
%\usepackage{algpseudocode}
\usepackage[noend]{algpseudocode}

\newcommand{\prob}[1]{\mathbb{P}\left\{#1\right\}}
\newcommand{\expected}[1]{\mathbb{E}\left\{#1\right\}}


\begin{document}
	
	\thispagestyle{firstpage}
	
	\begin{center}
		\textbf{\Large{Алгоритмы и модели вычислений. \\ Домашнее задание № 12}}
	\end{center}
	

\begin{task}
	В протоколе $RSA$ выбраны $p = 17$, $q = 23$, $N=391$, $e=3$. Выберите ключ $d$ и зашифруйте сообщение $41$. Затем расшифруйте полученное сообщение и убедитесь, что получится исходное $41$.
\end{task}	

\begin{solution}
	Зашифрованное сообщение: $y = m^e \mod N = 41^3 \mod 391 = 68921 \mod 391 = 105$
	
	Для того, чтобы расшифровать сообщение, сначала необходимо вычислить значение $d$:
	
	$d = e^{-1} \mod (p-1)(q-1) = 3^{-1} \mod 352 = 235$, так как $3 \cdot 235 = 705 \underset{352}{\equiv} 1$
	
	Теперь расшифровываем: $y^d \mod N = 105^{235} \mod 391$
	
	Долго думал, как посчитать остаток сразу по модулю 391, но ни одна идея сильно подсчёт не облегчает, так что посчитаем остаток по модулям $p = 17$ и $q = 23$, а потом воспользуемся КТО.
	
	\begin{itemize}
		\item $105^{235} = (105^{16})^{14} \cdot 105^{11} \overset{\text{по МТФ}}{\underset{17}{\equiv}} 1 \cdot 105^{11} \underset{17}{\equiv} 3^{11} = 27^3 \cdot 9 \underset{17}{\equiv} 10^3 \cdot 9 \underset{17}{\equiv} -3 \cdot 9 \underset{17}{\equiv} 7$
	\item $105^{235} = (105^{22})^{10} \cdot 105^{15} \overset{\text{по МТФ}}{\underset{23}{\equiv}} 1 \cdot 105^{15} \underset{23}{\equiv} 13^{15} = 169^7 \cdot 13 \underset{23}{\equiv} 8^7 \cdot 13 = 2^{22} \cdot 2^{-1} \cdot 13 \hm{\overset{\text{по МТФ}}{\underset{23}{\equiv}}} 2^{-1} \cdot 13 = 12 \cdot 13 \underset{23}{\equiv} 18$
	
	Воспользуемся КТО: $\left\{
	\begin{aligned}
	x &\underset{17}{\equiv} 7 \\
	x &\underset{23}{\equiv} 18\\
	\end{aligned}
	\right.
	$
\end{itemize}

	Сначала найдём обратные элементы:
	
	\begin{itemize}
		\item по модулю 17: $23^{-1} = 6^{-1} = 3$, так как $6 \cdot 3 = 18 \underset{17}{\equiv} 1$
		\item по модулю 23: $17^{-1} = 19$, так как $17 \cdot 19 = 323 \underset{23}{\equiv} 1$
	\end{itemize}
	
	Тогда $x  \underset{391}{\equiv} 7 \cdot 3 \cdot 23 + 18 \cdot 19 \cdot 17 = 483 + 5814 = 6297 \underset{391}{\equiv} 41$
	
	Таким образом $y^d \mod N = 105^{235} \mod 391 = 41$~---~сообщение успешно расшифровано
	
\end{solution}

\begin{task}
	Пусть в протоколе $RSA$ открытый ключ $(N, e)$, $e=3$. Покажите, что если злоумышленник узнаёт закрытый ключ $d$, то он может легко найти разложение $N$ на множители.
\end{task}

\begin{solution}
	Так как $d = e^{-1} \mod (p-1)(q-1)$, то $d \cdot e \underset{(p-1)(q-1)}{\equiv} 1 \Ra d\cdot e - 1 = k \cdot (p-1)(q-1), k \in \mathbb{Z}$
	
	$e=3$ по условию, так что $3d - 1 = k \cdot (p-1)(q-1), k \in \mathbb{Z}$.
	
	$k \cdot (p-1)(q-1) \underset{3}{\equiv} 2; (p-1) \underset{3}{\equiv} 2 (q-1) \underset{3}{\equiv} 1 \Ra k \underset{3}{\equiv} 2$
	
	$d = e^{-1} \mod (p-1)(q-1) \Ra d < (p-1)(q-1) \Ra k = 2$~---~единственный вариант.
	
	Таким образом $3d - 1 = 2 \cdot (p-1)(q-1) = 2 \cdot (pq - p - q + 1)$
	
	$p = -q + N + \dfrac{3}{2}(1-d)$
	
	$N = pq \Ra N = -q^2 + Nq + \dfrac{3}{2}(1-d)q$, откуда находим $q \Ra$ находим $p$.
\end{solution}

\begin{task}
	Схема $RSA$ позволяет также создавать защищенные электронные подписи. Если открытый ключ $(N, e)$, то автор сообщения, обладающий закрытым ключом $d$, отправляет сообщение $A^d$, где $A$~---~незашифрованное сообщение. После этого идентификация подписи - это возведение в степень $e$. Пусть открытый ключ $(2021, 25)$. В какую степень автору нужно возвести сообщение, чтобы отправить его за своей электронной подписью?
\end{task}

\begin{solution}
	$N = 2021 = 43 \cdot 47 = p \cdot q$. 
	
	Тогда степень, в которую автору нужно возвести сообщение, есть $d = e^{-1} \mod (p-1)(q-1) = 25^{-1} \mod 42\cdot46 = 25^{-1} \mod 1932$.
	
	$25 \cdot d \underset{1932}{\equiv} 1 \Ra 25d + 1932k = 1$, где $d, k\in \mathbb{Z}$~---~диофантово уравнение, которое мы решим с помощью расширенного алгоритма Евклида (учтём, что $(25, 1932) = 1$, так что решение в целых числах существует).
	
	\begin{tabular}{|c|c|c|c|c|c|}
		\hline
		$d$& 1 & 0 & -77 & 232 & -1391 \\
		\hline
		$k$& 0 & 1 & 1 & -3 & 18 \\
		\hline
		$25d+1932k$& 25 & 1932 & 7 & 4 & 1 \\
		\hline
	\end{tabular}
	
	\vspace{4mm}
	
	Таким образом, $d = -1391 + 1932m, m \in \mathbb{Z}$. Тогда берём $d = -1391 + 1932 \cdot 1 = 541$
	
	Это и есть искомая степень.
	
\end{solution}

\begin{task}
	Решите уравнение $\varphi(n) = 6$, где $\varphi(n)$~---~функция Эйлера (количество чисел, не превосходящих $n$ и взаимно простых с ним).
\end{task}

\begin{solution}
	Сразу заметим, что $n \ge 7$, причём $n = 7$ подходит, так как 7~---~простое число, а для простых чисел справедливо $\phi(p) = p-1$.
	
	$$2 \cdot 3 = 6 = \phi(n) = n \cdot \left(1 - \dfrac{1}{p_1}\right) \cdot \ldots \cdot \left(1 - \dfrac{1}{p_m}\right)$$
	
	Отсюда можно сказать, что простых делителей в разложении числа $n$ не больше двух. 
	
	Допустим, что простой делитель только один. Тогда $n = p_1^k \Ra 2 \cdot 3 = \phi(n) = p_1^{k-1} \cdot (p_1 - 1) \Ra p_1 = 7, k = 1$ и $p_1 = 3, k = 2$~---~единственно возможные варианты. То есть ещё добавляем к множеству решений $n = 9$.
	
	Смотрим теперь ситуацию, когда простых делителей два:
	
	\[2 \cdot 3 = n \cdot \left(1 - \dfrac{1}{p_1}\right) \cdot \left(1 - \dfrac{1}{p_2}\right) = n \cdot \dfrac{p_1 - 1}{p_1} \cdot \dfrac{p_2 - 1}{p_2} = p_1^{k_1 - 1} \cdot p_2^{k_2-1} \cdot (p_1 - 1) \cdot (p_2 - 1)\]
	
	Таким образом, чтобы в правой части было 2 делителя есть всего 2 возможных случая:
	\begin{itemize}
		\item $k_1 = k_2 = 1 \Ra 2 \cdot 3 = (p_1 - 1) \cdot (p_2 - 1)$. Существует всего 2 варианта: $2 \cdot 3$ и $1 \cdot 6$. Первый даёт нам ничего не даёт, так как $p_2 = 3 + 1 = 4$ на самом деле не является простым, а второй даёт $n = 2 \cdot 7 = 14$.
		\item $p_1 = 2$. Тогда $2\cdot 3 = 2^{k_1 - 1} \cdot p_2^{k_2-1} \cdot (p_2 - 1)$. Сразу заметим, что $1 \le k_1 \le 2$. Опять делим на 2 случая:
		
		\begin{itemize}
			\item $k_1 = 1 \Ra 2 \cdot 3 = p_2^{k_2-1} \cdot (p_2 - 1) \Ra k_2 = 2 \Ra 2\cdot 3 =p_2 \cdot (p_2 - 1) \Ra p_2 = 3$. То есть добавляем к множеству решений $n = 2 \cdot 3^2 = 18$.
			
			\item $k_2 = 2 \Ra 2 \cdot 3 = 2 \cdot p_2^{k_2-1} \cdot (p_2 - 1) \Ra k_2 = 1 \Ra p_2 - 1 = 3$~---~отбрасываем этот вариант, так как $p_2 = 3 + 1 = 4$ на самом деле не является простым.
		\end{itemize}
	\end{itemize}

	Таким образом, все решения уравнения $\varphi(n) = 6$~---~множество $\{7, 9, 14, 18\}$.

\end{solution}

\begin{task}
	Докажите, что в шифре Шамира в итоге у $B$ в действительности оказывается то сообщение, которое $A$ планировал передать.
\end{task}

\begin{solution}
	По условиям шифра мы знаем, что $c_a d_a \underset{p-1}{\equiv} c_b d_b \underset{p-1}{\equiv} 1$.
	
	Тогда $c_a d_a = u \cdot (p-1) + 1 = r + 1$, $c_b d_b = v \cdot (p-1) + 1 = s + 1$, где $u, v \in \mathbb{Z}$. Заметим также, что $r = u \cdot (p-1)$ и $s = v \cdot (p-1)$ делятся на $p-1$
	
	$B$ считает $x_4 = x_3^{d_b} = x_2^{d_a d_b} = x_1^{c_b d_a d_b} = m^{c_a c_b d_a d_b} = m^{(c_a d_a)\cdot (c_b d_b)} = m^{(r+1)\cdot (s+1)} = m^{rs + r + s + 1} = m \cdot m^{rs + r + s} \hm{\overset{\text{по МТФ}}{\underset{p}{\equiv}}} m \cdot 1 = m$, то есть $x_4 \underset{p}{\equiv} m$, что и требовалось доказать.
\end{solution}

\begin{task}
	Докажите, что в шифре Эль-Гамаля в итоге у $B$ в действительности оказывается то сообщение, которое $A$ планировал передать.
\end{task}

\begin{solution}
	$B$ считает $e \cdot r^{p-1-c_b} \mod p$, где $r = g^k \mod p; e = m \cdot d_b^k \mod p; d_b = g^{c_b} \mod p$.
	
	Таким образом $B$ считает $m \cdot d_b^k \cdot (g^k)^{p-1-c_b} = m \cdot g^{k c_b} \cdot g^{kp - k - k c_b} = m \cdot g^{k(p-1)} \hm{\overset{\text{по МТФ}}{\underset{p}{\equiv}}} m \cdot 1 = m$, что и требовалось доказать.
\end{solution}

\begin{task}
	Докажите, что в алгоритме шифрования Рабина $B$ в итоге сможет найти исходное передаваемое сообщение среди $(\pm apm_q \pm bqm_p)$.
\end{task}

\begin{solution}
	$A$ передаёт $y = m^2 \mod pq$
	
	$B$ вычисляет $ap+bq \underset{pq}{\equiv} 1$, а также $m_p = y^{\frac{p+1}{2}}, m_q = y^{\frac{q+1}{2}}$
	
	$m_p = m^{\frac{p+1}{2}} = m \cdot \left(\dfrac{m}{p}\right)$;
	$m_q = m^{\frac{q+1}{2}} = m \cdot \left(\dfrac{m}{q}\right)$, так как $p, q \underset{4}{\equiv} 3$ по условию алгоритма.
		
	Таким образом $\pm  apm_q \pm bqm_p = \pm ap m \cdot \left(\dfrac{m}{q}\right) \pm bq m \cdot \left(\dfrac{m}{p}\right) = \pm apm \pm bqm = m (\pm ap \pm bq)$
	
	И тогда существует комбинация из знаков, такая что $B$ в итоге получает $m \cdot (ap + bq) \hm{\underset{pq}{\equiv}} m \cdot 1 = m$, что и требовалось доказать.
\end{solution}

\end{document}