\documentclass[12pt,a5paper,fleqn]{article}
\usepackage[utf8]{inputenc}
\usepackage{amssymb, amsmath, multicol}
\usepackage[russian]{babel}
\usepackage{graphicx}
\usepackage[shortcuts,cyremdash]{extdash}
\usepackage{wrapfig}
\usepackage{floatflt}
\usepackage{lipsum}
\usepackage{concmath}
\usepackage{euler}
\usepackage{algorithm}
\usepackage{algpseudocode} 

\graphicspath{ {images/} }

\oddsidemargin=-17.9mm
\textwidth=133mm
\headheight=-35.4mm
\textheight=200mm
\parindent=0pt
\tolerance=100
\parskip=6pt
\pagestyle{empty}
\renewcommand{\tg}{\mathop{\mathrm{tg}}\nolimits}
\renewcommand{\ctg}{\mathop{\mathrm{ctg}}\nolimits}
\renewcommand{\arctan}{\mathop{\mathrm{arctg}}\nolimits}
\newcommand{\divisible}{\mathop{\raisebox{-2pt}{\vdots}}}


\RequirePackage{caption2}
\renewcommand\captionlabeldelim{}
\newcommand*{\hm}[1]{#1\nobreak\discretionary{}%
\newtheorem{Theorem}{Теорема}
{\hbox{$\mathsurround=0pt #1$}}{}}

\begin{document}


\begin{center}
{ \Large Предпоследнее задание}

\end{center}

{\bf 1.} В протоколе $RSA$ выбраны $p = 17$, $q = 23$, $N=391$, $e=3$. Выберите ключ $d$ и зашифруйте сообщение $41$. Затем расшифруйте полученное сообщение и убедитесь, что получится исходное $41$.

\medskip

{\bf 2.} Пусть в протоколе $RSA$ открытый ключ $(N, e)$, $e=3$. Покажите, что если злоумышленник узнаёт закрытый ключ $d$, то он может легко найти разложение $N$ на множители.

\medskip

{\bf 3.} Схема $RSA$ позволяет также создавать защищенные электронные подписи. Если открытый ключ $(N, e)$, то автор сообщения, обладающий закрытым ключом $d$, отправляет сообщение $A^d$, где $A$~---~незашифрованное сообщение. После этого идентификация подписи - это возведение в степень $e$. Пусть открытый ключ $(2021, 25)$. В какую степень автору нужно возвести сообщение, чтобы отправить его за своей электронной подписью?

{\bf 4.} Решите уравнение $\varphi(n) = 6$, где $\varphi(n)$~---~функция Эйлера (количество чисел, не превосходящих $n$ и взаимно простых с ним).

{\bf 5.} Докажите, что в шифре Шамира в итоге у $B$ в действительности оказывается то сообщение, которое $A$ планировал передать.

{\bf 6.} Докажите, что в шифре Эль-Гамаля в итоге у $B$ в действительности оказывается то сообщение, которое $A$ планировал передать.

{\bf 7.} Докажите, что в алгоритме шифрования Рабина $B$ в итоге сможет найти исходное передаваемое сообщение среди $(\pm apm_q \pm bqm_p)$.

{\bf 8.} Рассмотрим алгоритм Sakurai и Takagi, напоминающий схему RSA. Выбирается модуль $N = pq$, где $p$ и $q$~---~достаточно большие простые. Открытым ключом является пара $(N, e)$, где $e$ взаимно просто с $\varphi(N)$, а закрытый ключ Алисы имеет вид $d = e^{-1} mod((p-1)(q-1))$. Для сообщения $m \in \mathbb{Z}_n$ выбирается случайно и равновероятно $r \in \mathbb{Z}_n^*$, зашифрованное сообщение имеет вид $c = f(r, m) = r^e(1+mn) \ (\mbox{mod } n^2)$. 

а) Докажите, что для всякого $c \in Z_{n^2}^*$ найдется единственный $r \in \mathbb{Z}_n^*$ и $m \in \mathbb{Z}_n$ такие, что $c = r^e(1+mn)  \ (\mbox{mod } n^2)$. 

б) Покажите, как Алисе расшифровывать полученное сообщение $c$.

в) Заметим, что $f(r, m) f(\rho, \mu) = f(r\rho, m+\mu)$. Пусть Алиса получила от Боба сообщение $f(r, m)$. Пусть Алиса достаточно доверчива и готова для Евы расшифровать произвольное зашифрованное ею сообщение. Покажите, как Еве действовать, чтобы выяснить $m$ по $f(r, m)$, если Алиса согласна расшифровать для Евы одно любое сообщение $f(\rho, \mu)$ (но, конечно, что-то заподозрит, если попросить её расшифровывать непосредственно $f(r, m)$, и делать это откажется).




\end{document}


