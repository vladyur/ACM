%%% Работа с русским языком
\usepackage[cm]{fullpage}
\usepackage[T2A]{fontenc}
\usepackage[utf8]{inputenc}
\usepackage[english,russian]{babel}
\usepackage{footmisc}
\usepackage[document]{ragged2e}
\usepackage{amsmath,amsfonts,amssymb,mathtools}
\usepackage{framed}
\usepackage{pstricks}
%\usepackage[framed]{ntheorem}
\usepackage{tikz}
\usetikzlibrary{arrows,automata}
\usepackage{cmap}					% поиск в PDF
\usepackage{mathtext} 				% русские буквы в формулах
\usepackage{indentfirst}
\frenchspacing

\newcommand{\vyp}{\hookrightarrow}
\renewcommand{\epsilon}{\varepsilon}
\renewcommand{\phi}{\varphi}
\renewcommand{\kappa}{\varkappa}
\renewcommand{\le}{\leqslant}
\renewcommand{\leq}{\leqslant}
\renewcommand{\ge}{\geqslant}
\renewcommand{\geq}{\geqslant}
\renewcommand{\emptyset}{\varnothing}

%%% Дополнительная работа с математикой
\usepackage{amsmath,amsfonts,amssymb,amsthm,mathtools} % AMS
\usepackage{icomma} % "Умная" запятая: $0,2$ --- число, $0, 2$ --- перечисление

%% Номера формул
%\mathtoolsset{showonlyrefs=true} % Показывать номера только у тех формул, на которые есть \eqref{} в тексте.
%\usepackage{leqno} % Нумереация формул слева

%% Свои команды
\DeclareMathOperator{\sgn}{\mathop{sgn}}

%% Перенос знаков в формулах (по Львовскому)
\newcommand*{\hm}[1]{#1\nobreak\discretionary{}
{\hbox{$\mathsurround=0pt #1$}}{}}



%%% Работа с картинками
\usepackage{graphicx}  % Для вставки рисунков
\graphicspath{{images/}{images2/}}  % папки с картинками
\setlength\fboxsep{3pt} % Отступ рамки \fbox{} от рисунка
\setlength\fboxrule{1pt} % Толщина линий рамки \fbox{}
\usepackage{wrapfig} % Обтекание рисунков текстом

%%% Работа с таблицами
\usepackage{array,tabularx,tabulary,booktabs} % Дополнительная работа с таблицами
\usepackage{longtable}  % Длинные таблицы
\usepackage{multirow} % Слияние строк в таблице

%%% Теоремы
\theoremstyle{plain} % Это стиль по умолчанию, его можно не переопределять.
\newtheorem{theorem}{Теорема}[section]
\newtheorem{proposition}[theorem]{Утверждение}
 
\theoremstyle{definition} % "Определение"
\newtheorem{corollary}{Следствие}[theorem]
\newtheorem{problem}{Задача}[section]
 
\theoremstyle{remark} % "Примечание"
\newtheorem*{nonum}{Решение}

%%% Программирование
\usepackage{etoolbox} % логические операторы

%%% Страница
\usepackage{extsizes} % Возможность сделать 14-й шрифт
\usepackage{geometry} % Простой способ задавать поля
\geometry{top=20mm}
\geometry{bottom=20mm}
\geometry{left=10mm}
\geometry{right=20mm}
 %
\usepackage{fancyhdr} % Колонтитулы
 	\pagestyle{fancy}
 	\renewcommand{\headrulewidth}{0pt}  % Толщина линейки, отчеркивающей верхний колонтитул
\fancypagestyle{firstpage}{
	\rhead{\large{Рябых Владислав, Б05-905}}
}
% 	\lfoot{Нижний левый}
% 	\rfoot{Нижний правый}
% 	\rhead{Верхний правый]}
% 	\chead{Верхний в центре}
% 	\lhead{Верхний левый}
%	\cfoot{Нижний в центре} % По умолчанию здесь номер страницы

\usepackage{setspace} % Интерлиньяж
\onehalfspacing % Интерлиньяж 1.5
%\doublespacing % Интерлиньяж 2
%\singlespacing % Интерлиньяж 1

\usepackage{lastpage} % Узнать, сколько всего страниц в документе.

\usepackage{soul} % Модификаторы начертания

\usepackage{hyperref}
%\usepackage[usenames,dvipsnames,svgnames,table,rgb]{xcolor}
\hypersetup{				% Гиперссылки
    unicode=true,           % русские буквы в раздела PDF
    pdftitle={Заголовок},   % Заголовок
    pdfauthor={Автор},      % Автор
    pdfsubject={Тема},      % Тема
    pdfcreator={Создатель}, % Создатель
    pdfproducer={Производитель}, % Производитель
    pdfkeywords={keyword1} {key2} {key3}, % Ключевые слова
    colorlinks=true,       	% false: ссылки в рамках; true: цветные ссылки
    linkcolor=red,          % внутренние ссылки
    citecolor=black,        % на библиографию
    filecolor=magenta,      % на файлы
    urlcolor=blue           % на URL
}

\usepackage{csquotes} % Еще инструменты для ссылок

%\usepackage[style=authoryear,maxcitenames=2,backend=biber,sorting=nty]{biblatex}

\usepackage{multicol} % Несколько колонок

\usepackage{pgfplots}
\usepackage{pgfplotstable}
\newcommand{\tbf}{\textbf}


\usepackage[shortlabels]{enumitem}

\newtheorem{task}{\textbf{Задача}}

\newtheorem{innercustomthm}{\textbf{Задача}}
\newenvironment{tasknum}[1]
{\renewcommand\theinnercustomthm{#1}\innercustomthm}
{\endinnercustomthm}

\newtheorem*{solution}{\textbf{Решение}}
\newcommand{\Ra}{\Rightarrow}
\newcommand{\La}{\Leftarrow}
\newcommand{\ra}{\rightarrow}
\newcommand{\LRa}{\Leftrightarrow}
\newcommand{\n}{\mathbb}
\newcommand{\Le}{\leqslant}
\newcommand{\Ge}{\geqslant}


\renewcommand{\inf}{\infty}
\newcommand{\ol}{\overline}

\newcommand{\bigline}{\noindent\makebox[\linewidth]{\rule{\paperwidth}{0.4pt}}}

\usetikzlibrary{fit}
\newcommand{\dost}{\overset{*}{\vdash}}
\newcommand{\vyv}{\overset{*}{\Rightarrow}}


\usepackage{capt-of}
\usepackage{tikz-qtree}
\usepackage{systeme}

\newcommand{\polysv}{\leq_p}
\def\coNP{{\mathbf{\text{\textbf{co--}}\mathcal{NP}}}}
\newcommand{\NP}{\mathcal{NP}}
\renewcommand{\P}{\mathcal{P}}

\usepackage{algorithm}
%\usepackage{algpseudocode}
\usepackage[noend]{algpseudocode}

\newcommand{\prob}[1]{\mathbb{P}\left\{#1\right\}}
\newcommand{\expected}[1]{\mathbb{E}\left\{#1\right\}}