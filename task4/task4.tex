\documentclass[12pt,a5paper,fleqn]{article}
\usepackage[utf8]{inputenc}
\usepackage{amssymb, amsmath, multicol}
\usepackage[russian]{babel}
\usepackage{graphicx}
\usepackage[shortcuts,cyremdash]{extdash}
\usepackage{wrapfig}
\usepackage{floatflt}
\usepackage{lipsum}
\usepackage{concmath}
\usepackage{euler}

\graphicspath{ {images/} }

\oddsidemargin=-17.9mm
\textwidth=133mm
\headheight=-35.4mm
\textheight=200mm
\parindent=0pt
\tolerance=100
\parskip=6pt
\pagestyle{empty}
\renewcommand{\tg}{\mathop{\mathrm{tg}}\nolimits}
\renewcommand{\ctg}{\mathop{\mathrm{ctg}}\nolimits}
\renewcommand{\arctan}{\mathop{\mathrm{arctg}}\nolimits}
\newcommand{\divisible}{\mathop{\raisebox{-2pt}{\vdots}}}
\newcommand{\modulo}{\mathop{\mathrm{mod}}\nolimits}
\def\coNP{{\mathbf{\text{\textbf{co--}}\mathcal{NP}}}}


\RequirePackage{caption2}
\renewcommand\captionlabeldelim{}
\newcommand*{\hm}[1]{#1\nobreak\discretionary{}%
\newtheorem{Theorem}{Теорема}
{\hbox{$\mathsurround=0pt #1$}}{}}

\begin{document}


\begin{center}
{ \Large Задание на четвертую неделю.}

\end{center}


{\bf 1.} Решите последнюю задачу предыдущего задания про $NP$-сертификат простоты.

\smallskip

{\bf 2.} Найдите $\Theta$-асимптотику суммы $\displaystyle\sum_{k=1}^n \sqrt{k}$, оценив её с помощью интеграла $\displaystyle\int_1^n \sqrt{x}dx$ сверху и снизу. Выведите аналогичную формулу для асимптотики $\displaystyle\sum_{k=1}^n k^\alpha$ для $\alpha > 0$.

\smallskip

{\bf 3 (0.5+0.5+1 балла).} а) Верно ли что язык $5$-ДНФ-Л является полиномиально полным в $\coNP$?

Язык $5$-ДНФ-Л состоит из всех формул в дизъюнктивной нормальной форме, принимающих истинное значение при каких-то значениях переменных, в каждый конъюнкт которых входит не более пяти переменных. 

б) Верно ли что язык $5$-КНФ-Л является полиномиально полным в $\mathcal{NP}$?

Язык $5$-КНФ-Л состоит из всех формул в конъюнктивной нормальной форме, принимающих  ложное значение при каких-то значениях переменных, в каждый дизъюнкт которых входит не более пяти переменных.

\noindent {\em Можно использовать гипотезы $\mathcal{P}\neq\mathcal{NP}$ и $\mathcal{NP}\neq \coNP$.}

\smallskip

в) Расставьте и обоснуйте $\mathcal{P}$, $\mathcal{NP}-complete$, $co-\mathcal{NP}-complete$: 
\begin{center}
\begin{tabular}{|c|c|c|}
\hline
 & Выполнимость & Тавтологичность \\
\hline
КНФ &  &  \\
 \hline
ДНФ  &   &  \\
\hline
\end{tabular}
\end{center}

Под выполнимостью понимается задача проверки наличия набора значений переменных, на котором формула равна $1$. Под тавтологичностью понимается задача проверки свойства формулы принимать значение $1$ на всех наборах.


\smallskip

{\bf 4 (2 балла).} Рассматривается язык $L$ выполнимых формул от $n$ переменных вида $C_1\land C_2\land\dotsc\land C_m$, где каждый $C_k$ имеет один из трех видов: $(x_i \equiv x_j)$, $(\overline{x_i}\equiv x_j)$, $(x_i \equiv \overline{x_j})$, $(\overline{x_i} \equiv \overline{x_j})$.

($i$) Верно ли, что этот язык $\mathcal{NP}$--полон?

($ii$) Верно ли, что если каждый $C_k$ будет иметь вид $(x_{i_1}^{\alpha_{i_1}} \equiv x_{i_2}^{\alpha_{i_2}} \equiv \dotsc \equiv x_{i_l}^{\alpha_{i_l}})$, то язык будет $\mathcal{NP}$--полон? (Под $x_i^{\alpha_i}$ понимается либо $x_i$, либо $\overline{x_i}$)

{\bf 5.} Останется ли $3-SAT$ полной, если ограничиться формулами, в которых каждая переменная входит не более $3$ раз, а каждый литерал~---~не более $2$ раз?

а) Под $3-SAT$ понимается НЕ-БОЛЕЕ-$3-SAT$.

б) {\bf (Бонусная задача)} Покажите, что если имеется в виду РОВНО-$3-SAT$, то не бывает невыполнимых формул указанного вида.

\smallskip

{\bf 6.} Постройте сводимость по Карпу языка $(G, k)$ графов, в которых есть $k$-клика к языку графов, в которых есть клика хотя бы на половине вершин.

\smallskip

{\bf 7 (Доп).} Пусть язык $L\in\mathcal{NP}$. Покажите, что он полиномиально сводится (по Карпу) к языку $STOP$ описаний пар $(M, \omega)$ машин Тьюринга и входов таких, что $M$ останавливается на входе $\omega$.





\end{document}


