\documentclass[a4paper,12pt]{article}

%%% Работа с русским языком
\usepackage[cm]{fullpage}
\usepackage[T2A]{fontenc}
\usepackage[utf8]{inputenc}
\usepackage[english,russian]{babel}
\usepackage{footmisc}
\usepackage[document]{ragged2e}
\usepackage{amsmath,amsfonts,amssymb,mathtools}
\usepackage{framed}
\usepackage{pstricks}
%\usepackage[framed]{ntheorem}
\usepackage{tikz}
\usetikzlibrary{arrows,automata}
\usepackage{cmap}					% поиск в PDF
\usepackage{mathtext} 				% русские буквы в формулах
\usepackage{indentfirst}
\frenchspacing

\newcommand{\vyp}{\hookrightarrow}
\renewcommand{\epsilon}{\varepsilon}
\renewcommand{\phi}{\varphi}
\renewcommand{\kappa}{\varkappa}
\renewcommand{\le}{\leqslant}
\renewcommand{\leq}{\leqslant}
\renewcommand{\ge}{\geqslant}
\renewcommand{\geq}{\geqslant}
\renewcommand{\emptyset}{\varnothing}

%%% Дополнительная работа с математикой
\usepackage{amsmath,amsfonts,amssymb,amsthm,mathtools} % AMS
\usepackage{icomma} % "Умная" запятая: $0,2$ --- число, $0, 2$ --- перечисление

%% Номера формул
%\mathtoolsset{showonlyrefs=true} % Показывать номера только у тех формул, на которые есть \eqref{} в тексте.
%\usepackage{leqno} % Нумереация формул слева

%% Свои команды
\DeclareMathOperator{\sgn}{\mathop{sgn}}

%% Перенос знаков в формулах (по Львовскому)
\newcommand*{\hm}[1]{#1\nobreak\discretionary{}
{\hbox{$\mathsurround=0pt #1$}}{}}



%%% Работа с картинками
\usepackage{graphicx}  % Для вставки рисунков
\graphicspath{{images/}{images2/}}  % папки с картинками
\setlength\fboxsep{3pt} % Отступ рамки \fbox{} от рисунка
\setlength\fboxrule{1pt} % Толщина линий рамки \fbox{}
\usepackage{wrapfig} % Обтекание рисунков текстом

%%% Работа с таблицами
\usepackage{array,tabularx,tabulary,booktabs} % Дополнительная работа с таблицами
\usepackage{longtable}  % Длинные таблицы
\usepackage{multirow} % Слияние строк в таблице

%%% Теоремы
\theoremstyle{plain} % Это стиль по умолчанию, его можно не переопределять.
\newtheorem{theorem}{Теорема}[section]
\newtheorem{proposition}[theorem]{Утверждение}
 
\theoremstyle{definition} % "Определение"
\newtheorem{corollary}{Следствие}[theorem]
\newtheorem{problem}{Задача}[section]
 
\theoremstyle{remark} % "Примечание"
\newtheorem*{nonum}{Решение}

%%% Программирование
\usepackage{etoolbox} % логические операторы

%%% Страница
\usepackage{extsizes} % Возможность сделать 14-й шрифт
\usepackage{geometry} % Простой способ задавать поля
\geometry{top=20mm}
\geometry{bottom=20mm}
\geometry{left=10mm}
\geometry{right=20mm}
 %
\usepackage{fancyhdr} % Колонтитулы
 	\pagestyle{fancy}
 	\renewcommand{\headrulewidth}{0pt}  % Толщина линейки, отчеркивающей верхний колонтитул
\fancypagestyle{firstpage}{
	\rhead{\large{Рябых Владислав, Б05-905}}
}
% 	\lfoot{Нижний левый}
% 	\rfoot{Нижний правый}
% 	\rhead{Верхний правый]}
% 	\chead{Верхний в центре}
% 	\lhead{Верхний левый}
%	\cfoot{Нижний в центре} % По умолчанию здесь номер страницы

\usepackage{setspace} % Интерлиньяж
\onehalfspacing % Интерлиньяж 1.5
%\doublespacing % Интерлиньяж 2
%\singlespacing % Интерлиньяж 1

\usepackage{lastpage} % Узнать, сколько всего страниц в документе.

\usepackage{soul} % Модификаторы начертания

%\usepackage{hyperref}
%\usepackage[usenames,dvipsnames,svgnames,table,rgb]{xcolor}
%\hypersetup{				% Гиперссылки
%    unicode=true,           % русские буквы в раздела PDF
%    pdftitle={Заголовок},   % Заголовок
%    pdfauthor={Автор},      % Автор
%    pdfsubject={Тема},      % Тема
%    pdfcreator={Создатель}, % Создатель
%    pdfproducer={Производитель}, % Производитель
%    pdfkeywords={keyword1} {key2} {key3}, % Ключевые слова
%    colorlinks=true,       	% false: ссылки в рамках; true: цветные ссылки
%    linkcolor=red,          % внутренние ссылки
%    citecolor=black,        % на библиографию
%    filecolor=magenta,      % на файлы
%    urlcolor=cyan           % на URL
%}

\usepackage{csquotes} % Еще инструменты для ссылок

%\usepackage[style=authoryear,maxcitenames=2,backend=biber,sorting=nty]{biblatex}

\usepackage{multicol} % Несколько колонок

\usepackage{pgfplots}
\usepackage{pgfplotstable}
\newcommand{\tbf}{\textbf}


\usepackage[shortlabels]{enumitem}

\newtheorem{task}{\textbf{Задача}}

\newtheorem{innercustomthm}{\textbf{Задача}}
\newenvironment{tasknum}[1]
{\renewcommand\theinnercustomthm{#1}\innercustomthm}
{\endinnercustomthm}

\newtheorem*{solution}{\textbf{Решение}}
\newcommand{\Ra}{\Rightarrow}
\newcommand{\La}{\Leftarrow}
\newcommand{\ra}{\rightarrow}
\newcommand{\LRa}{\Leftrightarrow}
\newcommand{\n}{\mathbb}
\newcommand{\Le}{\leqslant}
\newcommand{\Ge}{\geqslant}


\renewcommand{\inf}{\infty}
\newcommand{\ol}{\overline}

\newcommand{\bigline}{\noindent\makebox[\linewidth]{\rule{\paperwidth}{0.4pt}}}

\usetikzlibrary{fit}
\newcommand{\dost}{\overset{*}{\vdash}}
\newcommand{\vyv}{\overset{*}{\Rightarrow}}


\usepackage{capt-of}
\usepackage{tikz-qtree}
\usepackage{systeme}

\newcommand{\polysv}{\leq_p}
\def\coNP{{\mathbf{\text{\textbf{co--}}\mathcal{NP}}}}
\newcommand{\NP}{\mathcal{NP}}
\renewcommand{\P}{\mathcal{P}}

\usepackage{algorithm}
%\usepackage{algpseudocode}
\usepackage[noend]{algpseudocode}

\begin{document}
	
	\thispagestyle{firstpage}
	
	\begin{center}
		\textbf{\Large{Алгоритмы и модели вычислений. \\ Домашнее задание № 4}}
	\end{center}
	
\begin{tasknum}{1}
	Постройте NP-сертификат простоты числа $p = 3911$, $g = 13$. Простыми в рекурсивном построении считаются только числа $2$, $3$, $5$.
\end{tasknum}

\begin{solution}
	
	$g$ --- первообразный корень, является NP-сертификатом простоты числа $p = 3911$.
	
	$g^{p-1} = 13^{3910} \underset{3911}{\equiv} 1$ ~---~ по малой теореме Ферма.
	
	Решающей структурой будет дерево, в корне которого находится само число $p$ со своим сертификатом $g$, а в его детях ~---~ делители числа $p-1$ и их сертификаты соответственно (и так далее по рекурсии).
	
	$p-1 = 3910 = 23 \cdot 17 \cdot 5 \cdot 2$, раскладываем дальше:
	
	\begin{itemize}
		\item $2$, $5$ считаем простыми по условию задачи.
		\item $17$ имеет своим сертификатом $3$, $17-1=16=2^4 \Ra 17$ ~---~ простое.
		\item $23$ имеет своим сертификатом $5$, $23-1=22=11 \cdot 2$, таким образом либо $23$ и $11$ просты одновременно, либо одновременно являются составными.
		
		\begin{itemize}
			\item $11$ имеет своим сертификатом $2$: $11-1=10 = 5 \cdot 2 \Ra 11$ ~---~ простое.
		\end{itemize}
		
	\end{itemize}
	
	Таким образом, $p=3911$ является простым.
	
\end{solution}

\begin{tasknum}{2}
	Найдите $\Theta$-асимптотику суммы $\displaystyle\sum_{k=1}^n \sqrt{k}$, оценив её с помощью интеграла $\displaystyle\int_1^n \sqrt{x}dx$ сверху и снизу. Выведите аналогичную формулу для асимптотики $\displaystyle\sum_{k=1}^n k^\alpha$ для $\alpha > 0$.
\end{tasknum}

\begin{solution}
	
	
	$$\displaystyle\int_0^n \sqrt{x}dx \le \displaystyle\sum_{k=1}^n \displaystyle\int_{k-1}^{k} \sqrt{x}dx \le \displaystyle\sum_{k=1}^n \displaystyle\int_{k-1}^{k} \sqrt{k}dx = \displaystyle\sum_{k=1}^n \sqrt{k} = \displaystyle\sum_{k=1}^n \displaystyle\int_{k}^{k+1} \sqrt{k}dx \le \displaystyle\sum_{k=1}^n \displaystyle\int_{k}^{k+1} \sqrt{x}dx \le \displaystyle\int_1^{n+1} \sqrt{x}dx$$
	
	Таким образом получаем, что $\displaystyle\int_0^n \sqrt{x}dx \le \displaystyle\sum_{k=1}^n \sqrt{k} \le \displaystyle\int_1^{n+1} \sqrt{x}dx$
	
	То есть $\dfrac{2}{3} \cdot n^{\frac{3}{2}} \le \displaystyle\sum_{k=1}^n \sqrt{k} \le \dfrac{2}{3} \cdot ((n+1)^{\frac{3}{2}} - 1)$, таким образом $\displaystyle\sum_{k=1}^n \sqrt{k} = \Theta(n^{\frac{3}{2}})$
	
	Так как $\displaystyle\sum_{k=1}^n \sqrt{k} = \displaystyle\sum_{k=1}^n k^{\frac{1}{2}}$, проведём теперь аналогичные действия для произвольного $\alpha > 0$:
	
	$$\displaystyle\int_0^n x^{\alpha}dx \le \displaystyle\sum_{k=1}^n \displaystyle\int_{k-1}^{k} x^{\alpha}dx \le \displaystyle\sum_{k=1}^n \displaystyle\int_{k-1}^{k} k^{\alpha}dx = \displaystyle\sum_{k=1}^n k^{\alpha} = \displaystyle\sum_{k=1}^n \displaystyle\int_{k}^{k+1} k^{\alpha}dx \le \displaystyle\sum_{k=1}^n \displaystyle\int_{k}^{k+1} x^{\alpha}dx \le \displaystyle\int_1^{n+1} x^{\alpha}dx$$
	
	Таким образом получаем, что $\displaystyle\int_0^n x^{\alpha}dx \le \displaystyle\sum_{k=1}^n k^{\alpha} \le \displaystyle\int_1^{n+1} x^{\alpha}dx$
	
	То есть $\dfrac{1}{\alpha + 1} \cdot n^{\alpha + 1} \le \displaystyle\sum_{k=1}^n \sqrt{k} \le \dfrac{1}{\alpha + 1} \cdot ((n+1)^{\alpha + 1} - 1)$, таким образом $\displaystyle\sum_{k=1}^n k^{\alpha} = \Theta(n^{\alpha + 1})$
	
\end{solution}

\vspace{5mm}

\begin{tasknum}{3}
	а) Верно ли что язык $5$-ДНФ-Л является полиномиально полным в $\coNP$?
	
	Язык $5$-ДНФ-Л состоит из всех формул в дизъюнктивной нормальной форме, принимающих истинное значение при каких-то значениях переменных, в каждый конъюнкт которых входит не более пяти переменных. 
	
	б) Верно ли что язык $5$-КНФ-Л является полиномиально полным в $\mathcal{NP}$?
	
	Язык $5$-КНФ-Л состоит из всех формул в конъюнктивной нормальной форме, принимающих  ложное значение при каких-то значениях переменных, в каждый дизъюнкт которых входит не более пяти переменных.
	
	\noindent {\em Можно использовать гипотезы $\mathcal{P}\neq\mathcal{NP}$ и $\mathcal{NP}\neq \coNP$.}
	
	\smallskip
	
	в) Расставьте и обоснуйте $\mathcal{P}$, $\mathcal{NP}-complete$, $co-\mathcal{NP}-complete$: 
	\begin{center}
		\begin{tabular}{|c|c|c|}
			\hline
			& Выполнимость & Тавтологичность \\
			\hline
			КНФ &  &  \\
			\hline
			ДНФ  &   &  \\
			\hline
		\end{tabular}
	\end{center}
	
	Под выполнимостью понимается задача проверки наличия набора значений переменных, на котором формула равна $1$. Под тавтологичностью понимается задача проверки свойства формулы принимать значение $1$ на всех наборах.
\end{tasknum}

\begin{solution}
	
	\begin{enumerate}
		\item Пусть $\ol{A} =$ $5$-ДНФ-Л. Допустим, что $\ol{A}$ ~---~ полиномиально полный в $\coNP$, то есть $\forall L \in \NP \vyp \ol{L} \polysv \ol{A}$
		
		Рассмотрим язык $\ol{\ol{A}} = A$. $A \in \NP$, так как $\ol{A} \in \coNP$ по предположению. $A$ ~---~ язык, содержащий либо не ДНФ, либо те ДНФ, в которых есть конъюнкт с не менее 6 литералами, либо невыполнимые 5-ДНФ. Все эти 3 кластера мы легко можем проверить за полиномиальное время: первые два очевидно, а для проверки невыполнимости будем пользоваться следующим. ДНФ невыполнима $\LRa$ каждый из конъюнктов тождественно ложен, то есть в каждом конъюнкте присутствует некоторый литерал со своим отрицанием ~---~ мы можем легко это проверить для каждого из конъюнктов, которых в свою очередь $\le n$, где $n$ ~---~ длина формулы, то есть по сути длина входа. Таким образом выходит, что мы за полиномиальное время детерминированно проверяем принадлежность входа к $A$, то есть таким образом $A \in \P$. А это означает, что $\ol{A} \in \P$, а следовательно, так как $\forall L \in \NP \vyp \ol{L} \polysv \ol{A}$, то получаем, что $\forall L \in \NP \vyp \ol{L} \in \P \vyp \ol{\ol{L}}=L \in \P \Ra \P = \NP$, что неверно по гипотезе. Таким образом мы пришли к противоречию, так что язык $\ol{A} =$ $5$-ДНФ-Л не является полиномиально полным в $\coNP$.
		
		\item Пусть $A =$ $5$-КНФ-Л. Формула в КНФ принимает ложное значение $\LRa$ существует дизъюнкт, обращающийся в 0, то есть все литералы в дизъюнкте должны обратиться в 0. Соответственно, мы можем подобрать подходящий набор переменных тогда и только тогда, когда существует дизъюнкт, в котором не находятся одновременно литерал со своим отрицанием. Аналогично прошлому пункту, мы можем легко проверить наличие такого дизъюнкта, причём, так как всего их $\le n$, где $n$ ~---~ длина формулы, то есть по сути длина входа, выходит, что мы за полиномиальное время детерминированно проверяем принадлежность входа к $A$, то есть таким образом $A \in \P$. Поэтому $A$ не может быть $\NP$-полным, так как иначе каждый язык из $\NP$ сводился бы к $A \in P$, то есть язык сам лежал бы в $\P$, и тогда мы бы получили $\P = \NP$, что неверно по гипотезе.
		
		\item 
		
		\begin{enumerate}
			
			\item КНФ тавтологична тогда и только тогда, когда каждый из её дизъюнктов обращается в 1 на любом наборе переменных, то есть на любом наборе переменных невозможно ни один из дизъюнктов обратить в 0. Это же выполняется тогда и только тогда, когда каждый из дизъюнктов содержит некоторый литерал со своим отрицанием. Соответственно, проверить это мы можем за $O(m^2) = O(n^2)$, где $m \le n$ ~---~ длина наибольшего из дизъюнктов, $n$ ~---~ длина всей формулы. Всего таких дизъюнктов $\le n$, поэтому итоговая сложность алгоритма $O(n^3)$ ~---~ полиномиальна, так что проверка тавтологичности КНФ лежит в $\P$.
			
			\item Аналогично с выполнимостью ДНФ, только теперь нам надо проверять, что существует конъюнкт, в котором не содержится некоторого литерала с его отрицанием. Также это можно выполнить за $O(n^3)$, так что проверка выполнимости ДНФ лежит в $\P$.
			
			\item Язык выполнимых КНФ есть SAT ~---~ фундаментальный пример $\NP_c$ языка, полнота которого в $\NP$ следует из теоремы Кука-Левина.
			
			\item ДНФ тавтологична $\LRa$ её отрицание является невыполнимой КНФ. Задача о выполнимости КНФ является $\NP$-полной, поэтому задача о её невыполнимости является $\coNP$-полной.
			
		\end{enumerate}
		
		Таким образом, табличка примет следующий вид:
		 \begin{center}
			\begin{tabular}{|c|c|c|}
				\hline
				& Выполнимость & Тавтологичность \\
				\hline
				КНФ & $\NP_c$ & $\P$ \\
				\hline
				ДНФ  & $\P$  &  $\coNP_c$ \\
				\hline
			\end{tabular}
		\end{center}
		
	\end{enumerate}
	
	
\end{solution}

\end{document}