\documentclass[a4paper,12pt]{article}

%%% Работа с русским языком
\usepackage[cm]{fullpage}
\usepackage[T2A]{fontenc}
\usepackage[utf8]{inputenc}
\usepackage[english,russian]{babel}
\usepackage{footmisc}
\usepackage[document]{ragged2e}
\usepackage{amsmath,amsfonts,amssymb,mathtools}
\usepackage{framed}
\usepackage{pstricks}
%\usepackage[framed]{ntheorem}
\usepackage{tikz}
\usetikzlibrary{arrows,automata}
\usepackage{cmap}					% поиск в PDF
\usepackage{mathtext} 				% русские буквы в формулах
\usepackage{indentfirst}
\frenchspacing

\newcommand{\vyp}{\hookrightarrow}
\renewcommand{\epsilon}{\varepsilon}
\renewcommand{\phi}{\varphi}
\renewcommand{\kappa}{\varkappa}
\renewcommand{\le}{\leqslant}
\renewcommand{\leq}{\leqslant}
\renewcommand{\ge}{\geqslant}
\renewcommand{\geq}{\geqslant}
\renewcommand{\emptyset}{\varnothing}

%%% Дополнительная работа с математикой
\usepackage{amsmath,amsfonts,amssymb,amsthm,mathtools} % AMS
\usepackage{icomma} % "Умная" запятая: $0,2$ --- число, $0, 2$ --- перечисление

%% Номера формул
%\mathtoolsset{showonlyrefs=true} % Показывать номера только у тех формул, на которые есть \eqref{} в тексте.
%\usepackage{leqno} % Нумереация формул слева

%% Свои команды
\DeclareMathOperator{\sgn}{\mathop{sgn}}

%% Перенос знаков в формулах (по Львовскому)
\newcommand*{\hm}[1]{#1\nobreak\discretionary{}
{\hbox{$\mathsurround=0pt #1$}}{}}



%%% Работа с картинками
\usepackage{graphicx}  % Для вставки рисунков
\graphicspath{{images/}{images2/}}  % папки с картинками
\setlength\fboxsep{3pt} % Отступ рамки \fbox{} от рисунка
\setlength\fboxrule{1pt} % Толщина линий рамки \fbox{}
\usepackage{wrapfig} % Обтекание рисунков текстом

%%% Работа с таблицами
\usepackage{array,tabularx,tabulary,booktabs} % Дополнительная работа с таблицами
\usepackage{longtable}  % Длинные таблицы
\usepackage{multirow} % Слияние строк в таблице

%%% Теоремы
\theoremstyle{plain} % Это стиль по умолчанию, его можно не переопределять.
\newtheorem{theorem}{Теорема}[section]
\newtheorem{proposition}[theorem]{Утверждение}
 
\theoremstyle{definition} % "Определение"
\newtheorem{corollary}{Следствие}[theorem]
\newtheorem{problem}{Задача}[section]
 
\theoremstyle{remark} % "Примечание"
\newtheorem*{nonum}{Решение}

%%% Программирование
\usepackage{etoolbox} % логические операторы

%%% Страница
\usepackage{extsizes} % Возможность сделать 14-й шрифт
\usepackage{geometry} % Простой способ задавать поля
\geometry{top=20mm}
\geometry{bottom=20mm}
\geometry{left=10mm}
\geometry{right=20mm}
 %
\usepackage{fancyhdr} % Колонтитулы
 	\pagestyle{fancy}
 	\renewcommand{\headrulewidth}{0pt}  % Толщина линейки, отчеркивающей верхний колонтитул
\fancypagestyle{firstpage}{
	\rhead{\large{Рябых Владислав, Б05-905}}
}
% 	\lfoot{Нижний левый}
% 	\rfoot{Нижний правый}
% 	\rhead{Верхний правый]}
% 	\chead{Верхний в центре}
% 	\lhead{Верхний левый}
%	\cfoot{Нижний в центре} % По умолчанию здесь номер страницы

\usepackage{setspace} % Интерлиньяж
\onehalfspacing % Интерлиньяж 1.5
%\doublespacing % Интерлиньяж 2
%\singlespacing % Интерлиньяж 1

\usepackage{lastpage} % Узнать, сколько всего страниц в документе.

\usepackage{soul} % Модификаторы начертания

\usepackage{hyperref}
%\usepackage[usenames,dvipsnames,svgnames,table,rgb]{xcolor}
\hypersetup{				% Гиперссылки
    unicode=true,           % русские буквы в раздела PDF
    pdftitle={Заголовок},   % Заголовок
    pdfauthor={Автор},      % Автор
    pdfsubject={Тема},      % Тема
    pdfcreator={Создатель}, % Создатель
    pdfproducer={Производитель}, % Производитель
    pdfkeywords={keyword1} {key2} {key3}, % Ключевые слова
    colorlinks=true,       	% false: ссылки в рамках; true: цветные ссылки
    linkcolor=red,          % внутренние ссылки
    citecolor=black,        % на библиографию
    filecolor=magenta,      % на файлы
    urlcolor=blue           % на URL
}

\usepackage{csquotes} % Еще инструменты для ссылок

%\usepackage[style=authoryear,maxcitenames=2,backend=biber,sorting=nty]{biblatex}

\usepackage{multicol} % Несколько колонок

\usepackage{pgfplots}
\usepackage{pgfplotstable}
\newcommand{\tbf}{\textbf}


\usepackage[shortlabels]{enumitem}

\newtheorem{task}{\textbf{Задача}}

\newtheorem{innercustomthm}{\textbf{Задача}}
\newenvironment{tasknum}[1]
{\renewcommand\theinnercustomthm{#1}\innercustomthm}
{\endinnercustomthm}

\newtheorem*{solution}{\textbf{Решение}}
\newcommand{\Ra}{\Rightarrow}
\newcommand{\La}{\Leftarrow}
\newcommand{\ra}{\rightarrow}
\newcommand{\LRa}{\Leftrightarrow}
\newcommand{\n}{\mathbb}
\newcommand{\Le}{\leqslant}
\newcommand{\Ge}{\geqslant}


\renewcommand{\inf}{\infty}
\newcommand{\ol}{\overline}

\newcommand{\bigline}{\noindent\makebox[\linewidth]{\rule{\paperwidth}{0.4pt}}}

\usetikzlibrary{fit}
\newcommand{\dost}{\overset{*}{\vdash}}
\newcommand{\vyv}{\overset{*}{\Rightarrow}}


\usepackage{capt-of}
\usepackage{tikz-qtree}
\usepackage{systeme}

\newcommand{\polysv}{\leq_p}
\def\coNP{{\mathbf{\text{\textbf{co--}}\mathcal{NP}}}}
\newcommand{\NP}{\mathcal{NP}}
\renewcommand{\P}{\mathcal{P}}

\usepackage{algorithm}
%\usepackage{algpseudocode}
\usepackage[noend]{algpseudocode}

\newcommand{\prob}[1]{\mathbb{P}\left\{#1\right\}}
\newcommand{\expected}[1]{\mathbb{E}\left\{#1\right\}}

\begin{document}
	
	\thispagestyle{firstpage}
	
	\begin{center}
		\textbf{\Large{Алгоритмы и модели вычислений. \\ Домашнее задание № 3}}
	\end{center}
	
\begin{tasknum}{1}
	Постройте полиномиальную сводимость языка 3-ВЫПОЛНИМОСТЬ (3-SAT) (выполнимые КНФ, в каждом дизъюнкте не более 3 литералов) к языку РОВНО-3-ВЫПОЛНИМОСТЬ (выполнимые КНФ, в каждом дизъюнкте в точности $3$ литерала).
\end{tasknum}

\begin{solution}
	
	Необходимо построить сводимость по Карпу 3-SAT $\polysv$ Exactly-3-SAT. 
	
	Пусть $A$ ~---~ формула, представленная в виде КНФ, в каждом конъюнкте которой не более 3 литералов (если мы на вход получили что-то другого вида, тогда, очевидно, вход не будет лежать ни в 3-SAT, ни в Exactly-3-SAT и наша функция $f$ будет работать как тождественная). Если же на вход действительно поступила формула $A$, то преобразуем каждый её дизъюнкт так, чтобы получить в нём в точности $3$ литерала: просто продублируем первый литерал столько раз, чтобы <<добить>> данный дизъюнкт до 3 литералов. Например, дизъюнкт ($a \vee \ol{b}$) преобразуется в ($a \vee \ol{b} \vee a$), а дизъюнкт ($\ol{c}$) ~---~ в ($\ol{c} \vee \ol{c} \vee \ol{c}$). Очевидно, что $(x) \LRa (x \vee x)$, поэтому на выполнимость формулы данное преобразование никак не влияет. Формула $A$ преобразуется к $f(A)$ за полиномиальное от $|A|$ время, так как мы просто добавляем к входу $\le 4 \cdot |A|$ символов (максимально по 2 литерала и 2 знака дизъюнкции для каждого дизъюнкта $D$ (тут, в свою очередь, использовано, что $|D| \le |A|$)), поэтому описанная функция $f$ полиномиально сводит язык 3-SAT к языку Exactly-3-SAT.
	
\end{solution}

\begin{tasknum}{2}
	Постройте сводимость языка ВЫПОЛНИМОСТЬ (SAT) к языку ПРОТЫКАЮЩЕЕ МНОЖЕСТВО.
\end{tasknum}

\begin{solution}
	
	Описание конструкции см. в приложенном листке. Осталось доказать, что исходная
	КНФ $\phi(x_1, x_2, \ldots, x_n)$ выполнима тогда и только тогда, когда $A_\phi$ имеет протыкающее множество мощности $n$.
	
	\begin{itemize}
		\item $\Ra$: КНФ $\phi(\cdot)$ выполнима $\Ra$ существует набор значений переменных, на котором функция обращается в $1$. Тогда, если в этом наборе $x_i = 1$, то добавим в множество $A$ литерал $x_i$, если же $x_i = 0$, то добавляем в $A$ литерал $\ol{x_i}$ (по этому правилу будем далее отождествлять набор значений переменных и набор значений литералов). Делая так для всех переменных, получаем множество $A$ мощности $n$ такое, что если все литералы, лежащие в этом множестве, обращаются в $1$, то и функция $\phi$ на этом наборе обращается в $1$. 
		
		Покажем, что это множество $A$ и будет являться протыкающим. Действительно, каждое из подмножеств $A_\phi$ вида $A_i = \left\{x_i, \ol{x_i}\right\} \ \forall i \in \ol{1, n}$ будет пересекаться с $A$ по построению, а каждое подмножество вида $A_C$, состоящее из всех входящих в дизъюнкт $C$ литералов, будет пересекаться с $A$ в силу того, что на наборе литералов из $A$ КНФ $\phi$ выполнима, а следовательно каждый из её дизъюнктов обращается в $1$, так что в каждом подмножестве вида $A_C$ есть по крайней мере один литерал, принимающий на этом наборе значение, равное $1$, а по построению множество $A$ как раз и содержит все такие литералы, то есть этот литерал и будет лежать в пересечении $A_C$ с $A$.
		
		\item $\La$: $A_\phi$ имеет протыкающее множество $A$ мощности $n$. $A$ пересекает все подмножества вида $A_i = \left\{x_i, \ol{x_i}\right\} \ \forall i \in \ol{1, n}$, поэтому состоит из литералов всех $n$ переменных. $A$ пересекает все подмножества вида $A_C$, состоящие из всех входящих в дизъюнкт $C$ литералов, так что если рассмотреть набор переменных, на котором все литералы из $A$ обращаются $1$ (имеем право, так как $A$ содержит ровно $n$ литералов от $n$ различных переменных), то на этом наборе сама КНФ $\phi(\cdot)$ обратится в $1$, потому что каждый из дизъюнктов обратится в $1$ в силу того, что на этом наборе переменных он будет содержать по крайней мере один литерал (так как пересечение $A_C$ с $A$ непусто для любого дизъюнкта $C$), обращающийся в $1$.
		
	\end{itemize}
	
	
\end{solution}

\begin{tasknum}{3}
	Постройте сводимость языка 3-ВЫПОЛНИМОСТЬ к языку ВЕРШИННОЕ ПОКРЫТИЕ (VERTEX-COVER).
\end{tasknum}

\begin{solution}
	
	Ранее было доказано, что 3-SAT $\polysv$ Exactly-3-SAT, в этом номере покажем, что Exactly-3-SAT $\polysv$ VERTEX-COVER и тогда получим 3-SAT $\polysv$ VERTEX-COVER по транзитивности.
	
	Описание конструкции см. в приложенном листке. Осталось доказать, что исходная
	КНФ $\phi(x_1, x_2, \ldots, x_n)$ выполнима тогда и только тогда, когда $G_\phi$ имеет вершинное покрытие $V$ мощности $n+2m$, где $m$ ~---~ количество дизъюнктов.
	
	\begin{itemize}
		\item $\Ra$: сначала определимся с выбором $n$ литеральных вершин. КНФ $\phi(\cdot)$ выполнима $\Ra$ существует набор значений переменных, на котором функция обращается в $1$. Тогда, если в этом наборе $x_i = 1$, то добавим в множество вершин $V$ литеральную вершину $x_i$, если же $x_i = 0$, то добавляем в $V$ литеральную вершину $\ol{x_i}$. Таким образом мы гарантируем, что в каждой паре литеральных вершин будет одна вершина, лежащая в $V$, а следовательно, рёбра между литеральными вершинами будут захвачены нашим вершинным покрытием $V$.
		
		 Теперь, так как КНФ $\phi$ выполнима (а следовательно каждый из её дизъюнктов обращается в $1$), в каждой из троек дизъюнктных вершин есть по крайней мере один литерал, принимающий на этом наборе значение, равное $1$ ~---~ он по построению соединён с уже лежащей в $V$ соответствующей литеральной вершиной, так что добавим в $V$ не эту дизъюнктную вершину, а другие две дизъюнктные, смежные ей. Таким образом мы гарантируем, что во-первых, из каждой тройки дизъюнктных вершин две будут лежать в $V$, а следовательно, покрытие захватит все рёбра каждой из дизъюнктных троек; а во-вторых, что две другие вершины, литералы которых могут быть не равными $1$, будут соединены с соответствующими им литеральными вершинами (иначе могли бы остаться непокрытые рёбра). Получаем, что наше вершинное покрытие $V$ захватит последние оставшиеся рёбра между невзятыми ранее литеральными вершинами и соответствующими им дизъюнктивными.
		 
		 \item $\La$: $V$ --- вершинное покрытие $\Ra$ оно содержит как минимум одну литеральную вершину в каждой паре соответствующих (иначе ребро между ними не было бы покрыто). Таким образом оно содержит $\ge n$ литеральных вершин. Заметим теперь, что $V$ содержит как минимум две дизъюнктных вершины в каждой тройке соответствующих (иначе в этом треугольнике нашлось бы непокрытое ребро). Таким образом оно содержит $\ge 2m$ дизъюнктных вершин. По условию же $|V| = n + 2m$, следовательно в вершинном покрытии находятся ровно $n$ литеральных и ровно $2m$ дизъюнктных вершин.
		 
		 Возьмём набор литералов, соответствующий набору литеральных вершин $V$. На этом наборе наша функция $\phi$ и будет обращаться в $1$ (если в $V$ лежит литеральная вершина $x_i$, то в нашем наборе переменных берём $x_i = 1$, если же в $V$ лежит $\ol{x_i}$ ~---~ берём $x_i = 0$). Докажем это от противного: допустим, что на этом наборе существует дизъюнкт, обращающийся в $0$, то есть существует тройка дизъюнктных вершин, все 3 литерала которой обращаются в $0$. Такого быть не может, так как в $V$ лежат ровно 2 дизъюнктные вершины из каждой тройки. То есть в этой тройке существует вершина, литерал которой обращается в 0 и которая не лежит в $V$ сама по себе ~---~ это противоречие к тому, что $V$ является вершинным покрытием, потому что ребро между этой вершиной и соответствующей ей литеральной (не лежит в $V$, так как все литералы, соответствующие литеральным вершинам из $V$, обращаются в 1 на этом наборе) не будет покрыто.
		  
	\end{itemize}
	
\end{solution}

\begin{tasknum}{4}
	Постройте сводимость языка РОВНО-3-ВЫПОЛНИМОСТЬ к языку КЛИКА (CLIQUE).
\end{tasknum}

\begin{solution}
	
	Описание конструкции см. в приложенном листке. Осталось доказать, что исходная
	КНФ $\phi(x_1, x_2, \ldots, x_n)$ выполнима тогда и только тогда, когда $G_\phi$ имеет клику размера $m$, где $m$ ~---~ количество дизъюнктов.
	
	\begin{itemize}
		\item $\Ra$: КНФ $\phi(\cdot)$ выполнима $\Ra$ существует набор значений переменных, на котором функция обращается в $1$, то есть на этом наборе в каждом дизъюнкте присутствует литерал, обращающийся в 1. Таким образом мы можем в каждом из дизъюнктов взять такой литерал и соединить их все вместе (такие рёбра есть, так как, очевидно, отвечающие этим литералам переменные, никак не могут быть отрицанием друг друга). Получим искомую клику размера $m$.
		
		\item $\La$: в нашем графе есть клика размера $m$. Так как вершины одного дизъюнкта не смежны между собой, то каждый дизъюнкт в клике представляет не более одной вершины. Всего дизъюнктов $m \Ra$ на самом деле каждый дизъюнкт представлен в клике ровно одной вершиной. Теперь возьмём следующий набор переменных: если в клике есть вершина, отвечающая литералу $x_i$, то в нашем наборе переменных берём $x_i = 1$, если же в клике лежит $\ol{x_i}$ ~---~ берём $x_i = 0$. Если же в клике вообще нет вершины, отвечающей переменной $x_i$ ~---~ положим для определённости $x_i = 1$ (хотя это ни на что не влияет, так как на самом деле $x_i$ ~---~ фиктивная переменная). Таким образом построенный набор переменных задаётся непротиворечиво, так как по условию наличия рёбер между вершинами, не может быть такого, что и литерал $x_i$, и литерал $\ol{x_i}$ присутствуют в клике. На этом наборе КНФ $\phi(\cdot)$ обращается в 1, так как в каждом дизъюнкте присутствует литерал, обращающийся в 1.
	\end{itemize}
	
\end{solution}

\begin{tasknum}{7}
	Покажите, что построенная на семинаре при сводимости 3-CNF к CIRCUIT-SAT формула равновыполнима с исходной. 
\end{tasknum}

\begin{solution}
	
	Докажем, что формула выполнима $\LRa$ выполнима схема:
	
	\begin{itemize}
		\item $\Ra$: формула выполнима $\Ra$ существует набор переменных, на котором все дизъюнкты обращаются в 1 $\Ra$ все эквивалентности верны, а также $y_m = 1$. Но таким образом мы можем <<развернуть>> выражение для $y_m$ через переменные $x_1, x_2, \ldots, x_n$ (мы имеем право так сделать, эти преобразования корректны как раз из-за того, что все эквивалентности выполняются). Таким образом $y_m$ представляется, как формула от $y_m = f(x_1, x_2, \ldots, x_n)$, равновыполнимая с исходной схемой. Таким образом, так как на данном наборе переменных $f(x_1, x_2, \ldots, x_n) = y_m = 1$, то исходная схема также выполнима.
		
		\item $\La$: схема выполнима $\Ra$ существует набор переменных, на котором $y_m = 1$. Таким образом получаем, что в нашей формуле все эквивалентности обращаются в $1$ по построению, а в силу того, что и $y_m=1$, все дизъюнкты обращаются в 1 $\Ra$ сама формула также обращается в $1$.
	\end{itemize}
	
\end{solution}

\end{document}