\documentclass[12pt,a5paper,fleqn]{article}
\usepackage[utf8]{inputenc}
\usepackage{amssymb, amsmath, multicol}
\usepackage[russian]{babel}
\usepackage{graphicx}
\usepackage[shortcuts,cyremdash]{extdash}
\usepackage{wrapfig}
\usepackage{floatflt}
\usepackage{lipsum}
\usepackage{concmath}
\usepackage{euler}

\graphicspath{ {images/} }

\oddsidemargin=-17.9mm
\textwidth=133mm
\headheight=-35.4mm
\textheight=200mm
\parindent=0pt
\tolerance=100
\parskip=6pt
\pagestyle{empty}
\renewcommand{\tg}{\mathop{\mathrm{tg}}\nolimits}
\renewcommand{\ctg}{\mathop{\mathrm{ctg}}\nolimits}
\renewcommand{\arctan}{\mathop{\mathrm{arctg}}\nolimits}
\newcommand{\divisible}{\mathop{\raisebox{-2pt}{\vdots}}}
\renewcommand{\not}{\overline}
\newcommand{\modulo}{\mathop{\mathrm{mod}}\nolimits}


\RequirePackage{caption2}
\renewcommand\captionlabeldelim{}
\newcommand*{\hm}[1]{#1\nobreak\discretionary{}%
\newtheorem{Theorem}{Теорема}
{\hbox{$\mathsurround=0pt #1$}}{}}

\begin{document}


\begin{center}
{ \Large Задание на третью неделю.}

\textit{Описание конструкций большинства сводимостей см. в отдельном выложенном файле (задачи 33-38). В этом задании нужно обосновать описанные там конструкции.}

\end{center}



{\bf 1.} Постройте полиномиальную сводимость языка 3-ВЫПОЛНИМОСТЬ (3-SAT) (выполнимые КНФ, в каждом конъюнкте не более 3 литералов) к языку РОВНО-3-ВЫПОЛНИМОСТЬ (выполнимые КНФ, в каждом конъюнкте в точности $3$ литерала.

\smallskip

{\bf 2.} Постройте сводимость языка ВЫПОЛНИМОСТЬ (SAT) к языку ПРОТЫКАЮЩЕЕ МНОЖЕСТВО. Перед этим проделайте пункты $(i)$ и $(ii)$ задачи $34$ для иллюстрации и понимания происходящего.

\smallskip

{\bf 3.} Постройте сводимость языка 3-ВЫПОЛНИМОСТЬ к языку ВЕРШИННОЕ ПОКРЫТИЕ (VERTEX-COVER). Перед этим проделайте пункты $(i)$ и $(ii)$ задачи $35$ для иллюстрации и понимания происходящего.

\smallskip

{\bf 4.} Постройте сводимость языка РОВНО-3-ВЫПОЛНИМОСТЬ к языку КЛИКА (CLIQUE). Перед этим проделайте пункты $(i)$ и $(ii)$ задачи $36$ для иллюстрации и понимания происходящего.

\smallskip

{\bf 5.} Постройте сводимость языка 3-ВЫПОЛНИМОСТЬ к языку max-2-ВЫПОЛНИМОСТЬ (max-2-SAT). Перед этим проделайте пункты $(i)$ и $(ii)$ задачи $37$ для иллюстрации и понимания происходящего.

\smallskip

{\bf 6.} Покажите, что если $3-COLOR$ лежит в $\mathcal{P}$, то за полиномиальное время можно не только определить, допускает ли граф раскраску, но и найти саму эту раскраску (если она есть). Обратите внимание, что на вход проверяющей 3-раскрашиваемость процедуры нельзя подавать частично окрашенные графы, спрашивая, можно ли дораскрасить оставшиеся вершины до полной правильной раскраски.

\smallskip

{\bf 7.} Покажите, что построенная при сводимости 3-CNF к CIRCUIT-SAT формула равновыполнима с исходной. 

\smallskip

{\bf 8.} Постройте NP-сертификат простоты числа $p = 3911$, $g = 13$. Простыми в рекурсивном построении считаются только числа $2$, $3$, $5$.

\smallskip

\newpage


\begin{center}
{ \Large Задание на третью неделю.}

\textit{Описание конструкций большинства сводимостей см. в отдельном выложенном файле (задачи 33-38). В этом задании нужно обосновать описанные там конструкции.}

\end{center}



{\bf 1.} Постройте полиномиальную сводимость языка 3-ВЫПОЛНИМОСТЬ (3-SAT) (выполнимые КНФ, в каждом конъюнкте не более 3 литералов) к языку РОВНО-3-ВЫПОЛНИМОСТЬ (выполнимые КНФ, в каждом конъюнкте в точности $3$ литерала.

\smallskip

{\bf 2.} Постройте сводимость языка ВЫПОЛНИМОСТЬ (SAT) к языку ПРОТЫКАЮЩЕЕ МНОЖЕСТВО. Перед этим проделайте пункты $(i)$ и $(ii)$ задачи $34$ для иллюстрации и понимания происходящего.

\smallskip

{\bf 3.} Постройте сводимость языка 3-ВЫПОЛНИМОСТЬ к языку ВЕРШИННОЕ ПОКРЫТИЕ (VERTEX-COVER). Перед этим проделайте пункты $(i)$ и $(ii)$ задачи $35$ для иллюстрации и понимания происходящего.

\smallskip

{\bf 4.} Постройте сводимость языка РОВНО-3-ВЫПОЛНИМОСТЬ к языку КЛИКА (CLIQUE). Перед этим проделайте пункты $(i)$ и $(ii)$ задачи $36$ для иллюстрации и понимания происходящего.

\smallskip

{\bf 5.} Постройте сводимость языка 3-ВЫПОЛНИМОСТЬ к языку max-2-ВЫПОЛНИМОСТЬ (max-2-SAT). Перед этим проделайте пункты $(i)$ и $(ii)$ задачи $37$ для иллюстрации и понимания происходящего.

\smallskip

{\bf 6.} Покажите, что если $3-COLOR$ лежит в $\mathcal{P}$, то за полиномиальное время можно не только определить, допускает ли граф раскраску, но и найти саму эту раскраску (если она есть). Обратите внимание, что на вход проверяющей 3-раскрашиваемость процедуры нельзя подавать частично окрашенные графы, спрашивая, можно ли дораскрасить оставшиеся вершины до полной правильной раскраски.

\smallskip

{\bf 7.} Покажите, что построенная при сводимости 3-CNF к CIRCUIT-SAT формула равновыполнима с исходной. 

\smallskip

{\bf 8.} Постройте NP-сертификат простоты числа $p = 3911$, $g = 13$. Простыми в рекурсивном построении считаются только числа $2$, $3$, $5$.

\smallskip

\end{document}


