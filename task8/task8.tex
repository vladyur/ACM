\documentclass[12pt,a5paper,fleqn]{article}
\usepackage[utf8]{inputenc}
\usepackage{amssymb, amsmath, multicol}
\usepackage[russian]{babel}
\usepackage{graphicx}
\usepackage[shortcuts,cyremdash]{extdash}
\usepackage{wrapfig}
\usepackage{floatflt}
\usepackage{lipsum}
\usepackage{concmath}
\usepackage{euler}
\usepackage{algorithm}
\usepackage{algpseudocode} 

\graphicspath{ {images/} }

\oddsidemargin=-17.9mm
\textwidth=133mm
\headheight=-35.4mm
\textheight=200mm
\parindent=0pt
\tolerance=100
\parskip=6pt
\pagestyle{empty}
\renewcommand{\tg}{\mathop{\mathrm{tg}}\nolimits}
\renewcommand{\ctg}{\mathop{\mathrm{ctg}}\nolimits}
\renewcommand{\arctan}{\mathop{\mathrm{arctg}}\nolimits}
\newcommand{\divisible}{\mathop{\raisebox{-2pt}{\vdots}}}


\RequirePackage{caption2}
\renewcommand\captionlabeldelim{}
\newcommand*{\hm}[1]{#1\nobreak\discretionary{}%
\newtheorem{Theorem}{Теорема}
{\hbox{$\mathsurround=0pt #1$}}{}}

\begin{document}


\begin{center}
{ \Large Задание на восьмую неделю.}

\end{center}

{\bf 1.} Подбрасываем <<честную>> монету $10$ раз. Подсчитайте вероятности следующих событий:

($i$) (1/6 балла)  число выпавших <<орлов>>  равно числу <<решек>>;

($ii$) (1/6 балла)  выпало больше <<орлов>>  чем <<решек>>;

($iii$) (1/6 балла) при $i=1,\dots,5$ одинаковы результаты $i$-го и $11-i$-го бросаний;

($iv$) (1/2 балла) <<орел>> выпал не менее четырех раз подряд. 

\smallskip

{\bf 2.} ($i$) Вычислите условную вероятность, что при бросаний двух игральных костей на первой выпало шесть, если сумма равна семи.

($ii$) При двух бросках игральной кости выпало $X_1$ и $X_2$, соответственно. Вычислите $\mathbb{E}\{\max\{X_1,X_2\}\} + \mathbb{E}\{\min\{X_1,X_2\}\}$.

($iii$) Покажите, что из попарной независимости случайных величин не следует независимость в совокупности. Приведите контрпример.

($iv$) Независимы ли события: <<при броске кубика выпало четное число>> и <<при броске кубика выпало число, кратное трём>>?

($v$) Найти вероятность, что случайно выбранный граф на $n$ вершинах является простым циклом; найти её предел при $n\rightarrow \infty$.

\smallskip

{\bf 3 (ВТФ).} Две урны содержат одинаковое количество шаров. Шары окрашены в белый и черный цвета. Из каждой урны вынимают по $n$ шаров с возвращением, где $n \geq 3$. Найдите $n$ и <<состав>> каждой урны, если вероятность того, что все шары, взятые из первой урны, белые, равна вероятности того, что все шары, взятые из второй урны, либо белые, либо черные.

\smallskip

{\bf 4.} Симметричную монетку бросают неограниченное число раз. Какая из последовательностей встретится раньше с большей вероятностью: РОР или РРО?

\smallskip

{\bf 5.} ($i$) Найти мат. ожидание числа простых циклов длины $r$ в случайном графе на $n$ вершинах. Любое из $C_n^2$ рёбер генерируется независимо от других с вероятностью $p$.

($ii$) Найти мат. ожидание числа простых циклов длины $r$ в случайной перестановке $n$ элементов в предположении, что все перестановки $\pi \in S_n$ равновероятны.

\smallskip

{\bf 6.} ($i$) Имеется генератор случайных битов, выдающий $0$ и $1$ с вероятностью $1/2$. Предложите алгоритм, использующий этот генератор и выдающий $0$ с вероятностью $1/3$ и $1$ с вероятностью $2/3$. Оцените его время работы в лучшем и в худшем случае.

($ii$) Обратно: из генератора $(1/3; 2/3)$ получите $(1/2)$.

\smallskip

{\bf 7.} ($i$) Найти мат. ожидание количества неподвижных элементов в случайно выбранной из $S_n$ перестановке.   

($ii$) Найти математическое ожидание числа бросаний кости до первого выпадения двух шестерок подряд. 

\smallskip

{\bf 8.} В экзаменационной программе обычного экзамена $25$ билетов, из которых $5$ простые, а вытянув любой из остальных, всякий студент точно завалит экзамен. Подряд заходят два студента. Какой из них с большей вероятностью вытянет простой билет?

\smallskip

{\bf 9.} Рассмотрим следующую вероятностную процедуру, на вход которой поступает массив из $n$ различных чисел $A[1..n]$. Внутри процедуры используется генератор случайных чисел $\textsc{Rand}(1,2,\ldots,n)$, который возвращает случайно и равновероятно число $j$ из множества $\{1,2,\ldots,n\}$.

\begin{algorithmic}[1]
    \Procedure{RandProcedure}{$A[1..n]$, $n$}
    \State Задать массив $C[1..n] := A[1..n]$
    \State Задать массив $B[1..n] := \{\textsc{False},\textsc{False},\ldots,\textsc{False}\}$
    \State Задать $i := 1$
    \While{$i < n+1$}
    \State $j := \textsc{Rand}(1,2,\ldots,n)$
    \If{$B[j] = \textsc{False}$}
    \State Задать $C[i] := A[j]$
    \State Задать $i : = i+1$
    \State Задать $B[j] := \textsc{True}$
    \EndIf
    \EndWhile
    \State \Return $C[1..n]$
    \EndProcedure
\end{algorithmic}

\medskip

($i$) Чему равен супремум чисел $k$, для которых вероятность события, что алгоритм сделает хотя бы $k$ итераций цикла {\bf while} положительна?  
 
($ii$) Докажите, что представленный алгоритм выдаёт некоторую перестановку массива $A$, и вычислите вероятность получения каждой конкретной перестановки.

($iii$) Сколько в среднем раз будет выполнена строчка $6$ в описанной выше процедуре?

\smallskip

{\bf 10 (Доп).}  На окружности случайным образом выбираются две точки. Найдите среднее расстояние между ними.





\end{document}


