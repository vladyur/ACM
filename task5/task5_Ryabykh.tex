\documentclass[a4paper,12pt]{article}

%%% Работа с русским языком
\usepackage[cm]{fullpage}
\usepackage[T2A]{fontenc}
\usepackage[utf8]{inputenc}
\usepackage[english,russian]{babel}
\usepackage{footmisc}
\usepackage[document]{ragged2e}
\usepackage{amsmath,amsfonts,amssymb,mathtools}
\usepackage{framed}
\usepackage{pstricks}
%\usepackage[framed]{ntheorem}
\usepackage{tikz}
\usetikzlibrary{arrows,automata}
\usepackage{cmap}					% поиск в PDF
\usepackage{mathtext} 				% русские буквы в формулах
\usepackage{indentfirst}
\frenchspacing

\newcommand{\vyp}{\hookrightarrow}
\renewcommand{\epsilon}{\varepsilon}
\renewcommand{\phi}{\varphi}
\renewcommand{\kappa}{\varkappa}
\renewcommand{\le}{\leqslant}
\renewcommand{\leq}{\leqslant}
\renewcommand{\ge}{\geqslant}
\renewcommand{\geq}{\geqslant}
\renewcommand{\emptyset}{\varnothing}

%%% Дополнительная работа с математикой
\usepackage{amsmath,amsfonts,amssymb,amsthm,mathtools} % AMS
\usepackage{icomma} % "Умная" запятая: $0,2$ --- число, $0, 2$ --- перечисление

%% Номера формул
%\mathtoolsset{showonlyrefs=true} % Показывать номера только у тех формул, на которые есть \eqref{} в тексте.
%\usepackage{leqno} % Нумереация формул слева

%% Свои команды
\DeclareMathOperator{\sgn}{\mathop{sgn}}

%% Перенос знаков в формулах (по Львовскому)
\newcommand*{\hm}[1]{#1\nobreak\discretionary{}
{\hbox{$\mathsurround=0pt #1$}}{}}



%%% Работа с картинками
\usepackage{graphicx}  % Для вставки рисунков
\graphicspath{{images/}{images2/}}  % папки с картинками
\setlength\fboxsep{3pt} % Отступ рамки \fbox{} от рисунка
\setlength\fboxrule{1pt} % Толщина линий рамки \fbox{}
\usepackage{wrapfig} % Обтекание рисунков текстом

%%% Работа с таблицами
\usepackage{array,tabularx,tabulary,booktabs} % Дополнительная работа с таблицами
\usepackage{longtable}  % Длинные таблицы
\usepackage{multirow} % Слияние строк в таблице

%%% Теоремы
\theoremstyle{plain} % Это стиль по умолчанию, его можно не переопределять.
\newtheorem{theorem}{Теорема}[section]
\newtheorem{proposition}[theorem]{Утверждение}
 
\theoremstyle{definition} % "Определение"
\newtheorem{corollary}{Следствие}[theorem]
\newtheorem{problem}{Задача}[section]
 
\theoremstyle{remark} % "Примечание"
\newtheorem*{nonum}{Решение}

%%% Программирование
\usepackage{etoolbox} % логические операторы

%%% Страница
\usepackage{extsizes} % Возможность сделать 14-й шрифт
\usepackage{geometry} % Простой способ задавать поля
\geometry{top=20mm}
\geometry{bottom=20mm}
\geometry{left=10mm}
\geometry{right=20mm}
 %
\usepackage{fancyhdr} % Колонтитулы
 	\pagestyle{fancy}
 	\renewcommand{\headrulewidth}{0pt}  % Толщина линейки, отчеркивающей верхний колонтитул
\fancypagestyle{firstpage}{
	\rhead{\large{Рябых Владислав, Б05-905}}
}
% 	\lfoot{Нижний левый}
% 	\rfoot{Нижний правый}
% 	\rhead{Верхний правый]}
% 	\chead{Верхний в центре}
% 	\lhead{Верхний левый}
%	\cfoot{Нижний в центре} % По умолчанию здесь номер страницы

\usepackage{setspace} % Интерлиньяж
\onehalfspacing % Интерлиньяж 1.5
%\doublespacing % Интерлиньяж 2
%\singlespacing % Интерлиньяж 1

\usepackage{lastpage} % Узнать, сколько всего страниц в документе.

\usepackage{soul} % Модификаторы начертания

%\usepackage{hyperref}
%\usepackage[usenames,dvipsnames,svgnames,table,rgb]{xcolor}
%\hypersetup{				% Гиперссылки
%    unicode=true,           % русские буквы в раздела PDF
%    pdftitle={Заголовок},   % Заголовок
%    pdfauthor={Автор},      % Автор
%    pdfsubject={Тема},      % Тема
%    pdfcreator={Создатель}, % Создатель
%    pdfproducer={Производитель}, % Производитель
%    pdfkeywords={keyword1} {key2} {key3}, % Ключевые слова
%    colorlinks=true,       	% false: ссылки в рамках; true: цветные ссылки
%    linkcolor=red,          % внутренние ссылки
%    citecolor=black,        % на библиографию
%    filecolor=magenta,      % на файлы
%    urlcolor=cyan           % на URL
%}

\usepackage{csquotes} % Еще инструменты для ссылок

%\usepackage[style=authoryear,maxcitenames=2,backend=biber,sorting=nty]{biblatex}

\usepackage{multicol} % Несколько колонок

\usepackage{pgfplots}
\usepackage{pgfplotstable}
\newcommand{\tbf}{\textbf}


\usepackage[shortlabels]{enumitem}

\newtheorem{task}{\textbf{Задача}}

\newtheorem{innercustomthm}{\textbf{Задача}}
\newenvironment{tasknum}[1]
{\renewcommand\theinnercustomthm{#1}\innercustomthm}
{\endinnercustomthm}

\newtheorem*{solution}{\textbf{Решение}}
\newcommand{\Ra}{\Rightarrow}
\newcommand{\La}{\Leftarrow}
\newcommand{\ra}{\rightarrow}
\newcommand{\LRa}{\Leftrightarrow}
\newcommand{\n}{\mathbb}
\newcommand{\Le}{\leqslant}
\newcommand{\Ge}{\geqslant}


\renewcommand{\inf}{\infty}
\newcommand{\ol}{\overline}

\newcommand{\bigline}{\noindent\makebox[\linewidth]{\rule{\paperwidth}{0.4pt}}}

\usetikzlibrary{fit}
\newcommand{\dost}{\overset{*}{\vdash}}
\newcommand{\vyv}{\overset{*}{\Rightarrow}}


\usepackage{capt-of}
\usepackage{tikz-qtree}
\usepackage{systeme}

\newcommand{\polysv}{\leq_p}
\def\coNP{{\mathbf{\text{\textbf{co--}}\mathcal{NP}}}}
\newcommand{\NP}{\mathcal{NP}}
\renewcommand{\P}{\mathcal{P}}

\usepackage{algorithm}
%\usepackage{algpseudocode}
\usepackage[noend]{algpseudocode}

\begin{document}
	
	\thispagestyle{firstpage}
	
	\begin{center}
		\textbf{\Large{Алгоритмы и модели вычислений. \\ Домашнее задание № 5}}
	\end{center}
	
\begin{tasknum}{1}
	($i$) Докажите, что в $\Sigma_2$ лежит язык булевых формул от двух наборов переменных $\varphi(x_1,\dotsc, x_n, y_1\dotsc y_n) = \varphi(\vec{x}, \vec{y})$ таких, что при некоторых значениях $\vec{x}$ они справедливы вне зависимости от значений $y_1,\dotsc, y_n$.
	
	($ii$) Придумайте какую-нибудь свою задачу из класса $\Sigma_3$ (или $\Pi_3$, на ваш вкус).
	
	($iii$) Докажите, что $\Sigma_k \subset \Sigma_{k+1}\cap \Pi_{k+1}$.
	
	($iv$) Докажите, что $\mathcal{NP}\subset\mathcal{PSPACE}\subset\mathcal{EXPTIME}$.
\end{tasknum}

\begin{solution}
	
	\begin{enumerate}
		\item Пусть $L$~---~язык из условия задачи. Заметим, что принадлежность булевой формулы языку можно переписать в виде: $\varphi(\vec{x}, \vec{y}) \in L \LRa \exists \vec{x} \ \forall \vec{y} \vyp \varphi(\vec{x}, \vec{y}) = 1$. А это в свою очередь сильно напоминает условие принадлежности языка классу $\Sigma_2$: $\Sigma_2 \hm{=} \left\{ L | input \in L \LRa \exists x \ \forall y \vyp R(input, x, y) = 1\right\}$, где $R$ вычислима за $poly(|input|)\}$.
		
		В нашем случае $input = (\vec{x}, \vec{y})$, вычисление значения булевой функции $\varphi(\vec{x}, \vec{y})$ при заданном значении производится за $poly(|\vec{x}| + |\vec{y}|) = poly(|input|)$. Таким образом, наш язык $L$ лежит в $\Sigma_2$.
		
		\item Как мы доказываем в следующем пункте, имеют место следующие вложения: $\P = \Sigma_0 \subseteq \Sigma_1 \subseteq \Sigma_2 \subseteq \Sigma_3$, так что можем взять любую задачу из $\P$: мне очень нравится задача поиска $k$-й порядковой статистики в заданном массиве: $kth-order-statistics \hm{=} \{ (a, k, x) \ | \ a_k = x \}$, которая, как известно из курса основных алгоритмов, \href{https://people.csail.mit.edu/rivest/BlumFloydPrattRivestTarjan-TimeBoundsForSelection.pdf}{решается} за $O(|a|)$.
		
		\item
		\begin{itemize}
			\item Докажем сначала, что $\Sigma_k \subseteq \Sigma_{k+1}$: возьмём произвольный язык $L \in \Sigma_k$. $x\in L \LRa \exists y_1 \ \forall y_2 \ \ldots \exists (\forall) y_k \vyp R(x, y_1, y_2, \ldots , y_k) = 1$. Тогда мы можем добавить в $R$ фиктивную переменную $y_{k+1}$, которая, соответственно, никак не будет влиять на работу $R$, тогда мы можем записать аналогичную формулу, из которой уже будет следовать, что $L \in \Sigma_{k+1}$: $x\in L \LRa \exists y_1 \ \forall y_2 \ \ldots \exists (\forall) y_k \ \forall (\exists) y_{k+1} \vyp R(x, y_1, y_2, \ldots , y_k, y_{k+1}) = 1$.
		
			\item Теперь $\Sigma_k \subseteq \Pi_{k+1}$: возьмём произвольный язык $L \in \Sigma_k$. $x\in L \LRa \exists y_1 \ \forall y_2 \ \ldots \exists (\forall) y_k \vyp R(x, y_1, y_2, \ldots , y_k) = 1$. Если в прошлом пункте мы добавляли фиктивную переменную $y_{k+1}$ в <<конец>>, то теперь добавим фиктивную переменную $y_0$ в <<начало>>, она никак не будет влиять на работу $R$, таким образом мы можем записать аналогичную формулу, из которой уже будет следовать, что $L \in \Pi_{k+1}$: $x\in L \LRa \forall y_0 \ \exists y_1 \ \forall y_2 \ \ldots \exists (\forall) y_k \ \vyp R(x, y_0, y_1, y_2, \ldots , y_k) = 1$.
		\end{itemize}
	
	\item \begin{itemize}
		\item сначала докажем, что $\NP \subseteq \mathcal{PSPACE}$: возьмём произвольный язык $L \in \NP$, по определению $x\in L \LRa \exists y: R(x, y) = 1$, причём $R$ вычислима за $poly(|x|)$. Если мы можем за полиномиальную от длины входа память перебрать все сертификаты, то таким образом мы докажем доказать, что $L \in \mathcal{PSPACE}$.
		
		Так как длина всех сертификатов не превышает длину некоторого полинома от $x$, то существует такой полином $p(|x|)$, что $\forall y \vyp y \le p(|x|)$. Тогда определим работу МТ следующим образом: первые $p(|x|)$ ячеек выделяются под сертификат, а следующие $n$~---~под вычисление $R(x,y)$ (причём $R$ вычислима за полином, так что $n$ тоже не больше некоторого полинома от $|x|$ и таким образом мы выделяем $p(|x|) + n = poly(|x|)$ ячеек). Также заметим, что всего возможных сертификатов конечное количество: если $m$~---~количество символов в алфавите МТ, то всего сертификатов $1 + m + m^2 + \ldots + m^{p(|x|)}$. Таким образом наша МТ выдаёт 1, если сертификат подходит, то есть $R(x, y) = 1$. Иначе МТ стирает текущий сертификат, записывает в <<начало>> своей ленты следующий и продолжает свою работу по поиску подходящего сертификата. Если же МТ перебрала все сертификаты и не нашла подходящий, то она останавливается и выдаёт 0. Таким образом мы за полиномиальную память проверяем наличие сертификата, то есть принадлежность входа $x$ к языку $L$, причём в таком случае по построенной нами распознающей МТ также верно, что $L \in \mathcal{PSPACE} \Ra \NP \subseteq \mathcal{PSPACE}$.
		
		\item $\mathcal{PSPACE} \subseteq \mathcal{EXPTIME}$: рассмотрим количество возможных конфигураций МТ, распознающей некоторый язык $L \in \mathcal{PSPACE}$ за полиномиальную от длины входа память. Пусть $p(|x|)=poly(|x|)$~---~верхняя оценка используемой памяти. Тогда верхней оценкой количества всех конфигураций МТ будет $p(|x|) \cdot |x| \cdot q \cdot m^{p(|x|)}$, где $q$~---~количество различных состояний МТ, $m$~---~количество символов в алфавите МТ. МТ не может сделать больше шагов, потому что в этом случае она бы зациклилась и не остановилась. Таким образом, всего МТ делает $\le p(|x|) \cdot |x| \cdot q \cdot m^{p(|x|)} = poly(|x|) \cdot m^{p(|x|)} \le m^{poly(|x|)} \cdot m^{p(|x|)} = m^{poly(|x|)}$~---~экспоненциальное время, так как $m$~---~константа. Таким образом произвольный язык $L \in \mathcal{PSPACE}$ распознаётся за экспоненциальное время, а следовательно $\mathcal{PSPACE} \subseteq \mathcal{EXPTIME}$.
		
	\end{itemize}
	
	Таким образом получаем: $\mathcal{NP}\subseteq \mathcal{PSPACE}\subseteq \mathcal{EXPTIME}$.
	
	\end{enumerate}
	
\end{solution}

\begin{tasknum}{2}
	Покажите, как свести следующую задачу к вычислению некоторого перманента: найти количество перестановок $n$ элементов, в которых части элементов (с номерами $i_1, i_2,\dotsc i_k$) запрещено занимать позиции $j_1, \dotsc j_k$ соответственно.
\end{tasknum}

\begin{solution}
	
	Для того, чтобы свести задачу к вычислению некоторого перманента, нам для начала нужно определить матрицу, перманент которой мы собираемся считать: возьмём булеву матрицу $A = (a_{ij})$ размера $n \times n$, причём $a_{ij} = 1 \LRa$ $i$-му элементу разрешено стоять на $j$-й позиции. Тогда исходным числом перестановок и будет $perm$ $A = \displaystyle\sum_{\sigma \in S_n} a_{1\sigma(1)} a_{2\sigma(2)} \ldots a_{n\sigma(n)}$, так как если комбинация разрешена по условию, то все $a_{k\sigma(k)}$ обращаются в 1 и перманент учитывает эту перестановку в своей сумме, а если комбинация запрещена, то $\exists k: a_{k\sigma(k)} = 0$ и перманент данную перестановку не учитывает.
	
\end{solution}

\begin{tasknum}{3}
	Докажите, что если всякий $\mathcal{NP}$-трудный язык является $\mathcal{PSPACE}$-трудным, то $\mathcal{PSPACE} = \mathcal{NP}$.
\end{tasknum}

\begin{solution}
	
	То что $\NP \subseteq \mathcal{PSPACE}$ мы доказали в 4м пункте первой задачи. Докажем теперь, что если всякий $\mathcal{NP}$-трудный язык является $\mathcal{PSPACE}$-трудным, то выполняется и обратное включение.
	
	Рассмотрим произвольный язык $A \in \NP_c$. $A \in \NP \cap \NP_h \Ra A \in \mathcal{PSPACE} \cap \mathcal{PSPACE}_h \Ra A \in \mathcal{PSPACE}_c$. Тогда любую задачу из $\mathcal{PSPACE}$ можно за полиномиальное время свести к задаче из $\NP$ и, собственно, за полиномиальное время на недетерминированной МТ решить её. То есть таким образом любая задача из $\mathcal{PSPACE}$ решается за полиномиальное время на недетерминированной МТ, а следовательно, лежит в $\NP$. Поэтому $\mathcal{PSPACE} \subseteq \NP$ и обратное включение доказано.
	
\end{solution}

\begin{tasknum}{4}
	Докажите, что следующие языки лежат в $\mathcal{L}$: 
	
	($i$) $\{a\#b\#c|c=a+b\}$ ($a$, $b$, $c$~---~числа в двоичной записи).
	
	($ii$) $\{a\#b\#c|c=a\cdot b\}$ ($a$, $b$, $c$~---~числа в двоичной записи).
	
	($iii$) $UCYCLE = \{G  \ | \ \mbox{В неориентированном графе $G$ есть цикл}\}$
\end{tasknum}

\begin{solution}
	\begin{enumerate}
		\item Вроде как было на ТФС, причём на самом деле для этой задачи достаточно константной памяти. Можно хранить $i$~---~указатель разряда, который мы складываем в данный момент времени (идём при этом с конца), бит $shift$, отвечающий за перенос и, собственно, результат последнего сложения двух разрядов $res = a_i + b_i + shift$, который мы на каждом шаге будем сравнивать с $c_i$. Если существует разряд, для которого равенство не выполняется~---~выводим 0 и завершаем работу. Если были обработаны все разряды и МТ при этом не остановилась~---~выводим 1.
		
		\item Вроде бы тоже было, но тут я уже не уверен. Однако алгоритм примерно тот же. Будем хранить $i$~---~указатель разряда, который мы складываем в данный момент времени (идём при этом с конца), число (а не бит, как в предыдущем пункте) $shift$, отвечающий за перенос и, собственно, результат последнего сложения двух разрядов $res = a_0 b_i + a_1 b_{i-1} + \ldots + a_i b_0 + shift$, последний разряд которого мы на каждом шаге будем сравнивать с $c_i$, а всё, кроме последнего разряда~---~отправлять в перенос. Если существует разряд, для которого равенство не выполняется~---~выводим 0 и завершаем работу. Если были обработаны все разряды и МТ при этом не остановилась~---~выводим 1.
		
		\item По теореме Рейнгольда $UPATH \in L$. Таким образом мы можем перебрать всевозможные упорядоченные пары вершин $(u, v)$ такие, что $(u, v) \in E$. Делать это будем следующим образом: пусть $G = (V, E)$~---~наш исходный граф. Возьмём граф $\tilde{G} = (V, \tilde{E})$, где $\tilde{E} = E \setminus (u, v)$. Проверим наличие пути между из вершины $v$ в вершину $u$ (можем это сделать за логарифмическую память по упомянутой выше теореме Рейнгольда). И заметим, что если в $\tilde{G}$ найден путь из $v$ в $u$, то в исходном графе $G$ есть цикл, так как добавляется ребро $(u, v)$, дополняющее найденный путь до цикла.
		
	\end{enumerate}
\end{solution}

\begin{tasknum}{5}
	Сертификатное определение $\mathcal{NL}$: $A \in \mathcal{NL}$ тогда и только тогда, когда для некоторой детерминированной машины $M$ выполнена эквивалентность: $x \in A \Leftrightarrow \exists s : \ M(x, s) = 1$. При этом длина $s$ должна быть полиномиальна от длины $x$, машина получает $s$ на отдельной ленте, по которой может двигаться только слева направо, а количество ячеек, занятых на рабочей ленте, должно быть логарифмическим.
	
	Вопрос задачи: а какой класс получится, если в предыдущем определении разрешить машине двигаться по сертификатной ленте в обе стороны?
\end{tasknum}

\begin{solution}
	Пусть получается некоторый класс $\mathcal{A}$. Из определения очевидно, что $\mathcal{A} \subseteq \NP$, так как мы по сути берём определение класса $\NP$, но накладываем дополнительное ограничение в виде того, что \textit{количество ячеек, занятых на рабочей ленте, должно быть логарифмическим}. 
	
	Докажем, что на самом деле верно и обратное включение. Для этого построим на нашей МТ верификатор для $\NP_c$ задачи 3КНФ, который будет при этом работать за логарифмическую память. Будем хранить всего 2 значения: значение текущего дизъюнкта, значение рассматриваемого в данный момент литерала. Таким образом, взяв сертификатом выполняющий набор, мы без проблем проверяем истинность формулы с помощью того, что дойдя до некоторого литерала, можем без проблем <<пробежаться по ленте назад>> и посмотреть его значение. Таким образом $\NP \subseteq \mathcal{A}$.
	
	Итого имеем: $A = \NP$.
\end{solution}

\begin{tasknum}{6}
	Докажите, что логарифмическая сводимость транзитивна, причем если $B \in \mathcal{L}$ и $A \leq_L B$, то $A \in \mathcal{L}$, а если $B \in \mathcal{NL}$ и $A \leq_L B$, то $A \in \mathcal{NL}$. 
\end{tasknum}

\begin{solution}
	\begin{enumerate}
		\item Транзитивность: если $A\leq_L B$ и $B\leq_L C$, то $\exists f, g$, вычислимые за логарифмическую память, такие что $x \in A \LRa f(x) \in B \LRa g(f(x)) \in C$. Заметим, что функция $g(f(x))$ вычисляется за $\log(|f(x)|) = \log(\log(|x|)) = \log(|x|)$, так как логарифм от логарифма также является логарифмом. Таким образом $x \in A \LRa g(f(x)) \in C \Ra A \leq_L C$

		\item \begin{itemize}
			\item $A \leq_L B \Ra \exists f$ --- функция, вычислимая за логарифмическую память, такая что $x \in A \LRa f(x) \in B$; таким образом существует детерминированная МТ $M_1$, вычисляющая функцию $f(x)$ за $\log(|x|)$
			\item $B \in \mathcal{L} \Ra \exists M_2$ --- детерминированная МТ, распознающая язык $B$ за логарифмическую память.
		\end{itemize}
		
		Построим теперь детерминированную МТ $M$, распознающую язык $A$ за полином: на входе $x$ сначала моделируется работа $M_1$, то есть за логарифмическую память от длины входа вычисляется функция $f(x)$, а затем моделируется работа $M_2$ на входе $f(x)$
		
		Таким образом детерминированная МТ $M$ распознаёт язык $A$ за $\log(|x|) + \log(|f(x)|) = \log(|x|) + \log(\log(|x|)) = \log(|x|)$, так как логарифм от логарифма также является логарифмом и сумма логарифмов есть логарифм. Таким образом язык $A$ распознаётся за логарифмическую память от длины входа на детерминированной МТ $M$, следовательно $A \in \mathcal{L}$
		
		\item Доказательство аналогично предыдущему пункту с точностью до замены детерминированной МТ на недетерминированную МТ, приведём его для полноты:
		
		 \begin{itemize}
			\item $A \leq_L B \Ra \exists f$ --- функция, вычислимая за логарифмическую память, такая что $x \in A \LRa f(x) \in B$; таким образом существует детерминированная МТ $M_1$, вычисляющая функцию $f(x)$ за $\log(|x|)$
			\item $B \in \mathcal{NL} \Ra \exists M_2$ --- недетерминированная МТ, распознающая язык $B$ за логарифмическую память.
		\end{itemize}
		
		Построим теперь недетерминированную МТ $M$, распознающую язык $A$ за полином: на входе $x$ сначала моделируется работа $M_1$, то есть за логарифмическую память от длины входа вычисляется функция $f(x)$, а затем моделируется работа $M_2$ на входе $f(x)$
		
		Таким образом недетерминированная МТ $M$ распознаёт язык $A$ за $\log(|x|) + \log(|f(x)|) = \log(|x|) + \log(\log(|x|)) = \log(|x|)$, так как логарифм от логарифма также является логарифмом и сумма логарифмов есть логарифм. Таким образом язык $A$ распознаётся за логарифмическую память от длины входа на недетерминированной МТ $M$, следовательно $A \in \mathcal{NL}$
	\end{enumerate}
\end{solution}

\end{document}