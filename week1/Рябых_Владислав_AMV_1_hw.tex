\documentclass[a4paper,12pt]{article}

%%% Работа с русским языком
\usepackage[cm]{fullpage}
\usepackage[T2A]{fontenc}
\usepackage[utf8]{inputenc}
\usepackage[english,russian]{babel}
\usepackage{footmisc}
\usepackage[document]{ragged2e}
\usepackage{amsmath,amsfonts,amssymb,mathtools}
\usepackage{framed}
\usepackage{pstricks}
%\usepackage[framed]{ntheorem}
\usepackage{tikz}
\usetikzlibrary{arrows,automata}
\usepackage{cmap}					% поиск в PDF
\usepackage{mathtext} 				% русские буквы в формулах
\usepackage{indentfirst}
\frenchspacing

\newcommand{\vyp}{\hookrightarrow}
\renewcommand{\epsilon}{\varepsilon}
\renewcommand{\phi}{\varphi}
\renewcommand{\kappa}{\varkappa}
\renewcommand{\le}{\leqslant}
\renewcommand{\leq}{\leqslant}
\renewcommand{\ge}{\geqslant}
\renewcommand{\geq}{\geqslant}
\renewcommand{\emptyset}{\varnothing}

%%% Дополнительная работа с математикой
\usepackage{amsmath,amsfonts,amssymb,amsthm,mathtools} % AMS
\usepackage{icomma} % "Умная" запятая: $0,2$ --- число, $0, 2$ --- перечисление

%% Номера формул
%\mathtoolsset{showonlyrefs=true} % Показывать номера только у тех формул, на которые есть \eqref{} в тексте.
%\usepackage{leqno} % Нумереация формул слева

%% Свои команды
\DeclareMathOperator{\sgn}{\mathop{sgn}}

%% Перенос знаков в формулах (по Львовскому)
\newcommand*{\hm}[1]{#1\nobreak\discretionary{}
{\hbox{$\mathsurround=0pt #1$}}{}}



%%% Работа с картинками
\usepackage{graphicx}  % Для вставки рисунков
\graphicspath{{images/}{images2/}}  % папки с картинками
\setlength\fboxsep{3pt} % Отступ рамки \fbox{} от рисунка
\setlength\fboxrule{1pt} % Толщина линий рамки \fbox{}
\usepackage{wrapfig} % Обтекание рисунков текстом

%%% Работа с таблицами
\usepackage{array,tabularx,tabulary,booktabs} % Дополнительная работа с таблицами
\usepackage{longtable}  % Длинные таблицы
\usepackage{multirow} % Слияние строк в таблице

%%% Теоремы
\theoremstyle{plain} % Это стиль по умолчанию, его можно не переопределять.
\newtheorem{theorem}{Теорема}[section]
\newtheorem{proposition}[theorem]{Утверждение}
 
\theoremstyle{definition} % "Определение"
\newtheorem{corollary}{Следствие}[theorem]
\newtheorem{problem}{Задача}[section]
 
\theoremstyle{remark} % "Примечание"
\newtheorem*{nonum}{Решение}

%%% Программирование
\usepackage{etoolbox} % логические операторы

%%% Страница
\usepackage{extsizes} % Возможность сделать 14-й шрифт
\usepackage{geometry} % Простой способ задавать поля
\geometry{top=20mm}
\geometry{bottom=20mm}
\geometry{left=10mm}
\geometry{right=20mm}
 %
\usepackage{fancyhdr} % Колонтитулы
 	\pagestyle{fancy}
 	\renewcommand{\headrulewidth}{0pt}  % Толщина линейки, отчеркивающей верхний колонтитул
\fancypagestyle{firstpage}{
	\rhead{\large{Рябых Владислав, Б05-905}}
}
% 	\lfoot{Нижний левый}
% 	\rfoot{Нижний правый}
% 	\rhead{Верхний правый]}
% 	\chead{Верхний в центре}
% 	\lhead{Верхний левый}
%	\cfoot{Нижний в центре} % По умолчанию здесь номер страницы

\usepackage{setspace} % Интерлиньяж
\onehalfspacing % Интерлиньяж 1.5
%\doublespacing % Интерлиньяж 2
%\singlespacing % Интерлиньяж 1

\usepackage{lastpage} % Узнать, сколько всего страниц в документе.

\usepackage{soul} % Модификаторы начертания

%\usepackage{hyperref}
%\usepackage[usenames,dvipsnames,svgnames,table,rgb]{xcolor}
%\hypersetup{				% Гиперссылки
%    unicode=true,           % русские буквы в раздела PDF
%    pdftitle={Заголовок},   % Заголовок
%    pdfauthor={Автор},      % Автор
%    pdfsubject={Тема},      % Тема
%    pdfcreator={Создатель}, % Создатель
%    pdfproducer={Производитель}, % Производитель
%    pdfkeywords={keyword1} {key2} {key3}, % Ключевые слова
%    colorlinks=true,       	% false: ссылки в рамках; true: цветные ссылки
%    linkcolor=red,          % внутренние ссылки
%    citecolor=black,        % на библиографию
%    filecolor=magenta,      % на файлы
%    urlcolor=cyan           % на URL
%}

\usepackage{csquotes} % Еще инструменты для ссылок

%\usepackage[style=authoryear,maxcitenames=2,backend=biber,sorting=nty]{biblatex}

\usepackage{multicol} % Несколько колонок

\usepackage{pgfplots}
\usepackage{pgfplotstable}
\newcommand{\tbf}{\textbf}


\usepackage[shortlabels]{enumitem}

\newtheorem{task}{\textbf{Задача}}

\newtheorem{innercustomthm}{\textbf{Задача}}
\newenvironment{tasknum}[1]
{\renewcommand\theinnercustomthm{#1}\innercustomthm}
{\endinnercustomthm}

\newtheorem*{solution}{\textbf{Решение}}
\newcommand{\Ra}{\Rightarrow}
\newcommand{\La}{\Leftarrow}
\newcommand{\ra}{\rightarrow}
\newcommand{\LRa}{\Leftrightarrow}
\newcommand{\n}{\mathbb}
\newcommand{\Le}{\leqslant}
\newcommand{\Ge}{\geqslant}


\renewcommand{\inf}{\infty}
\newcommand{\ol}{\overline}

\newcommand{\bigline}{\noindent\makebox[\linewidth]{\rule{\paperwidth}{0.4pt}}}

\usetikzlibrary{fit}
\newcommand{\dost}{\overset{*}{\vdash}}
\newcommand{\vyv}{\overset{*}{\Rightarrow}}


\usepackage{capt-of}
\usepackage{tikz-qtree}
\usepackage{systeme}

\newcommand{\polysv}{\leq_p}
\def\coNP{{\mathbf{\text{\textbf{co--}}\mathcal{NP}}}}
\newcommand{\NP}{\mathcal{NP}}
\renewcommand{\P}{\mathcal{P}}

\usepackage{algorithm}
%\usepackage{algpseudocode}
\usepackage[noend]{algpseudocode}

\begin{document}
	
	\thispagestyle{firstpage}
	
	\begin{center}
		\textbf{\Large{Алгоритмы и модели вычислений. \\ Домашнее задание № 1}}
	\end{center}


\begin{tasknum}{2}
	Имеются различные клетчатые таблички~---~нужно подсчитать число способов замостить ими большое поле из клеток без пробелов и наложений.
	
	Разрешено использовать таблички: чёрный квадрат $2\times 2$, белый квадрат $2\times 2$, серый прямоугольник $2\times 1$ с возможностью поворота. Поле представляет собой полосу $2\times n$. найдите асимптотику числа замощений и явную формулу для него.
\end{tasknum}

\begin{solution}
	
	Пусть $T(n)$ ~---~ количество способов замостить полосу $2\times n$. Для нахождения рекурсивной формулы воспользуемся следующим наблюдением: так как размеры наших табличек не превышают $2\times 2$, то принципиально у нас есть $2$ способа замостить край полосы: 
	
	\begin{enumerate}
		\item замостить один крайний столбец: на это у нас есть всего один способ ~---~ поставить серый прямоугольник $2\times 1$
		\item замостить сразу два крайних столбца: на это у нас есть $3$ способа ~---~ чёрный квадрат $2\times 2$, белый квадрат $2\times 2$, два серых прямоугольника $1\times 2$ каждый (случай двух серых прямоугольников $2\times 1$ покрывается предыдущим пунктом, поэтому тут не рассматривается)
	\end{enumerate}
	
	Таким образом, получаем $T(n) \hm{=} T(n-1) + 3T(n-2)$
	
	Найдём асимптотику $T(n)$ исходя из того, что $T(n-1) \hm{=} T(n-2) + 3T(n-3) \ge T(n-2)$:
	
	\begin{itemize}
		\item $T(n) \hm{=} T(n-1) + 3T(n-2) \le 4T(n-1) \hm{=} 4 \cdot 4T(n-2) \hm{=} \ldots \hm{=} 4^n \Ra T(n) \hm{=} O(4^n)$
		\item $T(n) \hm{=} T(n-1) + 3T(n-2) \ge 4T(n-2) \hm{=} 4 \cdot 4T(n-2 \cdot 2) \hm{=} \ldots \hm{=} 4^{\frac{n}{2}} \hm{=} 2^n \Ra T(n) \hm{=} \Omega(2^n)$
	\end{itemize}

	Решим линейное рекуррентное соотношение: для начала запишем характеристическое уравнение
	$$\lambda^2 \hm{=} \lambda + 3 \ \ \Ra \ \ \lambda_{1, 2} \hm{=} \dfrac{1 \pm \sqrt{13} }{2}$$
	Таким образом $T(n) \hm{=} C_1 \left( \dfrac{1 + \sqrt{13} }{2}\right) ^n + C_2 \left( \dfrac{1 - \sqrt{13} }{2}\right) ^n$, подберём константы из начальных условий: при $n \hm{=} 0$ у нас есть 1 способ замостить ленту, при $n\hm{=}1$ ~---~ также 1 способ, то есть $T(0) \hm{=} T(1) \hm{=} 1$, тогда
	$$C_1 + C_2 \hm{=} 1 \ \ \ \ \ \ C_1 \cdot \dfrac{1 + \sqrt{13} }{2} + C_2 \cdot \dfrac{1 - \sqrt{13} }{2} \hm{=} 1$$
	Решая, получаем $C_1 \hm{=} \dfrac{1}{2} + \dfrac{1}{2\sqrt{13}} \hm{=} \dfrac{1}{\sqrt{13}} \lambda_{1}$ и $C_2 \hm{=} \dfrac{1}{2} - \dfrac{1}{2\sqrt{13}} \hm{=} - \dfrac{1}{\sqrt{13}} \lambda_{2}$
	
	И, окончательно: $$T(n) \hm{=} \dfrac{1}{\sqrt{13}} \cdot \left( \dfrac{1 + \sqrt{13} }{2}\right) ^{n+1} - \dfrac{1}{\sqrt{13}} \cdot \left( \dfrac{1 - \sqrt{13} }{2}\right) ^{n+1}$$
	
\end{solution}

\begin{tasknum}{3}
	Найдите $\Theta$ асимптотику рекуррентности, которая определяется в следующем тексте.
	
	Colour the edges of a
	complete graph of n vertices by three colours so that the number of triangles
	all whose edges get a different colour is maximal. Denote this maximum
	by $G_3(k)$. They conjectured that $G_3(k)$ is obtained as follows: clearly
	$G_3(1) \hm{=} G_3(2) \hm{=} 0, G_3(3) \hm{=} 1, G_3(4) \hm{=} 4$. Suppose $G_3(k_1)$ has already been
	determined for every $k_1 < k$. Then
	$$G_3(k) \hm{=} G_3(u_1) + G_3(u_2) + G_3(u_3) + G_3(u_4)
	+ u_1 u_2 u_3 + u_1 u_2 u_4 + u_1 u_3 u_4 + u_2 u_3 u_4$$
	where $u_1 + u_2 + u_3 + u_4 \hm{=} k$ and the $u$’s are as nearly equal as possible.
\end{tasknum}

\begin{solution}
	<<$u$’s are as nearly equal as possible>>, так что $u_{min} \ge \dfrac{k}{5}, u_{max} \le \dfrac{k}{3}$, поэтому 
	\begin{itemize}
		\item $G_3(k) \ge G_{min}(k) \hm{=} 4 G_3\left(\dfrac{k}{5} \right) + \dfrac{4}{5^3} k^3$ 
		\item $G_3(k) \le G_{max}(k) \hm{=} 4 G_3\left(\dfrac{k}{3} \right) + \dfrac{4}{3^3} k^3$ 
	\end{itemize}

	Решим:
	
	\begin{itemize}
		\item $G_{min}(k) \hm{=} 4 G_3\left(\dfrac{k}{5} \right) + \dfrac{4}{5^3} k^3 \hm{=} 4 \cdot \left(4 G_3\left(\dfrac{k}{25} \right) + \dfrac{4}{5^3} \left(\dfrac{k}{5} \right) ^3 \right) + \dfrac{4}{5^3} k^3 \hm{=} \ldots \hm{=} \left(1 + \dfrac{4}{5^3} + \ldots + \left( \dfrac{4}{5^3}\right) ^{\log k}\right) \cdot \dfrac{4}{5^3} k \hm{=} \Omega(k^3)$
	
	Таким образом, $G_3(k) \hm{=} \Omega(k^3)$
	
		\item Аналогично $G_{max}(k) \hm{=} 4 G_3\left(\dfrac{k}{3} \right) + \dfrac{4}{3^3} k^3 \hm{=} 4 \cdot \left(4 G_3\left(\dfrac{k}{9} \right) + \dfrac{4}{3^3} \left(\dfrac{k}{3} \right) ^3 \right) + \dfrac{4}{3^3} k^3 \hm{=} \ldots \hm{=} \left(1 + \dfrac{4}{3^3} + \ldots + \left( \dfrac{4}{3^3}\right) ^{\log k}\right) \cdot \dfrac{4}{3^3} k \hm{=} O(k^3)$
	
	Таким образом, $G_3(k) \hm{=} O(k^3)$
	
\end{itemize}
	
	И, окончательно, $G_3(k) \hm{=} \Theta(k^3)$
	
\end{solution}

\begin{tasknum}{4}
	($i$) Вычислите число правильно составленных скобочных выражений, содержащих $n$ скобок, в которых в любом непустом \textit{собственном} префиксе число открывающих скобок больше числа закрывающих.
	
	($ii$) Найдите явное аналитическое выражение для производящей функции чисел $BR_{4n+2}$ правильных скобочных последовательностей длины $4n+2$ (ответ в виде суммы ряда не принимается). 
\end{tasknum}

\begin{solution}
	
	Пусть $CBS(n)$ ~---~ язык всех правильных скобочных последовательностей (Correct bracket sequences) из $n$ пар скобок, $L(n) \subseteq CBS(n)$ ~---~ язык из $CBS$, в которых в любом непустом префиксе число открывающих скобок больше числа закрывающих, $w \in \Sigma^*, \Sigma \hm{=} \{ \ ), ( \ \}$ ~---~ произвольное слово
		
		Докажем, что $w \in L(n) \LRa w \hm{=} (u), u \in CBS(n-1), n \ge 2$:
		
		\begin{itemize}
			\item $\La$: так как $u \in CBS(n-1)$, то $\forall x \in pref(u) \vyp |x|_( \ge |x|_)$, а так как каждый префикс $w$ представляется в виде $y \hm{=} (x$, то $\forall y \in pref(w) \vyp |y|_( > |y|_) \Ra w \in L(n)$
			\item $\Ra$: очевидно, $w[1] \hm{=} (, w[2n] \hm{=} )$ в силу того, что $w \in CBS(n)$.		
			%, помимо этого заметим, что $w[2] \hm{=} (, w[2n-1] \hm{=} )$. Действительно, иначе бы $w[1]w[2] \hm{=} () \in pref(w)$, но такого быть не может, так как $\forall y \in pref(w) \vyp |y|_( > |y|_)$ по условию; аналогично, если бы $w[2n-1] \hm{=} ($, то выполнялось бы $y \hm{=} w[1]\ldots w[2n-2] \in pref(w)$, причём $|w|_( - 1 \hm{=} |y|_( \hm{=} |y|_) \hm{=} |w|_) - 1$ ~---~ противоречие. 
			Таким образом, $w \hm{=} (u)$. Осталось показать, что $u$ ~---~ правильная скобочная последовательность. Если бы это на самом деле было не так, то существовал бы $x \in pref(u) : |x|_( < |x|_)$, но тогда существовал бы $y \hm{=} (x \in pref(w) : |y|_( \hm{=} |x|_( + 1 \le |x|_) \hm{=} |y|_)$ ~---~ противоречие и таким образом $u \in CBS(n-1)$, что и требовалось
			
		\end{itemize}
		
		Таким образом мы доказали, что $|L(n)| \hm{=} |CBS(n-1)| \hm{=} C_{(n-1}$, где $C_n$ ~---~ $n$-й элемент последовательности чисел Каталана
	
\end{solution}

\begin{tasknum}{5}
		Оцените трудоемкость рекурсивного алгоритма, разбивающего исходную задачу размера $n$ на три задачи размером $\lceil\frac{n}{\sqrt{3}}\rceil-5$, используя для этого $10\frac{n^3}{\log n}$ операций.
\end{tasknum}
	
	\begin{solution}
		
		$T(n) \hm{=} 3T\left(\left \lceil\dfrac{n}{\sqrt{3}}\right \rceil-5 \right) + 10\dfrac{n^3}{\log n}$
		\begin{itemize}
		\item $T(n) \ge 10\dfrac{n^3}{\log n} \Ra T(n) \hm{=} \Omega\left(\dfrac{n^3}{\log n}\right)$
		
		\item $T(n) \le 3T\left(\left \lceil\dfrac{n}{\sqrt{3}}\right \rceil\right) + 10\dfrac{n^3}{\log n}$; применим мастер теорему: $d \hm{=} \log_b a \hm{=} \log_{\sqrt{3}}3 \hm{=} 2; \ \ 10\dfrac{n^3}{\log n} \hm{=} \Omega(n^{d+\epsilon})$ при $\epsilon$, равном, например, $0.5 \Ra$ по мастер-теореме $3T\left(\left \lceil\dfrac{n}{\sqrt{3}}\right \rceil\right) + 10\dfrac{n^3}{\log n} \hm{=} \Theta\left(\dfrac{n^3}{\log n}\right) \Ra T(n) \hm{=} O\left(\dfrac{n^3}{\log n}\right)$
		
	\end{itemize}

	Таким образом, получаем, $T(n) \hm{=} \Theta\left(\dfrac{n^3}{\log n}\right)$
	
	\end{solution}

\begin{tasknum}{6}
	
	Рассмотрим детерминированный алгоритм поиска медианы по кальке известного линейного алгоритма, где используется разбиение массива на четвёрки элементов, в каждой из которых определяется \emph{нижняя} медиана, т.~е. из в каждой четверки выбирается второй по порядку элемент (элементы можно считать различными). Приведите рекуррентную оценку числа сравнений в этой процедуре и оцените сложность такой модификации.     
	
\end{tasknum}

\begin{solution}
	
	Пусть $x$ ~---~ опорный элемент, тогда в половине четвёрок элементов есть по $2$ элемента, больших $x$, за исключением, быть может, последней группы, в которой менее $4$-х элементов. Таким образом, элементов, больших $x$, по крайней мере $3 \cdot \left( \dfrac{1}{2} \cdot \dfrac{n}{4} - 2 \right) \hm{=} \dfrac{3n}{8} - 6$; а элементов, меньших $x$, соответственно $2 \cdot \left( \dfrac{1}{2} \cdot \dfrac{n}{4} - 2 \right) \hm{=} \dfrac{n}{4} - 4$
	
	Таким образом, $T(n) \hm{=} T\left(\left\lceil \dfrac{n}{4}\right\rceil \right) + T\left(\dfrac{5n}{8} + 6 \right) + O(n)$ 
	
\end{solution}

\begin{tasknum}{7}
	
	 Функция натурального аргумента $S(n)$ задана рекурсией:
	$$
	S(n)=\left\{\begin{array}{cc}
	100&, n\leq 100\\
	S(n-1)+S(n-3)&, n>100
	\end{array}
	\right.
	$$      
	Оцените число рекурсивных вызовов процедуры $S(\cdot)$ при вычислении $S(10^{12})$
	
\end{tasknum}

\begin{solution}
	
	Рассматриваем функцию $S(n)$ при больших $n$: $S(n) \hm{=} S(n-1)+S(n-3) \Ra S(n-1) \hm{=} S(n-3) + S(n-5) \ge S(n-3)$. Таким образом при вычислении $S(n-1)$ происходит больше рекурсивных вызовов, чем при вычислении $S(n-3)$. Пусть $K(m)$ ~---~ количество рекурсивных вызовов процедуры $S(\cdot)$ при вычислении $S(m)$. Тогда по условию нам надо оценить $K(10^{12})$, а только что мы показали, что $K(n-1) \ge K(n-3)$, следовательно $2K(n-3) \le K(n) \le 2K(n-1)$
	
	Таким образом получаем:
	$$2^{\frac{10^{12} - 100}{3}} \le \ldots \le 2K(10^{12}-3) \le K(10^{12}) \le 2K(10^{12} - 1) \le \ldots \le 2^{10^{12} - 100}$$
	
\end{solution}

\begin{tasknum}{9}
	Оцените трудоемкость рекурсивного алгоритма, разбивающего исходную задачу размера $n$ на $n$ задач размеров $\lceil \frac n 2 \rceil$ каждая, используя для этого $O(n)$ операций. 
\end{tasknum}

\begin{solution}
	
	$T(n) \hm{=} nT\left(\left\lceil \dfrac{n}{2} \right \rceil \right) + O(n)$. $d \hm{=} \log_b a \hm{=} \log_2 n$; $O(n) \hm{=} O(n^{d-\epsilon})$ при $\epsilon$, равном, например, $0.5 \Ra$ по мастер-теореме $T(n) \hm{=} \Theta(n^d) \hm{=} \Theta(n^{\log n})$
	
\end{solution}

\end{document}