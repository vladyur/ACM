\documentclass[12pt,a5paper,fleqn]{article}
\usepackage[utf8]{inputenc}
\usepackage{amssymb, amsmath, multicol}
\usepackage[russian]{babel}
\usepackage{graphicx}
\usepackage[shortcuts,cyremdash]{extdash}
\usepackage{wrapfig}
\usepackage{floatflt}
\usepackage{lipsum}
\usepackage{concmath}
\usepackage{euler}

\graphicspath{ {images/} }

\oddsidemargin=-17.9mm
\textwidth=133mm
\headheight=-35.4mm
\textheight=200mm
\parindent=0pt
\tolerance=100
\parskip=6pt
\pagestyle{empty}
\renewcommand{\tg}{\mathop{\mathrm{tg}}\nolimits}
\renewcommand{\ctg}{\mathop{\mathrm{ctg}}\nolimits}
\renewcommand{\arctan}{\mathop{\mathrm{arctg}}\nolimits}
\newcommand{\divisible}{\mathop{\raisebox{-2pt}{\vdots}}}
\renewcommand{\not}{\overline}

\RequirePackage{caption2}
\renewcommand\captionlabeldelim{}
\newcommand*{\hm}[1]{#1\nobreak\discretionary{}%
\newtheorem{Theorem}{Теорема}
{\hbox{$\mathsurround=0pt #1$}}{}}

\begin{document}


\begin{center}
{ \Large Задание на первую неделю.}

\end{center}

{\bf 1.} Пусть $A_n$~--- число натуральных решений уравнения $2x+3y=n$, т. е. $A_1=A_2\hm{=}A_3=A_4=0,\, A_5=1\, (x=1,y=1),\dotsc$.

\noindent ($i$) Найдите производящую функцию последовательности $A_n$, $n=1,2,\dotsc$.

\noindent ($ii$) Найдите $\theta$-асимптотику $A_n$.

\noindent ($iii$) Найдите явное аналитическое выражение для $A_n$.


\smallskip

{\bf 2.} Имеются различные клетчатые таблички~---~нужно подсчитать число способов замостить ими большое поле из клеток без пробелов и наложений.

Разрешено использовать таблички: чёрный квадрат $2\times 2$, белый квадрат $2\times 2$, серый прямоугольник $2\times 1$ с возможностью поворота. Поле представляет собой полосу $2\times n$. найдите асимптотику числа замощений и явную формулу для него.

\smallskip

{\bf 3.}   Найдите $\Theta$-асимптотику рекуррентности, которая определяется в следующем тексте.

Colour the edges of a
complete graph of n vertices by three colours so that the number of triangles
all whose edges get a different colour is maximal. Denote this maximum
by $G_3(k)$. They conjectured that $G_3(k)$ is obtained as follows: clearly
$G_3(1) = G_3(2) = 0, G_3(3) = 1, G_3(4) = 4$. Suppose $G_3(k_1)$ has already been
determined for every $k_1 < k$. Then
\begin{multline*}
$$G_3(k) = G_3(u_1) + G_3(u_2) + G_3(u_3) + G_3(u_4) +\\
+ u_1 u_2 u_3 + u_1 u_2 u_4 + u_1 u_3 u_4 + u_2 u_3 u_4,
\end{multline*}
where $u_1 + u_2 + u_3 + u_4 = k$ and the $u$’s are as nearly equal as possible.

\smallskip

{\bf 4.} ($i$) Вычислите число правильно составленных скобочных выражений, содержащих $n$ скобок, в которых в любом непустом префиксе число открывающих скобок больше числа закрывающих.

($ii$) Найдите явное аналитическое выражение для производящей функции чисел $BR_{4n+2}$ правильных скобочных последовательностей длины $4n+2$ (ответ в виде суммы ряда не принимается). 

\smallskip

{\bf 5.} Оцените трудоемкость рекурсивного алгоритма, разбивающего исходную задачу размера $n$ на три задачи размером $\lceil\frac{n}{\sqrt{3}}\rceil-5$, используя для этого $10\frac{n^3}{\log n}$ операций.

\smallskip

{\bf 6.} Рассмотрим детерминированный алгоритм поиска медианы по кальке известного линейного алгоритма, где используется разбиение массива на четвёрки элементов, в каждой из которых определяется \emph{нижняя} медиана, т.~е. из в каждой четверки выбирается второй по порядку элемент (элементы можно считать различными). Приведите рекуррентную оценку числа сравнений в этой процедуре и оцените сложность такой модификации.      

\smallskip

{\bf 7.} Функция натурального аргумента $S(n)$ задана рекурсией:
$$
S(n)=\left\{\begin{array}{cc}
100&, n\leq 100\\
S(n-1)+S(n-3)&, n>100
\end{array}
\right.
$$      
Оцените число рекурсивных вызовов процедуры $S(\cdot)$ при вычислении $S(10^{12})$.      

\smallskip

{\bf 8 (Доп).}  Оцените как можно точнее высоту дерева рекурсии для рекуррентности
$T(n) = T(n-\lfloor \sqrt{n} \rfloor) + T(\lfloor \sqrt{n} \rfloor) + \Theta(n)$.

\smallskip

{\bf 9 (Доп).} Оцените трудоемкость рекурсивного алгоритма, разбивающего исходную задачу размера $n$ на $n$ задач размеров $\lceil \frac n 2 \rceil$ каждая, используя для этого $O(n)$ операций. 

\smallskip


\end{document}





