\documentclass[12pt,a5paper,fleqn]{article}
\usepackage[utf8]{inputenc}
\usepackage{amssymb, amsmath, multicol}
\usepackage[russian]{babel}
\usepackage{graphicx}
\usepackage[shortcuts,cyremdash]{extdash}
\usepackage{wrapfig}
\usepackage{floatflt}
\usepackage{lipsum}
\usepackage{concmath}
\usepackage{euler}
\usepackage{algorithm}
\usepackage{algpseudocode} 

\graphicspath{ {images/} }

\oddsidemargin=-17.9mm
\textwidth=133mm
\headheight=-35.4mm
\textheight=200mm
\parindent=0pt
\tolerance=100
\parskip=6pt
\pagestyle{empty}
\renewcommand{\tg}{\mathop{\mathrm{tg}}\nolimits}
\renewcommand{\ctg}{\mathop{\mathrm{ctg}}\nolimits}
\renewcommand{\arctan}{\mathop{\mathrm{arctg}}\nolimits}
\newcommand{\divisible}{\mathop{\raisebox{-2pt}{\vdots}}}


\RequirePackage{caption2}
\renewcommand\captionlabeldelim{}
\newcommand*{\hm}[1]{#1\nobreak\discretionary{}%
\newtheorem{Theorem}{Теорема}
{\hbox{$\mathsurround=0pt #1$}}{}}

\begin{document}


\begin{center}
{ \Large Домашнее задание 11.}

\end{center}

{\bf 1.} Имеются окрашенные прямоугольные таблички трёх типов: черный квадрат размера $2\times 2$, белый квадрат того же размера и серый прямоугольник $2\times 1$ (последний можно поворачивать на $90^\circ$). Нужно подсчитать число способов $F_n$ замостить полосу размера $2\times n$. Найдите явную аналитическую формулу для $F_n$ и вычислите $F_{30000}$ по модулю $31$. 

\smallskip

{\bf 2.} Выполните задачи 1, Д-1 из приложенного файла (все по 1 баллу).

\smallskip

{\bf 3.} а) Верно ли, что существует такая функция $f: \mathbb{N} \rightarrow  \mathbb{N}$, для любых констант $\forall\, c,d> 0$ выполнено $$f(n) = \omega(n^c),\ f(n) = o(2^{nd}),$$ т.~е. функция $f(n)$ растет быстрее любого заданного полинома, но медленнее любой заданной экспоненты?

б) Некто анонсировал теорему (т.~е. утверждение может быть и неверно), что любой МТ требуется $\Omega(n \log_2^{\log_2 n} n)$ тактов для того, чтобы проверять тавтологичность формул, заданных в формате  {\sc 4-ДНФ}, т.~е. дизъюнктивных нормальных форм, в каждый конъюнкт которых входит не более четырех переменных (здесь $n$~--- длина входа).  Считаем, что теорема верна. Верно ли, что из этого вытекает, что $\mathcal{P}$ не совпадает с $co-\mathcal{NP}$?

\smallskip

{\bf 4 (по $0{,}5$ балла).} а) Делится ли $4^{1356}-9^{4824}$ на $35$? Делится ли $5^{30000} - 6^{123456}$ на $31$?

б) Найдите обратные $20 \ (\mbox{mod } 79)$, $3 \ (\mbox{mod } 62)$.

в) Найдите все решения уравнения $35x = 10 \ (\mbox{mod } 50)$.

г) Имеет ли решение сравнение $x^2 = 1597 $

д) Найдите наименьшее натуральное число, имеющее остатки $2$, $3$, $1$ от деления на $5$, $13$ и $7$ соответственно.

{\bf 5.} Предложите полиномиальный алгоритм нахождения количества натуральных решений диофантова уравнения $ax+by = c$. 

\textit{По какому параметру он полиномиальный?...}







\end{document}


