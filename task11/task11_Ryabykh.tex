\documentclass[a4paper,12pt]{article}

%%% Работа с русским языком
\usepackage[cm]{fullpage}
\usepackage[T2A]{fontenc}
\usepackage[utf8]{inputenc}
\usepackage[english,russian]{babel}
\usepackage{footmisc}
\usepackage[document]{ragged2e}
\usepackage{amsmath,amsfonts,amssymb,mathtools}
\usepackage{framed}
\usepackage{pstricks}
%\usepackage[framed]{ntheorem}
\usepackage{tikz}
\usetikzlibrary{arrows,automata}
\usepackage{cmap}					% поиск в PDF
\usepackage{mathtext} 				% русские буквы в формулах
\usepackage{indentfirst}
\frenchspacing

\newcommand{\vyp}{\hookrightarrow}
\renewcommand{\epsilon}{\varepsilon}
\renewcommand{\phi}{\varphi}
\renewcommand{\kappa}{\varkappa}
\renewcommand{\le}{\leqslant}
\renewcommand{\leq}{\leqslant}
\renewcommand{\ge}{\geqslant}
\renewcommand{\geq}{\geqslant}
\renewcommand{\emptyset}{\varnothing}

%%% Дополнительная работа с математикой
\usepackage{amsmath,amsfonts,amssymb,amsthm,mathtools} % AMS
\usepackage{icomma} % "Умная" запятая: $0,2$ --- число, $0, 2$ --- перечисление

%% Номера формул
%\mathtoolsset{showonlyrefs=true} % Показывать номера только у тех формул, на которые есть \eqref{} в тексте.
%\usepackage{leqno} % Нумереация формул слева

%% Свои команды
\DeclareMathOperator{\sgn}{\mathop{sgn}}

%% Перенос знаков в формулах (по Львовскому)
\newcommand*{\hm}[1]{#1\nobreak\discretionary{}
{\hbox{$\mathsurround=0pt #1$}}{}}



%%% Работа с картинками
\usepackage{graphicx}  % Для вставки рисунков
\graphicspath{{images/}{images2/}}  % папки с картинками
\setlength\fboxsep{3pt} % Отступ рамки \fbox{} от рисунка
\setlength\fboxrule{1pt} % Толщина линий рамки \fbox{}
\usepackage{wrapfig} % Обтекание рисунков текстом

%%% Работа с таблицами
\usepackage{array,tabularx,tabulary,booktabs} % Дополнительная работа с таблицами
\usepackage{longtable}  % Длинные таблицы
\usepackage{multirow} % Слияние строк в таблице

%%% Теоремы
\theoremstyle{plain} % Это стиль по умолчанию, его можно не переопределять.
\newtheorem{theorem}{Теорема}[section]
\newtheorem{proposition}[theorem]{Утверждение}
 
\theoremstyle{definition} % "Определение"
\newtheorem{corollary}{Следствие}[theorem]
\newtheorem{problem}{Задача}[section]
 
\theoremstyle{remark} % "Примечание"
\newtheorem*{nonum}{Решение}

%%% Программирование
\usepackage{etoolbox} % логические операторы

%%% Страница
\usepackage{extsizes} % Возможность сделать 14-й шрифт
\usepackage{geometry} % Простой способ задавать поля
\geometry{top=20mm}
\geometry{bottom=20mm}
\geometry{left=10mm}
\geometry{right=20mm}
 %
\usepackage{fancyhdr} % Колонтитулы
 	\pagestyle{fancy}
 	\renewcommand{\headrulewidth}{0pt}  % Толщина линейки, отчеркивающей верхний колонтитул
\fancypagestyle{firstpage}{
	\rhead{\large{Рябых Владислав, Б05-905}}
}
% 	\lfoot{Нижний левый}
% 	\rfoot{Нижний правый}
% 	\rhead{Верхний правый]}
% 	\chead{Верхний в центре}
% 	\lhead{Верхний левый}
%	\cfoot{Нижний в центре} % По умолчанию здесь номер страницы

\usepackage{setspace} % Интерлиньяж
\onehalfspacing % Интерлиньяж 1.5
%\doublespacing % Интерлиньяж 2
%\singlespacing % Интерлиньяж 1

\usepackage{lastpage} % Узнать, сколько всего страниц в документе.

\usepackage{soul} % Модификаторы начертания

%\usepackage{hyperref}
%\usepackage[usenames,dvipsnames,svgnames,table,rgb]{xcolor}
%\hypersetup{				% Гиперссылки
%    unicode=true,           % русские буквы в раздела PDF
%    pdftitle={Заголовок},   % Заголовок
%    pdfauthor={Автор},      % Автор
%    pdfsubject={Тема},      % Тема
%    pdfcreator={Создатель}, % Создатель
%    pdfproducer={Производитель}, % Производитель
%    pdfkeywords={keyword1} {key2} {key3}, % Ключевые слова
%    colorlinks=true,       	% false: ссылки в рамках; true: цветные ссылки
%    linkcolor=red,          % внутренние ссылки
%    citecolor=black,        % на библиографию
%    filecolor=magenta,      % на файлы
%    urlcolor=cyan           % на URL
%}

\usepackage{csquotes} % Еще инструменты для ссылок

%\usepackage[style=authoryear,maxcitenames=2,backend=biber,sorting=nty]{biblatex}

\usepackage{multicol} % Несколько колонок

\usepackage{pgfplots}
\usepackage{pgfplotstable}
\newcommand{\tbf}{\textbf}


\usepackage[shortlabels]{enumitem}

\newtheorem{task}{\textbf{Задача}}

\newtheorem{innercustomthm}{\textbf{Задача}}
\newenvironment{tasknum}[1]
{\renewcommand\theinnercustomthm{#1}\innercustomthm}
{\endinnercustomthm}

\newtheorem*{solution}{\textbf{Решение}}
\newcommand{\Ra}{\Rightarrow}
\newcommand{\La}{\Leftarrow}
\newcommand{\ra}{\rightarrow}
\newcommand{\LRa}{\Leftrightarrow}
\newcommand{\n}{\mathbb}
\newcommand{\Le}{\leqslant}
\newcommand{\Ge}{\geqslant}


\renewcommand{\inf}{\infty}
\newcommand{\ol}{\overline}

\newcommand{\bigline}{\noindent\makebox[\linewidth]{\rule{\paperwidth}{0.4pt}}}

\usetikzlibrary{fit}
\newcommand{\dost}{\overset{*}{\vdash}}
\newcommand{\vyv}{\overset{*}{\Rightarrow}}


\usepackage{capt-of}
\usepackage{tikz-qtree}
\usepackage{systeme}

\newcommand{\polysv}{\leq_p}
\def\coNP{{\mathbf{\text{\textbf{co--}}\mathcal{NP}}}}
\newcommand{\NP}{\mathcal{NP}}
\renewcommand{\P}{\mathcal{P}}

\usepackage{algorithm}
%\usepackage{algpseudocode}
\usepackage[noend]{algpseudocode}

\newcommand{\prob}[1]{\mathbb{P}\left\{#1\right\}}
\newcommand{\expected}[1]{\mathbb{E}\left\{#1\right\}}

\DeclareMathOperator{\circul}{circ}

\begin{document}
	
	\thispagestyle{firstpage}
	
	\begin{center}
		\textbf{\Large{Алгоритмы и модели вычислений. \\ Домашнее задание № 11}}
	\end{center}
	
	
\begin{tasknum}{0}
	Помимо очного семинара, запись предыдущего года я тоже посмотрел, так что вычисляю $\left(\dfrac{\text{N}}{41}\right)$, где $N=13$~---~мой номер в списке.
\end{tasknum}
	
\begin{solution}
	\[\left(\dfrac{13}{41}\right) = \left(-1\right)^{\frac{41 - 1}{2} \cdot \frac{13 - 1}{2}} \cdot \left(\dfrac{41}{13}\right) = \left(\dfrac{41}{13}\right) = \left(\dfrac{2}{13}\right) = (-1)^{\frac{13^2-1}{8}} = (-1)^{\frac{168}{8}} = (-1)^{21} = -1\]
\end{solution}	

\begin{task}
	Имеются окрашенные прямоугольные таблички трёх типов: черный квадрат размера $2\times 2$, белый квадрат того же размера и серый прямоугольник $2\times 1$ (последний можно поворачивать на $90^\circ$). Нужно подсчитать число способов $T(n)$ замостить полосу размера $2\times n$. Найдите явную аналитическую формулу для $T(n)$ и вычислите $T(30000)$ по модулю $31$. 
\end{task}

\begin{solution}
	
	Для нахождения рекурсивной формулы воспользуемся следующим наблюдением: так как размеры наших табличек не превышают $2\times 2$, то принципиально у нас есть $2$ способа замостить край полосы: 

\begin{enumerate}
	\item замостить один крайний столбец: на это у нас есть всего один способ~---~поставить серый прямоугольник $2\times 1$
	\item замостить сразу два крайних столбца: на это у нас есть $3$ способа~---~чёрный квадрат $2\times 2$, белый квадрат $2\times 2$, два серых прямоугольника $1\times 2$ каждый (случай двух серых прямоугольников $2\times 1$ покрывается предыдущим пунктом, поэтому тут не рассматривается)
\end{enumerate}

Таким образом, получаем $T(n) \hm{=} T(n-1) + 3T(n-2)$

Решим линейное рекуррентное соотношение: для начала запишем характеристическое уравнение
$$\lambda^2 \hm{=} \lambda + 3 \ \ \Ra \ \ \lambda_{1, 2} \hm{=} \dfrac{1 \pm \sqrt{13} }{2}$$
Таким образом $T(n) \hm{=} C_1 \left( \dfrac{1 + \sqrt{13} }{2}\right) ^n + C_2 \left( \dfrac{1 - \sqrt{13} }{2}\right) ^n$, подберём константы из начальных условий: при $n \hm{=} 0$ у нас есть 1 способ замостить ленту, при $n\hm{=}1$~---~также 1 способ, то есть $T(0) \hm{=} T(1) \hm{=} 1$, тогда
$$C_1 + C_2 \hm{=} 1 \ \ \ \ \ \ C_1 \cdot \dfrac{1 + \sqrt{13} }{2} + C_2 \cdot \dfrac{1 - \sqrt{13} }{2} \hm{=} 1$$
Решая, получаем $C_1 \hm{=} \dfrac{1}{2} + \dfrac{1}{2\sqrt{13}} \hm{=} \dfrac{1}{\sqrt{13}} \lambda_{1}$ и $C_2 \hm{=} \dfrac{1}{2} - \dfrac{1}{2\sqrt{13}} \hm{=} - \dfrac{1}{\sqrt{13}} \lambda_{2}$

И, окончательно: $$T(n) \hm{=} \dfrac{1}{\sqrt{13}} \cdot \left( \dfrac{1 + \sqrt{13} }{2}\right) ^{n+1} - \dfrac{1}{\sqrt{13}} \cdot \left( \dfrac{1 - \sqrt{13} }{2}\right) ^{n+1}$$

$T(n) = T(n-1) + 3T(n-2) = 4T(n-2)+3T(n-3) = 7T(n-3)+12T(n-4) = 19T(n-4)+21T(n-5) = 40T(n-5) + 57T(n-6) \underset{31}{\equiv} 9T(n-5)+26T(n-6) = 35T(n-6) + 27T(n-7) \underset{31}{\equiv} 4T(n-6) + 27T(n-7) = 31T(n-7) + 12T(n-8) \underset{31}{\equiv} 12T(n-8)$

Таким образом мы получили равенство $T(n) \underset{31}{\equiv} 12T(n-8)$. Тогда, так как $T(0) = 1 \underset{31}{\equiv} 1$, то

$T(30000) \underset{31}{\equiv} 12T(30000-8) \underset{31}{\equiv} \ldots \underset{31}{\equiv} 12^{\frac{30000}{8}}T(0) \underset{31}{\equiv} 12^{3750} = 12^{30 \cdot 125} = (12^{30})^{125} \overset{\text{по МТФ}}{\underset{31}{\equiv}} 1^{125} = 1$

Итого получили, что $T(30000) \underset{31}{\equiv} 1$

\end{solution}

\begin{task}
	Выполните задачи 1, Д-1 из приложенного файла (все по 1 баллу).
\end{task}

\begin{solution}
	
	\begin{enumerate}
		\item \begin{enumerate}
			\item Посчитаем все пути длины 2, начинающиеся в вершине 1. Это пути $1\ra x \ra y$, где $x \in \{1, 4\}, y \in \{1, 2, 3, 4\}$ и пути $1\ra z \ra w$, где $z \in \{2, 3\}, w\in \{1, 4\}$. Их всего 12: $g(2) = 12$.
			
			\begin{equation*}
			A = \left(
			\begin{array}{cccc}
			1 & 1 & 1 & 1\\
			1 & 0 & 0 & 1\\
			1 & 0 & 0 & 1\\
			1 & 1 & 1 & 1
			\end{array}
			\right) \Ra A^2 = \left(
			\begin{array}{cccc}
			4 & 2 & 2 & 4\\
			2 & 2 & 2 & 2\\
			2 & 2 & 2 & 2\\
			4 & 2 & 2 & 4
			\end{array}
			\right)
			\end{equation*}
			
			Заметим, что $g(1)$ совпадает с суммой элементов первой строки матрицы $A^1$, а $g(2)$ совпадает с суммой элементов первой строки матрицы $A^2$. Докажем по индукции, что на самом деле для любого $n\in \mathbb{N}$ верно, что $ a^{(n)}_{ij}$~---~элемент матрицы $A^n$, находящийся на $i$-й строке в $j$-м столбце, численно равен количеству путей из вершины $i$ в вершину $j$, имеющих длину $n$.
			
			\begin{itemize}
				\item База очевидна: при $n=1 \ \ A^n=A$~---~матрица смежности графа, которая по определению является тем, что нам нужно
				\item Переход: пусть для $k-1$ верно, тогда $a^{(k)}_{ij} = \displaystyle \sum a^{(k-1)}_{ir} \cdot a_{rj}$. В этой сумме пути $i\ra r$ длины $k-1$ для всех вершин $r$ графа просто достраиваются до путей $i\ra r \ra j$, которые уже имеют длину $k$ (если же пути $r \ra j$ не существует, то $a_{rj} = 0$ и такие пути не будут учтены). 
				
			\end{itemize}
			
			Таким образом, $g(n) =  \displaystyle \sum_{r=1}^{4} a^{(n)}_{1r}$
			
			\item Обратимся к структуре нашей матрицы $A$ и посмотрим, какие вообще элементы находятся в первой строке матрицы $A^n$.
			
			\begin{itemize}
				\item $a^{(n)}_{11} = a^{(n)}_{14} = g(n-1)$. Первое равенство верно в силу симметрии, а $a^{(n)}_{11} = g(n-1)$ следует из того, что в первом столбце матрицы $A$ все элементы равны единице, поэтому $a^{(n)}_{11} = \displaystyle \sum_{r=1}^{4} a^{(n-1)}_{1r} \cdot a_{r1} = \displaystyle \sum_{r=1}^{4} a^{(n-1)}_{1r} = g(n-1)$.
				
				\item $a^{(n)}_{12} = a^{(n)}_{13} = 2g(n-2)$. Первое равенство верно в силу симметрии, а $a^{(n)}_{12} = 2g(n-2)$ следует из того, что во втором столбце матрицы $A$ 1й и 4й элементы равны единице, в то время как 2й и 3й равны нулю, поэтому $a^{(n)}_{12} = \displaystyle \sum_{r=1}^{4} a^{(n-1)}_{1r} \cdot a_{r2} = a^{(n-1)}_{11} + a^{(n-1)}_{14} = g(n-2) + g(n-2) = 2g(n-2)$.
			\end{itemize}
			
			Тогда, соответственно, сумма элементов первой строки матрицы $A^n$ есть $$g(n) = 2g(n-1) + 4g(n-2)$$
			
			\item $g(n) \mod 29 = (2\cdot (g(n-1) \mod 29) + 4\cdot(g(n-2) \mod 29)) \mod 29$
			
			На каждом шаге мы вычисляем только 3 арифметические операции (два умножения и одно сложение) и одно взятие остатка, это выполняется за линейное время. Также заметим, что так как $g(n-1) \mod 29$ и $g(n-2)$ не превышают 29, а в памяти мы постоянно храним только их, то для этой задачи нам хватает константной памяти. Таким образом трудоёмкость процедуры есть $O(n)$ по времени и $O(1)$ по памяти. Конкретно при вычислении $g(20000) \mod 29$ мы вычисляем приблизительно $20000 \cdot 4 = 80000$ операций
			
			\item Для любого модуля $m$ верно, что $g(n) \mod m$ полностью определяется 2 числами: $g(n-1) \mod m$ и $g(n-2) \mod m$. Значит, всего различных вариантов $g(n) \mod m \le m^2$. Таким образом, на некотором шаге $T \le m^2$ появится пара $g(T-1) \mod m$ и $g(T-2) \mod m$, ранее встречавшаяся в последовательности $\Ra$ будет вычислено $g(T)$, ранее встречавшееся и так далее, последовательность зацикливается.
			
			Вернёмся к нашему модулю 29.
			
			Сложность нахождение периода $O(1)$ как по времени, так и по памяти, так как нам точно достаточно просмотреть и сохранить в памяти $29^2$ элементов последовательности.
			
			Теперь для вычисления $g(n) \mod 29$ нам достаточно вычислить $g(n \mod T) \mod 29$. Так что сначала вычисляем $n \mod T$ за $O(\log n)$, затем просто достаём из памяти значение $g(n \mod T) \mod 29$ (которое мы положили в память при нахождении самого периода $T$). Таким образом, итоговая сложность есть $O(\log n)$ времени и $O(1)$ памяти.
			
		\end{enumerate}
	
	\item
	\begin{enumerate}
		\item 
	 Решим линейное рекуррентное соотношение: для начала запишем характеристическое уравнение
	$$\lambda^2 \hm{=} 2\lambda + 4 \ \ \Ra \ \ \lambda_{1, 2} \hm{=} 1 \pm \sqrt{5}$$
	Таким образом $g(n) \hm{=} C_1 \left(1 + \sqrt{5}\right)^n + C_2 \left(1 - \sqrt{5}\right)^n$, подберём константы из начальных условий: $g(0) \hm{=} 1, g(1) \hm{=} 4$. Тогда окончательно получаем:
	
	$$g(n) \hm{=} \dfrac{1}{\sqrt{10}} \cdot \left( (5+3\sqrt{5})(1+\sqrt{5})^n + (5-3\sqrt{5})(1-\sqrt{5})^n \right)$$
	
	\begin{itemize}
		\item Найдём $10^{-1}$ по модулю 29: $10 \cdot 3 = 30 \underset{29}{\equiv} 1$, так что $10^{-1} = 3$
		\item Найдём $\sqrt5$ по модулю 29: $11^2 = 121 \underset{29}{\equiv} 5$, так что $\sqrt5 \underset{29}{\equiv} 11$
	\end{itemize}
	Теперь получаем $g(n) = 3 \cdot (9 \cdot 12^n + 19^n)$
	
	Также перед явным вычисление $g(n) \mod 29$ необходимо заметить, что 29~---~простое число, так что мы можем воспользоваться МТФ: $12^{28} \underset{29}{\equiv} 1$ и $19^{28} \underset{29}{\equiv} 1$, поэтому $12^n \underset{29}{\equiv} 12^{n\mod 28} \mod 29, 19^n \underset{29}{\equiv} 19^{n\mod 28} \mod 29$ и просто можем вычислить все эти 56 значений и положить в память (что займёт константное время и константную память).
	
	Теперь для вычисления $g(n) \mod 29$ осталось просто вычислить $n \mod 28$ за $O(\log n)$, достать из памяти значения $12^{n\mod 28} \mod 29, 19^{n\mod 28} \mod 29$, вычислить несколько арифметических операций и остаток результата по модулю 29 (результат не может превышать $3 \cdot (9 \cdot 28 + 28))$, так что взятие остатка от результата также есть константа.
	
	Итоговая сложность есть $O(\log n)$ времени и $O(1)$ памяти.
	
	$A = g(20000) \underset{29}{\equiv} g(8) = 3\cdot(9\cdot 12^8 + 19^8) \underset{29}{\equiv} 3\cdot (9 + 25) \underset{29}{\equiv} 3 \cdot 5 = 15$
	
	\item Если же рассматривать другой модуль, в котором 5 не является квадратичным вычетом, то необходимо расширить наше поле таким образом, чтобы в нём $x^2 \underset{p}{\equiv} 5$ было разрешимо. 
	
	\begin{itemize}
		\item Найдём $10^{-1}$ по модулю 23: $10 \cdot 7 = 70 \underset{23}{\equiv} 1$, так что $10^{-1} = 7$
	\end{itemize}
	
	Теперь получаем $g(n) = 7 \cdot \left((5+3x)(1+x)^n+(5-3x)(1-x)^n\right)$
	
	Так как $x^2 \underset{23}{\equiv} 5$, то в результате перемножения многочленов мы всё равно получаем многочлен первой степени.
	
	Заметим, что так как мы живём в кольце вычетов по модулю 23, то перемножение многочленов первой степени занимает константное время. В результате же нам надо перемножить $2n$ многочленов первой степени, так что итоговая сложность алгоритма будет $O(n)$ по времени и $O(1)$ по памяти.
	
	\end{enumerate}	
	\end{enumerate}
	
\end{solution}

\begin{task}
	 Верно ли, что существует такая функция $f: \mathbb{N} \rightarrow  \mathbb{N}$, для любых констант $\forall\, c,d> 0$ выполнено $$f(n) = \omega(n^c),\ f(n) = o(2^{nd}),$$ т.~е. функция $f(n)$ растет быстрее любого заданного полинома, но медленнее любой заданной экспоненты?
\end{task}

\begin{solution}
		\item Возьмём функцию $\tilde{f} = n^{\log n}$. Она, очевидно, растёт быстрее  любого заданного полинома. Также она растёт медленнее любой заданной экспоненты (как такой предел считать я не знаю, но вольфрам подтверждает этот факт). Однако $\tilde{f}: \mathbb{N} \rightarrow  \mathbb{R}$, так что возьмём округление вверх.
		
		Итоговая функция $f = \lceil n^{\log n} \rceil$
\end{solution}

\begin{task}
	\begin{enumerate}
	\item Делится ли $4^{1356}-9^{4824}$ на $35$? Делится ли $5^{30000} - 6^{123456}$ на $31$?
	
	\item Найдите обратные $20 \ (\mbox{mod } 79)$, $3 \ (\mbox{mod } 62)$.
	
	\item Найдите все решения уравнения $35x = 10 \ (\mbox{mod } 50)$.
	
	\item Имеет ли решение сравнение $x^2 = 1597 \ (\mbox{mod } 3911)$
	
	\item Найдите наименьшее натуральное число, имеющее остатки $2$, $3$, $1$ от деления на $5$, $13$ и $7$ соответственно.
	\end{enumerate}
\end{task}

\begin{solution}
	\begin{enumerate}
		\item \begin{enumerate}
			\item Сразу посчитаем $\phi(35) = 35 \cdot (1-\frac{1}{5}) \cdot (1-\frac17) = 24$. Тогда, так как (4, 35) = 1 и (9, 35) = 1, можем пользоваться теоремой Лагранжа \[4^{1356}-9^{4824} = (4^{24})^{56} \cdot 4^{12} - (9^{24})^{201} \underset{35}{\equiv} 1 \cdot 4^{12} - 1 = 2^{24} - 1 \underset{35}{\equiv} 1 - 1 = 0\]
			
			Так что $4^{1356}-9^{4824}$ действительно делится на $35$
			\item 31~---~простое число, так что будем пользоваться МТФ
			\[5^{30000} - 6^{123456} = (5^{30})^{1000} - (6^{30})^{4115} \cdot 6^6 \underset{31}{\equiv} 1 - 1\cdot 6^6 = 1 - 36^3 \underset{31}{\equiv} 1 - 5^3 = -124 \underset{31}{\equiv} 0\]
			
			Так что $5^{30000} - 6^{123456}$ действительно делится на $31$
		\end{enumerate}
	
	\item
	\begin{enumerate}
	\item Легко заметить, что $20 \cdot 4 = 80 \underset{79}{\equiv} 1 \Ra$ обратный к 20 по модулю 79 есть 4
	\item Легко заметить, что $3 \cdot 21 = 63 \underset{62}{\equiv} 1 \Ra$ обратный к 3 по модулю 62 есть 21
	\end{enumerate} 
	
	\item Для этого решим диофантово уравнение $35x + 50y = 10 \LRa 7x + 10y = 2$.
	
	(7, 10) = 1 так что решение в целых числах существует, для его нахождения воспользуемся расширенным алгоритмом Евклида
	
	\begin{tabular}{|c|c|c|c|c|c|}
		\hline
		x& 1 & 0 & -1 & 3 & 6 \\
		\hline
		y& 0 & 1 & 1 & -2 & -4 \\
		\hline
		7x+10y& 7 & 10 & 3 & 1 & 2 \\
		\hline
	\end{tabular}
	
	Таким образом, $x = 6 + 10m, m \in \mathbb{Z}$ задаёт все решения уравнения $35x = 10 \ (\mbox{mod } 50)$.
	
	\item Посчитаем символ Лежандра $\left(\dfrac{1597}{3911}\right)$, если он будет равен 1, то 1591~---~квадратичный вычет и решение есть, если же символ Лежандра окажется равным -1, то невычет и решений у сравнения нет.
	
\begin{multline*}\left(\dfrac{1597}{3911}\right) = (-1)^{\frac{3910}{2}\ \cdot \frac{1596}{2}} \cdot \left(\dfrac{3911}{1597}\right) = \left(\dfrac{717}{1597}\right) = (-1)^{\frac{716}{2} \cdot \frac{1596}{2}} \cdot \left(\dfrac{1597}{717}\right) = \\ = \left(\dfrac{163}{717}\right) = (-1)^{\frac{162}{2} \cdot \frac{716}{2}} \cdot \left(\dfrac{717}{163}\right) = \left(\dfrac{65}{163}\right) = (-1)^{\frac{64}{2} \cdot \frac{162}{2}} \cdot \left(\dfrac{163}{65}\right) = \\ = \left(\dfrac{33}{65}\right) = (-1)^{\frac{32}{2} \cdot \frac{64}{2}} \cdot \left(\dfrac{65}{33}\right) = \left(\dfrac{-1}{33}\right) = (-1)^{\frac{32}{2}} = 1
\end{multline*}
	
$\left(\dfrac{1597}{3911}\right) = 1$, так что 1597 действительно является квадратичным вычетом по модулю 3911, а следовательно сравнение $x^2 = 1597 \ (\mbox{mod } 3911)$ имеет решение

\item Воспользуемся КТО, чтобы найти множество целых чисел $x$, таких что $\left\{
\begin{aligned}
x &\underset{5}{\equiv} 2 \\
x &\underset{13}{\equiv} 3\\
x &\underset{7}{\equiv} 1\\
\end{aligned}
\right.
$

Сначала найдём обратные элементы:

\begin{itemize}
	\item по модулю 5: $91^{-1} = 1^{-1} = 1$
	\item по модулю 13: $35^{-1} = 9^{-1} = 3$, так как $9 \cdot 3 = 27 \underset{13}{\equiv} 1$
	\item по модулю 7: $65^{-1} = 2^{-1} = 4$, так как $2 \cdot 4 = 8 \underset{7}{\equiv} 1$
\end{itemize}

Тогда $x  \underset{455}{\equiv} 2 \cdot 1 \cdot 91 + 3 \cdot 3 \cdot 35 + 1 \cdot 4 \cdot 65 = 757 \underset{455}{\equiv} 302$

Таким образом условию задачи удовлетворяют числа $x = 302 + 455n, n \in \mathbb{Z}$.

Наименьшее натуральное из них $m = 302$.
	
	\end{enumerate}
\end{solution}

\begin{task}
	Предложите полиномиальный алгоритм нахождения количества натуральных решений диофантова уравнения $ax+by = c$. 
	
	\textit{По какому параметру он полиномиальный?...}
\end{task}

\begin{solution}
	Если $d = (a, b)$ не делит $c$, то решений нет, что проверяется за $poly(|a|+|b|)$, если вычислять НОД с помощью алгоритма Евклида.
	
	Если делит, то пользуемся расширенным алгоритмом Евклида, который также работает за полином от длины входа и выдаёт нам частное решение уравнения $(x_0, y_0)$, которое, в свою очередь, также полиномиально от длины входа.
	
	Теперь мы можем сказать, что решением нашего уравнения является бесконечное подмножество целых чисел: $(x_0 + md, y_0 - md), m \in \mathbb{Z}$.
	
	Чтобы определить количество решений в натуральных числах нам необходимо найти количество таких значений $m \in \mathbb{Z}$, что выполняется $\left\{
	\begin{aligned}
	x_0 + md > 0 \\
	y_0 - md > 0\\
	\end{aligned}
	\right.
	 \Ra \left\{
	 \begin{aligned}
	 m > -\dfrac{x_0}{d} \\
	 m > \dfrac{y_0}{d}\\
	 \end{aligned}
	 \right.$

Таким образом мы вычисляем 2 значения $-\dfrac{x_0}{d}, \dfrac{y_0}{d}$, что происходит за $poly(|a|+|b| + |c|)$, так как все входящие значения занимают полином от длины входа. После чего выводим количество значений $m$, удовлетворяющих условиям из системы выше. Для этого достаточно взять модуль от разности двух выше вычисленных значений и округлить полученное значение вниз.
\end{solution}

\end{document}