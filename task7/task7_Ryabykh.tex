\documentclass[a4paper,12pt]{article}

%%% Работа с русским языком
\usepackage[cm]{fullpage}
\usepackage[T2A]{fontenc}
\usepackage[utf8]{inputenc}
\usepackage[english,russian]{babel}
\usepackage{footmisc}
\usepackage[document]{ragged2e}
\usepackage{amsmath,amsfonts,amssymb,mathtools}
\usepackage{framed}
\usepackage{pstricks}
%\usepackage[framed]{ntheorem}
\usepackage{tikz}
\usetikzlibrary{arrows,automata}
\usepackage{cmap}					% поиск в PDF
\usepackage{mathtext} 				% русские буквы в формулах
\usepackage{indentfirst}
\frenchspacing

\newcommand{\vyp}{\hookrightarrow}
\renewcommand{\epsilon}{\varepsilon}
\renewcommand{\phi}{\varphi}
\renewcommand{\kappa}{\varkappa}
\renewcommand{\le}{\leqslant}
\renewcommand{\leq}{\leqslant}
\renewcommand{\ge}{\geqslant}
\renewcommand{\geq}{\geqslant}
\renewcommand{\emptyset}{\varnothing}

%%% Дополнительная работа с математикой
\usepackage{amsmath,amsfonts,amssymb,amsthm,mathtools} % AMS
\usepackage{icomma} % "Умная" запятая: $0,2$ --- число, $0, 2$ --- перечисление

%% Номера формул
%\mathtoolsset{showonlyrefs=true} % Показывать номера только у тех формул, на которые есть \eqref{} в тексте.
%\usepackage{leqno} % Нумереация формул слева

%% Свои команды
\DeclareMathOperator{\sgn}{\mathop{sgn}}

%% Перенос знаков в формулах (по Львовскому)
\newcommand*{\hm}[1]{#1\nobreak\discretionary{}
{\hbox{$\mathsurround=0pt #1$}}{}}



%%% Работа с картинками
\usepackage{graphicx}  % Для вставки рисунков
\graphicspath{{images/}{images2/}}  % папки с картинками
\setlength\fboxsep{3pt} % Отступ рамки \fbox{} от рисунка
\setlength\fboxrule{1pt} % Толщина линий рамки \fbox{}
\usepackage{wrapfig} % Обтекание рисунков текстом

%%% Работа с таблицами
\usepackage{array,tabularx,tabulary,booktabs} % Дополнительная работа с таблицами
\usepackage{longtable}  % Длинные таблицы
\usepackage{multirow} % Слияние строк в таблице

%%% Теоремы
\theoremstyle{plain} % Это стиль по умолчанию, его можно не переопределять.
\newtheorem{theorem}{Теорема}[section]
\newtheorem{proposition}[theorem]{Утверждение}
 
\theoremstyle{definition} % "Определение"
\newtheorem{corollary}{Следствие}[theorem]
\newtheorem{problem}{Задача}[section]
 
\theoremstyle{remark} % "Примечание"
\newtheorem*{nonum}{Решение}

%%% Программирование
\usepackage{etoolbox} % логические операторы

%%% Страница
\usepackage{extsizes} % Возможность сделать 14-й шрифт
\usepackage{geometry} % Простой способ задавать поля
\geometry{top=20mm}
\geometry{bottom=20mm}
\geometry{left=10mm}
\geometry{right=20mm}
 %
\usepackage{fancyhdr} % Колонтитулы
 	\pagestyle{fancy}
 	\renewcommand{\headrulewidth}{0pt}  % Толщина линейки, отчеркивающей верхний колонтитул
\fancypagestyle{firstpage}{
	\rhead{\large{Рябых Владислав, Б05-905}}
}
% 	\lfoot{Нижний левый}
% 	\rfoot{Нижний правый}
% 	\rhead{Верхний правый]}
% 	\chead{Верхний в центре}
% 	\lhead{Верхний левый}
%	\cfoot{Нижний в центре} % По умолчанию здесь номер страницы

\usepackage{setspace} % Интерлиньяж
\onehalfspacing % Интерлиньяж 1.5
%\doublespacing % Интерлиньяж 2
%\singlespacing % Интерлиньяж 1

\usepackage{lastpage} % Узнать, сколько всего страниц в документе.

\usepackage{soul} % Модификаторы начертания

%\usepackage{hyperref}
%\usepackage[usenames,dvipsnames,svgnames,table,rgb]{xcolor}
%\hypersetup{				% Гиперссылки
%    unicode=true,           % русские буквы в раздела PDF
%    pdftitle={Заголовок},   % Заголовок
%    pdfauthor={Автор},      % Автор
%    pdfsubject={Тема},      % Тема
%    pdfcreator={Создатель}, % Создатель
%    pdfproducer={Производитель}, % Производитель
%    pdfkeywords={keyword1} {key2} {key3}, % Ключевые слова
%    colorlinks=true,       	% false: ссылки в рамках; true: цветные ссылки
%    linkcolor=red,          % внутренние ссылки
%    citecolor=black,        % на библиографию
%    filecolor=magenta,      % на файлы
%    urlcolor=cyan           % на URL
%}

\usepackage{csquotes} % Еще инструменты для ссылок

%\usepackage[style=authoryear,maxcitenames=2,backend=biber,sorting=nty]{biblatex}

\usepackage{multicol} % Несколько колонок

\usepackage{pgfplots}
\usepackage{pgfplotstable}
\newcommand{\tbf}{\textbf}


\usepackage[shortlabels]{enumitem}

\newtheorem{task}{\textbf{Задача}}

\newtheorem{innercustomthm}{\textbf{Задача}}
\newenvironment{tasknum}[1]
{\renewcommand\theinnercustomthm{#1}\innercustomthm}
{\endinnercustomthm}

\newtheorem*{solution}{\textbf{Решение}}
\newcommand{\Ra}{\Rightarrow}
\newcommand{\La}{\Leftarrow}
\newcommand{\ra}{\rightarrow}
\newcommand{\LRa}{\Leftrightarrow}
\newcommand{\n}{\mathbb}
\newcommand{\Le}{\leqslant}
\newcommand{\Ge}{\geqslant}


\renewcommand{\inf}{\infty}
\newcommand{\ol}{\overline}

\newcommand{\bigline}{\noindent\makebox[\linewidth]{\rule{\paperwidth}{0.4pt}}}

\usetikzlibrary{fit}
\newcommand{\dost}{\overset{*}{\vdash}}
\newcommand{\vyv}{\overset{*}{\Rightarrow}}


\usepackage{capt-of}
\usepackage{tikz-qtree}
\usepackage{systeme}

\newcommand{\polysv}{\leq_p}
\def\coNP{{\mathbf{\text{\textbf{co--}}\mathcal{NP}}}}
\newcommand{\NP}{\mathcal{NP}}
\renewcommand{\P}{\mathcal{P}}

\usepackage{algorithm}
%\usepackage{algpseudocode}
\usepackage[noend]{algpseudocode}

\begin{document}
	
	\thispagestyle{firstpage}
	
	\begin{center}
		\textbf{\Large{Алгоритмы и модели вычислений. \\ Домашнее задание № 7}}
	\end{center}
	
\begin{tasknum}{1}
	\begin{enumerate}
		\item Докажите, что $NC^d \subseteq AC^d \subseteq NC^{d+1}$.
		\item Докажите, что $\bigcup\limits_{d=1}^{\infty} AC^d = NC$.
\end{enumerate}
\end{tasknum}

\begin{solution}
	\begin{enumerate}
		\item Вложение $NC^d \subseteq AC^d$ очевидно: если некоторый язык распознаётся схемами с двумя входными литералами с глубиной $O(\log^d n)$, то он распознаётся схемами с произвольным количеством входных литералов с глубиной $O(\log^d n)$ (можно просто взять ту же последовательность схем)
		
		Докажем теперь вложение $AC^d \subseteq NC^{d+1}$: для этого сначала покажем, как мы будем разворачивать вершины $\vee$ и $\wedge$ произвольных схем из $AC^d$: 
		
		Изначально имеем следующую схему:
		
		\begin{center}
			\begin{tikzpicture}[->,shorten >=1pt,auto,node distance=2.8cm]
			\node[state]                (A)                  {$x_1$};
			\node[state]              (B) [right of=A]     {$x_2$};
			\node[state]                        (C) [right of=B]     {$x_3$};
			\node[state]                        (E) [right of=C]     {$x_4$};
			\node[state]                        (D) [below right of=B]   {$\vee$};
			
			\path
			(A) edge node [swap] {} (D)
			(B) edge  node {} (D)
			(E) edge  node {} (D)
			(C) edge             node {} (D);
			\end{tikzpicture}
		\end{center}
		
		Разворачиваем её следующим образом:
		
		\begin{center}
			\begin{tikzpicture}[->,shorten >=1pt,auto,node distance=2.8cm]
			\node[state]                (A)                  {$x_1$};
			\node[state]              (B) [right of=A]     {$x_2$};
			\node[state]                        (C) [right of=B]     {$x_3$};
			\node[state]                        (F) [right of=C]     {$x_4$};
			\node[state]                        (D) [below right of=A]   {$\vee$};
			\node[state]                        (E) [below left of=F]   {$\vee$};
			\node[state]                        (G) [below right of=D]   {$\vee$};
			
			\path
			(A) edge node [swap] {} (D)
			(B) edge  node {} (D)
			(F) edge  node {} (E)
			(E) edge  node {} (G)
			(D) edge  node {} (G)
			(C) edge             node {} (E);
			\end{tikzpicture}
		\end{center}
		
		Докажем, что если мы в произвольной схеме глубины $O(\log^d n)$ развернём все вершины $\vee$ и $\wedge$ входящей степени $> 2$, то мы получим схему глубиной $O(\log^{d+1} n)$. Пусть некоторая вершина исходной схемы имеет входную степень $k$. Тогда заметим, что так как в нашей развёртке количество входов с каждым уравнем сокращается вдвое, то эта вершина преобразуется в дерево глубины $\log_2 k$. Таким образом глубина исходной схемы $O(\log^d n)$, а глубина каждого из её уровней после преобразования есть $O(\log n)$, то есть итоговая глубина преобразованной схемы будет $O(\log^d n) \cdot O(\log n) = O(\log^{d+1} n)$, таким образом $AC^d \subseteq NC^{d+1}$, что и требовалось.
		
		\item По доказанному в прошлом пункте $NC^d \subseteq AC^d \subseteq NC^{d+1}$, тогда $NC = \bigcup\limits_{d=1}^{\infty} NC^d \subseteq \bigcup\limits_{d=1}^{\infty} AC^d \subseteq \bigcup\limits_{d=1}^{\infty} NC^{d+1} \subseteq \bigcup\limits_{d=1}^{\infty} NC^d = NC$
		
		То есть $NC \subseteq \bigcup\limits_{d=1}^{\infty} AC^d \subseteq NC$, то есть  $\bigcup\limits_{d=1}^{\infty} AC^d = NC$, что и требовалось.
	\end{enumerate}
\end{solution}

\begin{tasknum}{2}
	Докажите, что язык $PAL = \{a \ | \ a = a^R\}$, где $a^R$~---~слово $a$, записанное в обратном порядке, лежит в $AC^0$
\end{tasknum}

\begin{solution}
	$AC^0$~---~язык слов, распознаваемых схемами типа $AC$ и имеющих при этом константную глубину. Зафиксируем базис $\neg, \vee, \wedge$. Так как $a \oplus b = (a \wedge \neg b) \vee (\neg a \wedge b)$, то добавление в базис символа $\oplus$ увеличивает размер схемы в константу раз (будем использовать $\oplus$, так как предикат <<$a_i$ совпадает с $a_j$>> можно записать в виде $\neg(a_i \oplus a_j)$)
	
	Таким образом $a=a_1a_2\ldots a_n \in PAL \LRa \neg(a_1 \oplus a_n) \wedge \neg(a_2 \oplus a_{n-1}) \wedge \ldots = 1$, и, собственно, мы можем вычислить эту функцию на схеме типа $AC$ константной глубины:
	
	
	\begin{center}
	\begin{tikzpicture}[->,shorten >=1pt,auto,node distance=2.8cm]
	\node[state]                (A)                  {$a_1$};
	\node[state]              (B) [right of=A]     {$a_2$};
	\node[state]                        (C) [right of=B]     {$a_3$};
	\node[state]                        (F) [right of=C]     {$a_4$};
	\node[state]                        (a5) [right of=F]     {$a_5$};
	\node[state]                        (a6) [right of=a5]     {$a_6$};
	\node[state]                        (op) [below left of=a6]   {$\oplus$};
	\node[state]                        (D) [below right of=A]   {$\oplus$};
	\node[state]                        (E) [below right of=C]   {$\oplus$};
	\node[state]                        (neg1) [below of=D]   {$\neg$};	
	\node[state]                        (neg2) [below of=E]   {$\neg$};	
	\node[state]                        (neg3) [below of=op]   {$\neg$};	
	\node[state]                        (G) [below of=neg2]   {$\wedge$};
	
	\path
	(A) edge node [swap] {} (D)
	(B) edge  node {} (op)
	(a5) edge  node {} (op)
	(a6) edge  node {} (D)
	(op) edge  node {} (neg3)
	(F) edge  node {} (E)
	(D) edge  node {} (neg1)
	(E) edge  node {} (neg2)
	(neg1) edge  node {} (G)
	(neg2) edge  node {} (G)
	(neg3) edge  node {} (G)
	(C) edge             node {} (E);
	\end{tikzpicture}
\end{center}

Мы получили схему константной длины, которая вычисляет язык $PAL \Ra PAL \in AC^0$, что и требовалось.

\end{solution}


\end{document}