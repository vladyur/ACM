\documentclass[12pt,a5paper,fleqn]{article}
\usepackage[utf8]{inputenc}
\usepackage{amssymb, amsmath, multicol ,wrapfig}
\usepackage[russian]{babel}
\usepackage{graphicx}
\usepackage[shortcuts,cyremdash]{extdash}
\usepackage{wrapfig}
\usepackage{floatflt}
\usepackage{lipsum}
\usepackage{concmath}
\usepackage{euler}

\graphicspath{ {images/} }

\oddsidemargin=-17.9mm
\textwidth=133mm
\headheight=-35.4mm
\textheight=200mm
\parindent=0pt
\tolerance=100
\parskip=6pt
\pagestyle{empty}
\renewcommand{\tg}{\mathop{\mathrm{tg}}\nolimits}
\renewcommand{\ctg}{\mathop{\mathrm{ctg}}\nolimits}
\renewcommand{\arctan}{\mathop{\mathrm{arctg}}\nolimits}
\newcommand{\divisible}{\mathop{\raisebox{-2pt}{\vdots}}}
\renewcommand{\not}{\overline}
\newcommand{\modulo}{\mathop{\mathrm{mod}}\nolimits}




\RequirePackage{caption2}
\renewcommand\captionlabeldelim{}
\newcommand*{\hm}[1]{#1\nobreak\discretionary{}%
\newtheorem{Theorem}{Теорема}
{\hbox{$\mathsurround=0pt #1$}}{}}

\begin{document}

\begin{center}
{ \Large Задание на восьмую неделю.}

\end{center}

{\bf 1.} Докажите, что $\mathcal{RP}\subset \mathcal{NP}$.

{\bf 2.} Докажите, что если $\mathcal{P} = \mathcal{NP}$, то $\mathcal{P} = \mathcal{BPP}$. Докажите, что если $\mathcal{NP} \subseteq $ co-$\mathcal{RP}$, то $\mathcal{ZPP} = \mathcal{NP}$.

{\bf 3.} Покажите, что в задаче сравнения больших файлов, разобранной на семинаре, вероятность ошибки действительно не превосходит $3/4$ при достаточно больших $n$. Оцените, насколько должно быть велико $n$ и покажите, что $n \textrm{ бит} \geq 32$~мегабайта~---~достаточное количество для справедливости оценок.

{\bf 4.} Задача 1 из приложенного файла.

{\bf 5.} Задача 2 из приложенного файла (пункты $(i)$ и $(iv)$). 

\textit{Указание.} В этой задаче может быть полезна лемма Шварца-Зиппеля.

{\bf 6.} Задача 4 из приложенного файла (разберитесь с алгоритмом Каргера по конспекту или любым другим источникам и выполните это упражнение).

{\bf 7.} Докажите, что 2-$CNF$~---~задача из $\mathcal{P}$. Задача 3$(ii)$ из приложенного файла. 

{\bf 8 (Бонусные задачи).} 1) Докажите теорему Татта; 2) Д-1 из файла.


\end{document}
