\documentclass[a4paper,12pt]{article}

%%% Работа с русским языком
\usepackage[cm]{fullpage}
\usepackage[T2A]{fontenc}
\usepackage[utf8]{inputenc}
\usepackage[english,russian]{babel}
\usepackage{footmisc}
\usepackage[document]{ragged2e}
\usepackage{amsmath,amsfonts,amssymb,mathtools}
\usepackage{framed}
\usepackage{pstricks}
%\usepackage[framed]{ntheorem}
\usepackage{tikz}
\usetikzlibrary{arrows,automata}
\usepackage{cmap}					% поиск в PDF
\usepackage{mathtext} 				% русские буквы в формулах
\usepackage{indentfirst}
\frenchspacing

\newcommand{\vyp}{\hookrightarrow}
\renewcommand{\epsilon}{\varepsilon}
\renewcommand{\phi}{\varphi}
\renewcommand{\kappa}{\varkappa}
\renewcommand{\le}{\leqslant}
\renewcommand{\leq}{\leqslant}
\renewcommand{\ge}{\geqslant}
\renewcommand{\geq}{\geqslant}
\renewcommand{\emptyset}{\varnothing}

%%% Дополнительная работа с математикой
\usepackage{amsmath,amsfonts,amssymb,amsthm,mathtools} % AMS
\usepackage{icomma} % "Умная" запятая: $0,2$ --- число, $0, 2$ --- перечисление

%% Номера формул
%\mathtoolsset{showonlyrefs=true} % Показывать номера только у тех формул, на которые есть \eqref{} в тексте.
%\usepackage{leqno} % Нумереация формул слева

%% Свои команды
\DeclareMathOperator{\sgn}{\mathop{sgn}}

%% Перенос знаков в формулах (по Львовскому)
\newcommand*{\hm}[1]{#1\nobreak\discretionary{}
{\hbox{$\mathsurround=0pt #1$}}{}}



%%% Работа с картинками
\usepackage{graphicx}  % Для вставки рисунков
\graphicspath{{images/}{images2/}}  % папки с картинками
\setlength\fboxsep{3pt} % Отступ рамки \fbox{} от рисунка
\setlength\fboxrule{1pt} % Толщина линий рамки \fbox{}
\usepackage{wrapfig} % Обтекание рисунков текстом

%%% Работа с таблицами
\usepackage{array,tabularx,tabulary,booktabs} % Дополнительная работа с таблицами
\usepackage{longtable}  % Длинные таблицы
\usepackage{multirow} % Слияние строк в таблице

%%% Теоремы
\theoremstyle{plain} % Это стиль по умолчанию, его можно не переопределять.
\newtheorem{theorem}{Теорема}[section]
\newtheorem{proposition}[theorem]{Утверждение}
 
\theoremstyle{definition} % "Определение"
\newtheorem{corollary}{Следствие}[theorem]
\newtheorem{problem}{Задача}[section]
 
\theoremstyle{remark} % "Примечание"
\newtheorem*{nonum}{Решение}

%%% Программирование
\usepackage{etoolbox} % логические операторы

%%% Страница
\usepackage{extsizes} % Возможность сделать 14-й шрифт
\usepackage{geometry} % Простой способ задавать поля
\geometry{top=20mm}
\geometry{bottom=20mm}
\geometry{left=10mm}
\geometry{right=20mm}
 %
\usepackage{fancyhdr} % Колонтитулы
 	\pagestyle{fancy}
 	\renewcommand{\headrulewidth}{0pt}  % Толщина линейки, отчеркивающей верхний колонтитул
\fancypagestyle{firstpage}{
	\rhead{\large{Рябых Владислав, Б05-905}}
}
% 	\lfoot{Нижний левый}
% 	\rfoot{Нижний правый}
% 	\rhead{Верхний правый]}
% 	\chead{Верхний в центре}
% 	\lhead{Верхний левый}
%	\cfoot{Нижний в центре} % По умолчанию здесь номер страницы

\usepackage{setspace} % Интерлиньяж
\onehalfspacing % Интерлиньяж 1.5
%\doublespacing % Интерлиньяж 2
%\singlespacing % Интерлиньяж 1

\usepackage{lastpage} % Узнать, сколько всего страниц в документе.

\usepackage{soul} % Модификаторы начертания

%\usepackage{hyperref}
%\usepackage[usenames,dvipsnames,svgnames,table,rgb]{xcolor}
%\hypersetup{				% Гиперссылки
%    unicode=true,           % русские буквы в раздела PDF
%    pdftitle={Заголовок},   % Заголовок
%    pdfauthor={Автор},      % Автор
%    pdfsubject={Тема},      % Тема
%    pdfcreator={Создатель}, % Создатель
%    pdfproducer={Производитель}, % Производитель
%    pdfkeywords={keyword1} {key2} {key3}, % Ключевые слова
%    colorlinks=true,       	% false: ссылки в рамках; true: цветные ссылки
%    linkcolor=red,          % внутренние ссылки
%    citecolor=black,        % на библиографию
%    filecolor=magenta,      % на файлы
%    urlcolor=cyan           % на URL
%}

\usepackage{csquotes} % Еще инструменты для ссылок

%\usepackage[style=authoryear,maxcitenames=2,backend=biber,sorting=nty]{biblatex}

\usepackage{multicol} % Несколько колонок

\usepackage{pgfplots}
\usepackage{pgfplotstable}
\newcommand{\tbf}{\textbf}


\usepackage[shortlabels]{enumitem}

\newtheorem{task}{\textbf{Задача}}

\newtheorem{innercustomthm}{\textbf{Задача}}
\newenvironment{tasknum}[1]
{\renewcommand\theinnercustomthm{#1}\innercustomthm}
{\endinnercustomthm}

\newtheorem*{solution}{\textbf{Решение}}
\newcommand{\Ra}{\Rightarrow}
\newcommand{\La}{\Leftarrow}
\newcommand{\ra}{\rightarrow}
\newcommand{\LRa}{\Leftrightarrow}
\newcommand{\n}{\mathbb}
\newcommand{\Le}{\leqslant}
\newcommand{\Ge}{\geqslant}


\renewcommand{\inf}{\infty}
\newcommand{\ol}{\overline}

\newcommand{\bigline}{\noindent\makebox[\linewidth]{\rule{\paperwidth}{0.4pt}}}

\usetikzlibrary{fit}
\newcommand{\dost}{\overset{*}{\vdash}}
\newcommand{\vyv}{\overset{*}{\Rightarrow}}


\usepackage{capt-of}
\usepackage{tikz-qtree}
\usepackage{systeme}

\newcommand{\polysv}{\leq_p}
\def\coNP{{\mathbf{\text{\textbf{co--}}\mathcal{NP}}}}
\newcommand{\NP}{\mathcal{NP}}
\renewcommand{\P}{\mathcal{P}}

\usepackage{algorithm}
%\usepackage{algpseudocode}
\usepackage[noend]{algpseudocode}

\newcommand{\prob}[1]{\mathbb{P}\left\{#1\right\}}
\newcommand{\expected}[1]{\mathbb{E}\left\{#1\right\}}

\begin{document}
	
	\thispagestyle{firstpage}
	
	\begin{center}
		\textbf{\Large{Алгоритмы и модели вычислений. \\ Домашнее задание № 9}}
	\end{center}
	

\begin{task}
	Докажите, что $\mathcal{RP}\subseteq \mathcal{NP}$.
\end{task}

\begin{solution}
	Будем работать с представлением ВМТ $M(x)$, как МТ с дополнительной лентой случайных битов, которая читается слева направо. Рассмотрим произвольный язык $L \in \mathcal{RP}$, произвольное слово $x \in L$.
	
	По определению класса $\mathcal{RP}: \ x\in L \Ra \prob{M(x) = 1} \ge \frac{1}{2}$, тогда существует, обрабатываемая МТ часть ленты случайных битов $q$, подав которую в качестве сертификата для слова $x$ недетерминированной машине Тьюринга $A(x, q)$ (которая строится как исходная ВМТ $M(x)$, только вместо ленты случайных битов имеет ленту, с которой читает сертификат), мы получим единицу:  $A(x, q) = 1$. Заметим, что так как в определении класса $\mathcal{RP}$ МТ полиномиальная, то, приняв слово $x$, использованная при этом часть ленты также имеет полиномиальную от размера входа длину: $q = poly(|x|)$. Также, так как по определению класса $\mathcal{RP}: \ x\notin L \Ra \prob{M(x) = 0} = 1$, то на любом сертификате $s$ верно, что $\forall s \  \forall x\notin L \vyp A(x, s) = 0$.
	
	Таким образом получаем, что $x \in L \LRa \exists q = poly(|x|): A(x, q) = 1$, где $A(x, q)$ вычислима за $poly(|x|)$. Что и есть определение класса $\NP$, так что $L \in \NP$, а следовательно $\mathcal{RP}\subseteq \mathcal{NP}$.
\end{solution}


\begin{task}
	\begin{enumerate}
		\item Докажите, что если $\mathcal{P} = \mathcal{NP}$, то $\mathcal{P} = \mathcal{BPP}$.
		\item Докажите, что если $\mathcal{NP} \subseteq $ co-$\mathcal{RP}$, то $\mathcal{ZPP} = \mathcal{NP}$.
	\end{enumerate}
\end{task}

\begin{solution}
\begin{enumerate}
	\item \begin{itemize}
		\item 
	Из семинара мы знаем, что $\P \subseteq \mathcal{RP}$; а по предыдущей задаче, что $\mathcal{RP}\subseteq \mathcal{NP}$. Тогда, в силу предположения $\mathcal{P} = \mathcal{NP}$, получаем $\P = \mathcal{RP} = \NP$. Тогда $\P = \mathcal{RP} \subseteq \mathcal{BPP}$.
	
	\item
	Также мы знаем, что $\mathcal{BPP} \subseteq \Pi_2 \cap \Sigma_2$. Тогда $\mathcal{BPP} \subseteq \mathcal{PH}$. Однако, как мы знаем, $\mathcal{P} = \mathcal{NP} \LRa \mathcal{P} = \mathcal{NP} = \mathcal{PH}$, то есть отсюда получаем $\mathcal{BPP} \subseteq \mathcal{P}$.
\end{itemize}
	
	Таким образом, из предположения $\mathcal{P} = \mathcal{NP}$ следует $\mathcal{P} = \mathcal{BPP}$.
	
	\item $\mathcal{ZPP} \subseteq \mathcal{RP}$ по своему определению, а $\mathcal{RP} \subseteq \mathcal{NP}$ по первой задаче. Тогда, если $\mathcal{NP} \subseteq $ co-$\mathcal{RP}$, то $\mathcal{ZPP} \subseteq \mathcal{RP} \subseteq \NP \subseteq $ co-$\mathcal{RP}$.
	
	В таком случае $L \in \mathcal{RP} \Ra L \in $ co-$\mathcal{RP} \Ra \ol{L} \in \mathcal{RP} \Ra \ol{\ol{L}} = L \in \mathcal{RP}$. Тогда co-$\mathcal{RP} \subseteq \mathcal{RP}$ и получаем, что $\mathcal{RP} = \NP = $ co-$\mathcal{RP}$, то есть  $\mathcal{ZPP} \subseteq \mathcal{RP} \cap $ co-$\mathcal{RP} = \NP$
	
	Осталось доказать, что  $\NP = \mathcal{RP} \cap $ co-$\mathcal{RP} \subseteq \mathcal{ZPP}$. Возьмём произвольный язык $L \in \mathcal{RP}$. Для того, чтобы доказать, что $L \in \mathcal{ZPP}$ будем запускать ВМТ как для $L$, так и для $\ol{L}$. В таком случае ВМТ $A$ проверяет, что $x \in L$ (и безошибочно определяет, что $x \notin L$), а ВМТ $B$, что $x \in \ol{L}$ (и, таким образом, безошибочно определяет, что $x \in L$). Тогда мы будем запускать эти 2 ВМТ, пока одна из них не выдаст 0 в то время как другая выдаст 1 (в таком случае мы безошибочно можем сказать, что если $A(x) = 1 \wedge B(x) = 0$, то $x \in L$ и, если $A(x) = 0 \wedge B(x) = 1$, то $x \notin L$). Матожидание количества запусков равняется 2 (так как вероятность остановиться есть $\frac{1}{2}$), так что матожидание времени работы полиномиально.
	
	Таким образом получаем, что $\NP = \mathcal{RP} \cap $ co-$\mathcal{RP} \subseteq \mathcal{ZPP}$, чего нам и не хватало для доказательства $\mathcal{ZPP} = \mathcal{NP}$.
\end{enumerate}
\end{solution}

\begin{task}
	Покажите, что в задаче сравнения больших файлов, разобранной на семинаре, вероятность ошибки действительно не превосходит $3/4$ при достаточно больших $n$. Оцените, насколько должно быть велико $n$ и покажите, что $n \textrm{ бит} \geq 32$~мегабайта~---~достаточное количество для справедливости оценок.
\end{task}

\begin{solution}
	Оценка с семинара на количество <<плохих>> простых чисел: $k \le \ln 2 \cdot \dfrac{n}{\ln n}$.
	
	Оценка с семинара на количество простых чисел: $0.99 \cdot \dfrac{k}{\ln k} \le prime(k) \le 1.01 \cdot \dfrac{k}{\ln k}$.
	
	Тогда на $[n, 2n]$ как минимум $0.99 \cdot \dfrac{2n}{\ln 2n} - 1.01 \cdot \dfrac{n}{\ln n}$ простых чисел и, таким образом, вероятность выбрать <<плохое>> простое число $P \le \dfrac{\ln 2 \cdot \dfrac{n}{\ln n}}{0.99 \cdot \dfrac{2n}{\ln 2n} - 1.01 \cdot \dfrac{n}{\ln n}}$
	
	Тогда $P \le \dfrac{3}{4} \Ra \dfrac{\dfrac{\ln 2}{\ln n}}{\dfrac{2 \cdot 0.99}{\ln 2n} - \dfrac{1.01}{\ln n}} \le \dfrac{3}{4} \Ra n \ge 5.1 \cdot 10^{12}$
	
	Однако 32 мегабайта это приблизительно $2.6 \cdot 10^{8}$ бит, так что наша оценка оказалась слишком грубой. Возьмём тогда в качестве оценки на количество простых чисел $0.999 \cdot \dfrac{k}{\ln k} \le prime(k) \le 1.0045 \cdot \dfrac{k}{\ln k}$.
	
	Теперь получим $P \le \dfrac{3}{4} \Ra \dfrac{\dfrac{\ln 2}{\ln n}}{\dfrac{2 \cdot 0.999}{\ln 2n} - \dfrac{1.0045}{\ln n}} \le \dfrac{3}{4} \Ra n \ge 2.4 \cdot 10^{8}$
	
	Так что $n \ge 32$~мегабайта $\approx 2.6 \cdot 10^{8}$ бит действительно достаточно
\end{solution}

\begin{task}
	Покажите, что класс $\mathcal{BPP}$ не изменится, если 
	\begin{enumerate}
		\item константу стандарта Монте-Карло $\frac13$  заменить на любое число, строго меньшее $\frac12$
		\item полиномиальное в среднем число шагов заменить на полиномиальное число шагов.
	\end{enumerate}
\end{task}

\begin{solution}
	\begin{enumerate}
		\item пусть новая константа $C$ ($C \in [0, \frac{1}{2}$)), а класс, получившийся при замене исходной константы стандарта Монте-Карло $\frac13$ на $C$, обозначим как $\mathcal{BPP}_C$.
		
		\begin{itemize}
			\item Рассмотрим, случай с  $C \le \frac13$: очевидно, $\mathcal{BPP}_C \subseteq \mathcal{BPP}$. Обратное включение получим, нужное количество раз (пока вероятность ошибки не уменьшится до значения $C$, само количество можно оценить с помощью ) запустив исходную ВМТ для языка $\mathcal{BPP}$ и выбрав наиболее часто встречающийся ответ (что по факту есть функция большинства).
			
			\item Случай с  $C \in [\frac13, \frac{1}{2})$ аналогично: $\mathcal{BPP} \subseteq \mathcal{BPP}_C$. Обратное включение получим, нужное количество раз (пока вероятность ошибки не уменьшится до значения $C$, само количество можно оценить с помощью ) запустив исходную ВМТ для языка $\mathcal{BPP}_C$ и выбрав наиболее часто встречающийся ответ (что по факту есть функция большинства).
		\end{itemize}
	
	\item Обозначим получившийся класс как $\mathcal{BPP}'$. Очевидно, что полиномиальное число шагов также полиномиально в среднем, то есть $\mathcal{BPP}' \subseteq \mathcal{BPP}$. Докажем обратное включение: пусть алгоритм работает в среднем $t(|x|)$ тактов. Мы же будем останавливать его через, допустим, $8 \cdot t(|x|)$ тактов. Тогда по неравенству Маркова вероятность того, что мы прервались и выдали неправильный ответ: $\prob{\text{неправильный | прервались}} = \dfrac{t(|x|)}{8 \cdot t(|x|)} = \dfrac{1}{8}$.
	$\prob{\text{неправильный | не прерывались}} = \dfrac{1}{3}$.
	
	Таким образом:
	
	$\prob{\text{неправильный}} \le \prob{\text{неправильный | прервались}} + \prob{\text{неправильный | не прерывались}} = \dfrac{1}{3} + \dfrac{1}{8} = \dfrac{11}{24} < \dfrac12$.
	
	По предыдущему пункту мы можем заменить исходную константу на любую константу $C$, такую что $C \in [0, \frac{1}{2}$), так что получаем: $\mathcal{BPP} \subseteq \mathcal{BPP}_C = \mathcal{BPP}'$, чего нам и не хватало для доказательства $\mathcal{BPP}' = \mathcal{BPP}$
	
	\end{enumerate}
\end{solution}

\begin{task}
	Задача 2 из приложенного файла (пункты $(i)$ и $(iv)$). 
\end{task}

\begin{solution}
$(i)$: $(AB - C)x = 0 \LRa \forall k \in \ol{1, n} \ \vyp \ \ \displaystyle \sum_{i = 1}^{n} \alpha_{ki} \cdot x_i = 0$
	
	Компоненты вектора $x$ выбираются из множества $\{0, 1, \ldots, N - 1\}$, наш полином имеет степень 1. Тогда по лемме Шварца-Зиппеля вероятность того, что на данном наборе компонент полином обращается в 0 не превышает $\dfrac1N$. Тогда вероятность того, что все полиномы системы обращаются в 0 (а это и есть ошибка в нашей задаче), не превышает $\dfrac1{N^n}$. Тогда, если мы хотим, чтобы вероятность ошибки была меньше заданной вероятности $p$, то $\dfrac1{N^n} < p \Ra N > p^{-\frac1n}$
	
	\vspace{3mm}
	
$(iv)$: $(A(Bx)x)=(Cx)x \LRa ((AB)x)x - (Cx)x = 0 \LRa ((AB-C)x)x = 0$

Это многочлен 2 степени от переменных $x_k$, так что по лемме Шварца-Зиппеля вероятность обращения в 0 не превышает $\dfrac2N$. Тогда $\dfrac2N < p \Ra N > \dfrac2p$

$(A(Bx)y)=(Cx)y \LRa ((AB)x)y - (Cx)y = 0 \LRa ((AB-C)x)y = 0$

Теперь это многочлен 2 степени от $2n$ переменных, которые, впрочем, выбираются также из множества $\{0, 1, \ldots, N - 1\}$, так что по лемме Шварца-Зиппеля вероятность обращения в 0 не превышает $\dfrac2N$. Тогда $\dfrac2N < p \Ra N > \dfrac2p$
\end{solution}

\begin{tasknum}{7}
	Докажите, что 2-$CNF$~---~задача из $\mathcal{P}$. В какой из вероятностных классов, определенных выше, попадает язык 2-$CNF$?
\end{tasknum}

\begin{solution}
	Во 2м номере домашнего задания на шестую неделю было доказано, что 2-$CNF \in \mathcal{NL}$. Но, как мы знаем, $\mathcal{NL} \subseteq \P$, так что 2-$CNF \in \P$.
	
	Если 2-КНФ выполнима, то МТ останавливается за полиномиальное от длины входа время. Пусть матожидание времени работы есть $t(|x|)$. Аналогично задаче 4, будем останавливать алгоритм и выдавать, что $x \notin$ 2-$CNF$ через $8 \cdot t(|x|)$ тактов. Тогда, если $x \in$ 2-$CNF$, то вероятность ошибки не превышает $\dfrac{1}{2}$, а если $x \notin$ 2-$CNF$, то мы не ошибаемся вовсе.
	
	Таким образом, 2-$CNF \in \mathcal{RP} \subseteq \mathcal{BPP}$.
\end{solution}

\end{document}