\documentclass[a4paper,12pt]{article}

%%% Работа с русским языком
\usepackage[cm]{fullpage}
\usepackage[T2A]{fontenc}
\usepackage[utf8]{inputenc}
\usepackage[english,russian]{babel}
\usepackage{footmisc}
\usepackage[document]{ragged2e}
\usepackage{amsmath,amsfonts,amssymb,mathtools}
\usepackage{framed}
\usepackage{pstricks}
%\usepackage[framed]{ntheorem}
\usepackage{tikz}
\usetikzlibrary{arrows,automata}
\usepackage{cmap}					% поиск в PDF
\usepackage{mathtext} 				% русские буквы в формулах
\usepackage{indentfirst}
\frenchspacing

\newcommand{\vyp}{\hookrightarrow}
\renewcommand{\epsilon}{\varepsilon}
\renewcommand{\phi}{\varphi}
\renewcommand{\kappa}{\varkappa}
\renewcommand{\le}{\leqslant}
\renewcommand{\leq}{\leqslant}
\renewcommand{\ge}{\geqslant}
\renewcommand{\geq}{\geqslant}
\renewcommand{\emptyset}{\varnothing}

%%% Дополнительная работа с математикой
\usepackage{amsmath,amsfonts,amssymb,amsthm,mathtools} % AMS
\usepackage{icomma} % "Умная" запятая: $0,2$ --- число, $0, 2$ --- перечисление

%% Номера формул
%\mathtoolsset{showonlyrefs=true} % Показывать номера только у тех формул, на которые есть \eqref{} в тексте.
%\usepackage{leqno} % Нумереация формул слева

%% Свои команды
\DeclareMathOperator{\sgn}{\mathop{sgn}}

%% Перенос знаков в формулах (по Львовскому)
\newcommand*{\hm}[1]{#1\nobreak\discretionary{}
{\hbox{$\mathsurround=0pt #1$}}{}}



%%% Работа с картинками
\usepackage{graphicx}  % Для вставки рисунков
\graphicspath{{images/}{images2/}}  % папки с картинками
\setlength\fboxsep{3pt} % Отступ рамки \fbox{} от рисунка
\setlength\fboxrule{1pt} % Толщина линий рамки \fbox{}
\usepackage{wrapfig} % Обтекание рисунков текстом

%%% Работа с таблицами
\usepackage{array,tabularx,tabulary,booktabs} % Дополнительная работа с таблицами
\usepackage{longtable}  % Длинные таблицы
\usepackage{multirow} % Слияние строк в таблице

%%% Теоремы
\theoremstyle{plain} % Это стиль по умолчанию, его можно не переопределять.
\newtheorem{theorem}{Теорема}[section]
\newtheorem{proposition}[theorem]{Утверждение}
 
\theoremstyle{definition} % "Определение"
\newtheorem{corollary}{Следствие}[theorem]
\newtheorem{problem}{Задача}[section]
 
\theoremstyle{remark} % "Примечание"
\newtheorem*{nonum}{Решение}

%%% Программирование
\usepackage{etoolbox} % логические операторы

%%% Страница
\usepackage{extsizes} % Возможность сделать 14-й шрифт
\usepackage{geometry} % Простой способ задавать поля
\geometry{top=20mm}
\geometry{bottom=20mm}
\geometry{left=10mm}
\geometry{right=20mm}
 %
\usepackage{fancyhdr} % Колонтитулы
 	\pagestyle{fancy}
 	\renewcommand{\headrulewidth}{0pt}  % Толщина линейки, отчеркивающей верхний колонтитул
\fancypagestyle{firstpage}{
	\rhead{\large{Рябых Владислав, Б05-905}}
}
% 	\lfoot{Нижний левый}
% 	\rfoot{Нижний правый}
% 	\rhead{Верхний правый]}
% 	\chead{Верхний в центре}
% 	\lhead{Верхний левый}
%	\cfoot{Нижний в центре} % По умолчанию здесь номер страницы

\usepackage{setspace} % Интерлиньяж
\onehalfspacing % Интерлиньяж 1.5
%\doublespacing % Интерлиньяж 2
%\singlespacing % Интерлиньяж 1

\usepackage{lastpage} % Узнать, сколько всего страниц в документе.

\usepackage{soul} % Модификаторы начертания

%\usepackage{hyperref}
%\usepackage[usenames,dvipsnames,svgnames,table,rgb]{xcolor}
%\hypersetup{				% Гиперссылки
%    unicode=true,           % русские буквы в раздела PDF
%    pdftitle={Заголовок},   % Заголовок
%    pdfauthor={Автор},      % Автор
%    pdfsubject={Тема},      % Тема
%    pdfcreator={Создатель}, % Создатель
%    pdfproducer={Производитель}, % Производитель
%    pdfkeywords={keyword1} {key2} {key3}, % Ключевые слова
%    colorlinks=true,       	% false: ссылки в рамках; true: цветные ссылки
%    linkcolor=red,          % внутренние ссылки
%    citecolor=black,        % на библиографию
%    filecolor=magenta,      % на файлы
%    urlcolor=cyan           % на URL
%}

\usepackage{csquotes} % Еще инструменты для ссылок

%\usepackage[style=authoryear,maxcitenames=2,backend=biber,sorting=nty]{biblatex}

\usepackage{multicol} % Несколько колонок

\usepackage{pgfplots}
\usepackage{pgfplotstable}
\newcommand{\tbf}{\textbf}


\usepackage[shortlabels]{enumitem}

\newtheorem{task}{\textbf{Задача}}

\newtheorem{innercustomthm}{\textbf{Задача}}
\newenvironment{tasknum}[1]
{\renewcommand\theinnercustomthm{#1}\innercustomthm}
{\endinnercustomthm}

\newtheorem*{solution}{\textbf{Решение}}
\newcommand{\Ra}{\Rightarrow}
\newcommand{\La}{\Leftarrow}
\newcommand{\ra}{\rightarrow}
\newcommand{\LRa}{\Leftrightarrow}
\newcommand{\n}{\mathbb}
\newcommand{\Le}{\leqslant}
\newcommand{\Ge}{\geqslant}


\renewcommand{\inf}{\infty}
\newcommand{\ol}{\overline}

\newcommand{\bigline}{\noindent\makebox[\linewidth]{\rule{\paperwidth}{0.4pt}}}

\usetikzlibrary{fit}
\newcommand{\dost}{\overset{*}{\vdash}}
\newcommand{\vyv}{\overset{*}{\Rightarrow}}


\usepackage{capt-of}
\usepackage{tikz-qtree}
\usepackage{systeme}

\newcommand{\polysv}{\leq_p}
\def\coNP{{\mathbf{\text{\textbf{co--}}\mathcal{NP}}}}
\newcommand{\NP}{\mathcal{NP}}
\renewcommand{\P}{\mathcal{P}}

\usepackage{algorithm}
%\usepackage{algpseudocode}
\usepackage[noend]{algpseudocode}

\newcommand{\prob}[1]{\mathbb{P}\left\{#1\right\}}
\newcommand{\expected}[1]{\mathbb{E}\left\{#1\right\}}

\DeclareMathOperator{\circul}{circ}

\begin{document}
	
	\thispagestyle{firstpage}
	
	\begin{center}
		\textbf{\Large{Алгоритмы и модели вычислений. \\ Домашнее задание № 10}}
	\end{center}
	
\begin{task}
Докажите формулу обращения: $(M_n(\omega))^{-1} = \dfrac{1}{n}M_n(\omega^{-1})$. Вычислите также матрицу $(M_n(\omega))^4$.
\end{task}

\begin{solution}
	Домножим обе части слева на  $M_n(\omega)$, теперь имеем равенство \[E = \dfrac{1}{n} M_n(\omega) M_n(\omega^{-1}) \] Нам надо доказать, что справа действительно получается единичная матрица. Для этого раскроем по определению произведения матриц, какой элемент стоит на $i$-й строчке в $j$-м столбце: \[a_{ij} = \dfrac1n \cdot \displaystyle \sum_{s = 0}^{n-1} (w_{n}^i)^s (w_{n}^{s})^{-j} = \dfrac1n \cdot \displaystyle \sum_{s = 0}^{n-1} (w_{n}^{i-j})^s \]
	
	\begin{itemize}
		\item $i \ne j$: $a_{ij} = \dfrac1n \cdot \displaystyle \sum_{s = 0}^{n-1} (w_{n}^{i-j})^s = \dfrac{(w_{n}^{i-j})^n - 1}{w_{n}^{i-j} - 1} = 0$, так как $w$~---~корень из единицы
		\item $i = j$: $a_{ij} = \dfrac1n \cdot \displaystyle \sum_{s = 0}^{n-1} \exp\left(\frac{2\pi s (i-j)}{n}\right) = \dfrac1n \cdot \displaystyle \sum_{s = 0}^{n-1} \exp(0) = \dfrac1n \cdot \displaystyle \sum_{s = 0}^{n-1} 1 = 1$
	\end{itemize}

	Таким образом матрица $A = (a_{ij}) = \dfrac{1}{n} M_n(\omega) M_n(\omega^{-1})$ действительно является единичной, что доказывает формулу обращения $(M_n(\omega))^{-1} = \dfrac{1}{n}M_n(\omega^{-1})$.


Вычислим теперь матрицу $(M_n(\omega))^4$. Так как $(M_n(\omega))^4 = (M_n(\omega))^2 \cdot (M_n(\omega))^2$, то вычислим сначала $(M_n(\omega))^2$, для этого раскроем по определению произведения матриц, какой элемент стоит на $i$-й строчке в $j$-м столбце: \[a_{ij} = \displaystyle \sum_{s = 0}^{n-1} (w_{n}^i)^s (w_{n}^{s})^{j} = \displaystyle \sum_{s = 0}^{n-1} (w_{n}^{i+j})^s\]

Аналогично предыдущему пункту:

\begin{equation*}
a_{ij} =
\left\{
\begin{array}{lr}
0, & (i+j) \text{ не кратно } n\\
n, & (i+j) \ \ \ \ \text{ кратно } n
\end{array}
\right.
\end{equation*}

Таким образом, матрица $(M_n(\omega))^2 = (a_{ij})$ имеет вид

\begin{equation*}
(M_n(\omega))^2 = \left(
\begin{array}{ccccc}
n & 0 & \ldots & 0\\
0 & 0 & \ldots & n\\
\vdots & \vdots & \ddots & \vdots\\
0 & 0 & \ldots & 0 \\
0 & n & \ldots & 0
\end{array}
\right)
\end{equation*}

И теперь легко видеть, что $(M_n(\omega))^4 = (M_n(\omega))^2 \cdot (M_n(\omega))^2 = n^2 \cdot E$

\end{solution}

\begin{task}
	Найдите произведение многочленов $A(x) = x^3+3x+2$ и $B(x)=3x^3+3x^2+2$ с помощью алгоритма быстрого преобразования Фурье. Для этого найдите рекурсивно дискретное преобразование Фурье двух массивов $A=(0,0,0,0,1,0,3,2)$ и $B=(0,0,0,0,3,3,0,2)$, затем вычислите ДПФ массива $C$ и восстановите коэффициенты многочлена-произведения, используя обратное преобразование.
\end{task}

\begin{solution}
	$reversed(A) = (a_0, a_1, a_2, a_3, a_4, a_5, a_6, a_7) = (2,3,0,1,0,0,0,0) \ra (2,0,0,0); (3,1,0,0) \ra (2,0), (0, 0), (3,0), (1,0) = (a_0, a_4), (a_2, a_6), (a_1, a_5), (a_3, a_7)$
	
	На каждом шаге мы делим массив на 2 части, помещая в один все элементы, стоящие на нечётных местах, а в другой~---~все, стоящие на чётных.
	
	Умножим каждый из получившихся массивов (рассматриваем их как столбцы) на матрицу $M_2$.
	
	\begin{equation*}
	M_2 = \left(
	\begin{array}{cc}
	1 & 1 \\
	1 & -1
	\end{array}
	\right)
	\end{equation*}
	
	Получим:
	
	\begin{align*}
	M_2 \cdot (a_0,a_4)^T = (2,2) \\
	M_2 \cdot (a_2,a_6)^T = (0,0) \\
	M_2 \cdot (a_1,a_5)^T = (3,3) \\
	M_2 \cdot (a_3,a_7)^T = (1,1) \\
	\end{align*}
	
	По рекурсии:
	
	\begin{gather*}
	M_4 \cdot (a_0,a_2,a_4,a_6)^T =  \left(
	\begin{array}{c}
	M_2 \cdot (a_0,a_4)^T \\
	M_2 \cdot (a_0,a_4)^T
	\end{array}
	\right) + \left(
	\begin{array}{c}
	D \cdot M_2 \cdot (a_2,a_6)^T \\
	-D \cdot M_2 \cdot (a_2,a_6)^T
	\end{array}
	\right) = \left(
	\begin{array}{c}
	2 \\
	2 \\
	2 \\
	2
	\end{array}
	\right) + \left(
	\begin{array}{c}
	0 \\
	0 \\
	0 \\
	0
	\end{array}
	\right) = \left(
	\begin{array}{c}
	2 \\
	2 \\
	2 \\
	2
	\end{array}
	\right)\\
	M_4 \cdot (a_1,a_3,a_5,a_7)^T =  \left(
	\begin{array}{c}
	M_2 \cdot (a_1,a_5)^T \\
	M_2 \cdot (a_1,a_5)^T
	\end{array}
	\right) + \left(
	\begin{array}{c}
	D \cdot M_2 \cdot (a_3,a_7)^T \\
	-D \cdot M_2 \cdot (a_3,a_7)^T
	\end{array}
	\right) = \left(
	\begin{array}{c}
	3 \\
	3 \\
	3 \\
	3
	\end{array}
	\right) + \left(
	\begin{array}{c}
	1 \\
	-i \\
	-1 \\
	-i
	\end{array}
	\right) = \left(
	\begin{array}{c}
	4 \\
	3+i \\
	2 \\
	3-i
	\end{array}
	\right)\\
	\end{gather*}
	
	Итого получаем
	
	\begin{equation*}
	A = M_8 \cdot \left(
	\begin{array}{c}
	a_0 \\
	a_1 \\
	a_2 \\
	a_3 \\
	a_4 \\
	a_5 \\
	a_6 \\
	a_7
	\end{array}
	\right) = \left(
	\begin{array}{c}
	2 \\
	2 \\
	2 \\
	2 \\
	2 \\
	2 \\
	2 \\
	2
	\end{array}
	\right) + \left(
	\begin{array}{c}
	4 \\
	2\sqrt{2}i + \sqrt{2} \\
	2i \\
	2\sqrt{2}i - \sqrt{2} \\
	-4 \\
	-2\sqrt{2}i - \sqrt{2} \\
	-2i \\
	-2\sqrt{2}i + \sqrt{2}
	\end{array}
	\right) = \left(
	\begin{array}{c}
	6 \\
	2\sqrt{2}i + 2 + \sqrt{2} \\
	2i + 2 \\
	2\sqrt{2}i + 2 - \sqrt{2} \\
	-2 \\
	-2\sqrt{2}i + 2 - \sqrt{2} \\
	-2i + 2 \\
	-2\sqrt{2}i + 2 + \sqrt{2}
	\end{array}
	\right)
	\end{equation*}
	
	Для многочлена $B$ аналогично получаем:
	
	\begin{equation*}
	B = M_8 \cdot \left(
	\begin{array}{c}
	b_0 \\
	b_1 \\
	b_2 \\
	b_3 \\
	b_4 \\
	b_5 \\
	b_6 \\
	b_7
	\end{array}
	\right) = \left(
	\begin{array}{c}
	8 \\
	\left(2 - \frac{3\sqrt{2}}{2}\right) + \left(3 + \frac{3\sqrt{2}}{2}\right)i \\
	-1 - 3i \\
	\left(2 + \frac{3\sqrt{2}}{2}\right) + \left(-3 + \frac{3\sqrt{2}}{2}\right)i \\
	2 \\
	\left(2 + \frac{3\sqrt{2}}{2}\right) + \left(3 - \frac{3\sqrt{2}}{2}\right)i \\
	-1 + 3i \\
	\left(2 - \frac{3\sqrt{2}}{2}\right) + \left(-3 - \frac{3\sqrt{2}}{2}\right)i
	\end{array}
	\right)
	\end{equation*}
	
	Умножим поэлементно $A$ на $B$:
	\begin{equation*}
	C = A^T \underset{\text{элем}}{\cdot} B = \left(
	\begin{array}{c}
	48 \\
	(-5 - 7\sqrt{2}) + (3+10\sqrt{2})i \\
	4-8i \\
	(-5 + 7\sqrt{2}) + (-3+10\sqrt{2})i \\
	-4 \\
	(-5 + 7\sqrt{2}) + (3-10\sqrt{2})i \\
	4+8i \\
	(-5 - 7\sqrt{2}) + (-3-10\sqrt{2})i
	\end{array}
	\right)
	\end{equation*}
	
	Так как $M_n^{-1}(w) = M_n(w^{-1})$, то посчитаем произведение $M_8^{-1} \cdot C$, итого получим
	
	\begin{equation*}
	\left(
	\begin{array}{c}
	c_0 \\
	c_1 \\
	c_2 \\
	c_3 \\
	c_4 \\
	c_5 \\
	c_6 \\
	c_7
	\end{array}
	\right) =  \left(
	\begin{array}{c}
	4 \\
	6 \\
	6 \\
	17 \\
	9 \\
	3 \\
	3 \\
	0
	\end{array}
	\right)
	\end{equation*}
	
\end{solution}

\begin{task}
	Даны числа $x_1,\dotsc, x_n$. Доказать, что коэффициенты многочлена  $f(x) = \displaystyle\prod_{i = 1}^n (x-x_i)$, можно найти за $O(n \log_2^2 n)$ арифметических операций.
\end{task}

\begin{solution}
	На первом уровне проводим $\frac{n}{2}$ перемножений: перемножаются многочлены $(x-x_1)$ c $(x-x_2)$; $(x-x_3)$ с $(x-x_4); \ldots$. Затем также перемножаем полученные многочлены между собой, пока не останется только один многочлен.
	
	Всего мы получим $\log_2 n$ уровней, также необходимо заметить, что степень многочленов возрастает вдвое на каждом уровне (а многочлены степени $k$ перемножаются за $O(k\log k)$ с помощью преобразования Фурье). Таким образом, итоговая сложность \[T(n) \le C \cdot \displaystyle \sum_{i = 1}^{\log_2 n} \dfrac{n}{2} i = C \cdot \dfrac{n}{2} \cdot \dfrac{\log_2 n \cdot (\log_2 n + 1)}{2} \Ra T(n) = O(n \log_2^2 n)\]
\end{solution}

\begin{task}
	Используя  ДПФ, найдите решение системы линейных уравнений $Cx=b$. где $C$~--- это циркулянтная матрица, порожденная вектором столбцом $(1,2,4,8)^T$, а $b^T=(16,8,4,2)$.
\end{task}

\begin{solution}
	Циркулянтная матрица $C$ есть
	
	\begin{equation*}
	C = \left(
	\begin{array}{cccc}
	1 & 8 & 4 & 2\\
	2 & 1 & 8 & 4\\
	4 & 2 & 1 & 8\\
	8 & 4 & 2 & 1
	\end{array}
	\right)
	\end{equation*}
	
	Матрица $C$ имеет размерность $4\times4$, так что в $w = i$. Обозначим $\lambda_j$~---~собственные значения $C$, тогда $\lambda_i = \displaystyle \sum_{k = 0}^{n-1} c_{n-k} \cdot (i)^{jk}$ (считаем, что $c_0 = c_n$; формула из задачи 6). Тогда получаем
	
	\begin{gather*}
	\lambda_0 = 1 + 8 + 4 + 2 =15\\
	\lambda_1=1+8i-4-2i=-3+6i\\
	\lambda_2=1-8+4-2=-5\\
	\lambda_3=1-8i-4+2i=-3-6i\\
	\end{gather*}
	
	Таким образом матрицы $\Lambda$ и $\Lambda^{-1}$ имеют вид:
	
	
	\begin{equation*}
	\Lambda = \left(
	\begin{array}{cccc}
	15 & 0 & 0 & 0\\
	0 & -3+6i & 0 & 0\\
	0 & 0 & -5 & 0\\
	0 & 0 & 0 & -3-6i
	\end{array}
	\right) \Ra
	\Lambda^{-1} = \left(
	\begin{array}{cccc}
	\dfrac{1}{15} & 0 & 0 & 0\\
	0 & \dfrac{1}{-3+6i} & 0 & 0\\
	0 & 0 & -\dfrac{1}{5} & 0\\
	0 & 0 & 0 & -\dfrac{1}{3+6i}
	\end{array}
	\right)
	\end{equation*}
	
	В нашей задаче матрица Фурье принимает вид:
	
	\begin{equation*}
	F = \left(
	\begin{array}{cccc}
	1 & 1 & 1 & 1\\
	1 & i & -1 & -i\\
	1 & -1 & 1 & -1\\
	1 & -i & -1 & i
	\end{array}
	\right) \Ra
	F^{*} = \left(
	\begin{array}{cccc}
	1 & 1 & 1 & 1\\
	1 & -i & -1 & i\\
	1 & -1 & 1 & -1\\
	1 & i & -1 & -i
	\end{array}
	\right)
	\end{equation*}

$Cx = b \Ra x = C^{-1}b; C = F\Lambda F^{-1} \Ra C^{-1} = F \Lambda^{-1} F^{-1} = \dfrac{1}{4}F\Lambda^{-1}F^{*} \Ra x = \dfrac14 \cdot F\Lambda^{-1}F^{*} \cdot b$

\begin{equation*}
x = \dfrac{1}{4} \cdot \left(
\begin{array}{cccc}
1 & 1 & 1 & 1\\
1 & i & -1 & -i\\
1 & -1 & 1 & -1\\
1 & -i & -1 & i
\end{array}
\right) \cdot \left(
\begin{array}{cccc}
\dfrac{1}{15} & 0 & 0 & 0\\
0 & \dfrac{1}{-3+6i} & 0 & 0\\
0 & 0 & -\dfrac{1}{5} & 0\\
0 & 0 & 0 & -\dfrac{1}{3+6i}
\end{array}
\right) \cdot \left(
\begin{array}{cccc}
1 & 1 & 1 & 1\\
1 & -i & -1 & i\\
1 & -1 & 1 & -1\\
1 & i & -1 & -i
\end{array}
\right) \cdot \left(
\begin{array}{c}
16 \\
8 \\
4 \\
2
\end{array}
\right) = \left(
\begin{array}{c}
-0.8 \\
1.6 \\
0.8 \\
0.4
\end{array}
\right)
\end{equation*}
	
\end{solution}

\begin{task}
	Обозначим для вектора $\vec{x}$ циркулянтную матрицу с первым столбцом $\vec{x}$ за $\circul(\vec{x})$. Назовём циклической свёрткой $\vec{x}*\vec{y}$ двух векторов произведение матрицы на вектор $\circul(\vec{x})\vec{y}$. Докажите, что $FFT(\vec{x}*\vec{y})$ есть произведение векторов $FFT(\vec{x})$ и $FFT(\vec{y})$ по Адамару (т.~е. поэлементное: $i$-ая компонента вектора-произведения есть произведение $i$-ых компонент сомножителей).
\end{task}

\begin{solution}
	\begin{equation*}
	X = \circul(\vec{x}) = \left(
	\begin{array}{ccccc}
	x_0 & x_{n-1} & x_{n-2} & \ldots & x_1\\
	x_1 & x_0 & x_{n-1} & \ldots & x_2\\
	x_2 & x_1 & x_0 & \ldots & x_3\\
	\vdots & \vdots & \ddots & \vdots\\
	x_{n-1} & x_{n-2} & x_{n-3} & \ldots & x_0
	\end{array}
	\right)
	\end{equation*}
	
Обозначим $A = \circul(\vec{x})\vec{y}$.

Тогда $a_j$ можно записать в виде $a_j = \displaystyle \sum_{m=0}^{j} x_m y_{j-m} + \displaystyle \sum_{m = j+1}^{n-1} x_m y_{n+j-m}$

Обозначим $B = FFT(\vec{x}*\vec{y})$.

Так как $B = FA$, запишем $b_i$ в виде $b_i = \displaystyle \sum_{k=0}^{n-1} (w^i)^k \left( \displaystyle \sum_{m=0}^{k} x_m y_{k-m} + \displaystyle \sum_{m=k+1}^{n-1} x_m y_{n+k-m} \right)$
	
Обозначим $C = FFT(\vec{x}) \underset{\text{элем}}{\cdot} FFT(\vec{y}) = (F\vec{x}) \cdot (F\vec{y})$ (умножение поэлементное).

Тогда $c_i = \displaystyle \sum_{j=0}^{n-1} (w^i)^j x_j \cdot \displaystyle \sum_{k=0}^{n-1} (w^i)^k y_k$

Так как $(w^i)^n = 1$, то коэффициент при $(w^i)^k$ имеет вид $\displaystyle \sum_{j=0}^{k} x_j y_{k-j} + \displaystyle \sum_{j=k+1}^{n-1} x_j y_{n+k-j}$
	
Коэффициенты при $(w^i)^k$ в выражении $b_i$ абсолютно такие же, из чего и следует утверждение задачи: $FFT(\vec{x}*\vec{y}) = FFT(\vec{x}) \underset{\text{элем}}{\cdot} FFT(\vec{y})$
	
\end{solution}

\begin{task}
	Рассмотрим циркулянтную матрицу порядка $n+1$, первый столбец которой равен $(c_0, c_1, \dotsc, c_n)^T$, т. е. матрицу вида
	$$\begin{bmatrix}
	c_0 & c_n & c_{n-1} & \dots & c_1 \\
	c_1 & c_0 & c_n & \dots & c_2 \\ 
	c_2 & c_1 & c_0 & \dots & c_3 \\ 
	\vdots & \vdots & \vdots & \ddots & \vdots \\ 
	c_n & c_{n-1} & c_{n-2} & \dots & c_0 \end{bmatrix}$$
	
	Докажите, что все её собственные значения, домноженные на $\frac{1}{\sqrt{n+1}}$, могут быть найдены умножением матрицы Фурье $F_n = \frac{1}{\sqrt{n+1}}\left(\omega_n^{ij}\right)_{i, j = 0}^{n}$ размеров $(n+1)\times (n+1)$, где $\omega_n~=~e^{\frac{2\pi i}{n}}$~--- корень из единицы, на вектор $(c_0, c_n, c_{n-1},\dotsc, c_1)^T$. Найдите с помощью алгоритма БПФ собственные значения циркулянтной матрицы, первый столбец которой имеет вид $(1, 2, 4, 6)^T$.
\end{task}

\begin{solution}
	$\Lambda F_n = F_n C^T$. Посмотрим на первые столбцы обеих получающихся матриц:
	\begin{itemize}
		\item Первый столбец левой матрицы есть $\dfrac{1}{\sqrt{n+1}}(\lambda_0, \ldots, \lambda_n)^T$
		\item Первый столбец правой матрицы есть $F_n(c_0, c_n, c_{n-1}, \ldots, c_1)^T$
	\end{itemize}
	
	Таким образом \begin{equation*}
	\left(
	\begin{array}{c}
	\lambda_1 \\
	\lambda_2 \\
	\vdots \\
	\lambda_n
	\end{array}
	\right) = \sqrt{n+1} \cdot F_n \cdot \left(
	\begin{array}{c}
	c_1 \\
	c_2 \\
	\vdots \\
	c_n
	\end{array}
	\right)
	\end{equation*}
	
	Как в задаче номер 4, считаем собственные значения циркулянтной матрицы:
	
\begin{gather*}
\lambda_0 = 1 + 6 + 4 + 2 =13\\
\lambda_1=1+2i-4-6i=-3-4i\\
\lambda_2=1-2+4-6=-3\\
\lambda_3=1-2i-4+6i=-3+4i\\
\end{gather*}
	
\end{solution}

\begin{task}
	Дано множество различных чисел $A\subseteq \{1,\dots,m\}$. Рассмотрим множество $A+A$, образованное суммами элементов $A$. 
	Докажите или опровергните существование процедур построения
	$A+A$, имеющих субквадратичную трудоемкость $o(m^2)$.
\end{task}

\begin{solution}
	Заведём массив $B$ из $m$ элементов, определённых следующим образом \begin{equation*}
	b_{i} =
	\left\{
	\begin{array}{lr}
	0, & i \notin A\\
	1, & i \in A
	\end{array}
	\right.
	\end{equation*}
	Определим $f(x) = \displaystyle \sum_{i=1}^{m} b_i x^i$, возведём массив во вторую степень за $O(m \log m)$ с помощью преобразования Фурье, получим $f^2(x) = \displaystyle \sum_{k=1}^{m} \displaystyle \sum_{l=1}^{m} b_k b_l x^{k+l}$
	
	Теперь можно сказать, что $b_k b_l \neq 0 \LRa k, l \in A \Ra k+l \in A + A$.
	
	Сложность алгоритма есть $O(m \log m)$, так как всё, помимо возведения многочлена $f$ в квадрат, выполняется за не более, чем линейное время. А так как алгоритм имеет сложность $O(m \log m)$, то также можно сказать, что он имеет сложность $o(m^2)$
\end{solution}

\end{document}