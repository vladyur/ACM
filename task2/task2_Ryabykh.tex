\documentclass[a4paper,12pt]{article}

%%% Работа с русским языком
\usepackage[cm]{fullpage}
\usepackage[T2A]{fontenc}
\usepackage[utf8]{inputenc}
\usepackage[english,russian]{babel}
\usepackage{footmisc}
\usepackage[document]{ragged2e}
\usepackage{amsmath,amsfonts,amssymb,mathtools}
\usepackage{framed}
\usepackage{pstricks}
%\usepackage[framed]{ntheorem}
\usepackage{tikz}
\usetikzlibrary{arrows,automata}
\usepackage{cmap}					% поиск в PDF
\usepackage{mathtext} 				% русские буквы в формулах
\usepackage{indentfirst}
\frenchspacing

\newcommand{\vyp}{\hookrightarrow}
\renewcommand{\epsilon}{\varepsilon}
\renewcommand{\phi}{\varphi}
\renewcommand{\kappa}{\varkappa}
\renewcommand{\le}{\leqslant}
\renewcommand{\leq}{\leqslant}
\renewcommand{\ge}{\geqslant}
\renewcommand{\geq}{\geqslant}
\renewcommand{\emptyset}{\varnothing}

%%% Дополнительная работа с математикой
\usepackage{amsmath,amsfonts,amssymb,amsthm,mathtools} % AMS
\usepackage{icomma} % "Умная" запятая: $0,2$ --- число, $0, 2$ --- перечисление

%% Номера формул
%\mathtoolsset{showonlyrefs=true} % Показывать номера только у тех формул, на которые есть \eqref{} в тексте.
%\usepackage{leqno} % Нумереация формул слева

%% Свои команды
\DeclareMathOperator{\sgn}{\mathop{sgn}}

%% Перенос знаков в формулах (по Львовскому)
\newcommand*{\hm}[1]{#1\nobreak\discretionary{}
{\hbox{$\mathsurround=0pt #1$}}{}}



%%% Работа с картинками
\usepackage{graphicx}  % Для вставки рисунков
\graphicspath{{images/}{images2/}}  % папки с картинками
\setlength\fboxsep{3pt} % Отступ рамки \fbox{} от рисунка
\setlength\fboxrule{1pt} % Толщина линий рамки \fbox{}
\usepackage{wrapfig} % Обтекание рисунков текстом

%%% Работа с таблицами
\usepackage{array,tabularx,tabulary,booktabs} % Дополнительная работа с таблицами
\usepackage{longtable}  % Длинные таблицы
\usepackage{multirow} % Слияние строк в таблице

%%% Теоремы
\theoremstyle{plain} % Это стиль по умолчанию, его можно не переопределять.
\newtheorem{theorem}{Теорема}[section]
\newtheorem{proposition}[theorem]{Утверждение}
 
\theoremstyle{definition} % "Определение"
\newtheorem{corollary}{Следствие}[theorem]
\newtheorem{problem}{Задача}[section]
 
\theoremstyle{remark} % "Примечание"
\newtheorem*{nonum}{Решение}

%%% Программирование
\usepackage{etoolbox} % логические операторы

%%% Страница
\usepackage{extsizes} % Возможность сделать 14-й шрифт
\usepackage{geometry} % Простой способ задавать поля
\geometry{top=20mm}
\geometry{bottom=20mm}
\geometry{left=10mm}
\geometry{right=20mm}
 %
\usepackage{fancyhdr} % Колонтитулы
 	\pagestyle{fancy}
 	\renewcommand{\headrulewidth}{0pt}  % Толщина линейки, отчеркивающей верхний колонтитул
\fancypagestyle{firstpage}{
	\rhead{\large{Рябых Владислав, Б05-905}}
}
% 	\lfoot{Нижний левый}
% 	\rfoot{Нижний правый}
% 	\rhead{Верхний правый]}
% 	\chead{Верхний в центре}
% 	\lhead{Верхний левый}
%	\cfoot{Нижний в центре} % По умолчанию здесь номер страницы

\usepackage{setspace} % Интерлиньяж
\onehalfspacing % Интерлиньяж 1.5
%\doublespacing % Интерлиньяж 2
%\singlespacing % Интерлиньяж 1

\usepackage{lastpage} % Узнать, сколько всего страниц в документе.

\usepackage{soul} % Модификаторы начертания

\usepackage{hyperref}
%\usepackage[usenames,dvipsnames,svgnames,table,rgb]{xcolor}
\hypersetup{				% Гиперссылки
    unicode=true,           % русские буквы в раздела PDF
    pdftitle={Заголовок},   % Заголовок
    pdfauthor={Автор},      % Автор
    pdfsubject={Тема},      % Тема
    pdfcreator={Создатель}, % Создатель
    pdfproducer={Производитель}, % Производитель
    pdfkeywords={keyword1} {key2} {key3}, % Ключевые слова
    colorlinks=true,       	% false: ссылки в рамках; true: цветные ссылки
    linkcolor=red,          % внутренние ссылки
    citecolor=black,        % на библиографию
    filecolor=magenta,      % на файлы
    urlcolor=blue           % на URL
}

\usepackage{csquotes} % Еще инструменты для ссылок

%\usepackage[style=authoryear,maxcitenames=2,backend=biber,sorting=nty]{biblatex}

\usepackage{multicol} % Несколько колонок

\usepackage{pgfplots}
\usepackage{pgfplotstable}
\newcommand{\tbf}{\textbf}


\usepackage[shortlabels]{enumitem}

\newtheorem{task}{\textbf{Задача}}

\newtheorem{innercustomthm}{\textbf{Задача}}
\newenvironment{tasknum}[1]
{\renewcommand\theinnercustomthm{#1}\innercustomthm}
{\endinnercustomthm}

\newtheorem*{solution}{\textbf{Решение}}
\newcommand{\Ra}{\Rightarrow}
\newcommand{\La}{\Leftarrow}
\newcommand{\ra}{\rightarrow}
\newcommand{\LRa}{\Leftrightarrow}
\newcommand{\n}{\mathbb}
\newcommand{\Le}{\leqslant}
\newcommand{\Ge}{\geqslant}


\renewcommand{\inf}{\infty}
\newcommand{\ol}{\overline}

\newcommand{\bigline}{\noindent\makebox[\linewidth]{\rule{\paperwidth}{0.4pt}}}

\usetikzlibrary{fit}
\newcommand{\dost}{\overset{*}{\vdash}}
\newcommand{\vyv}{\overset{*}{\Rightarrow}}


\usepackage{capt-of}
\usepackage{tikz-qtree}
\usepackage{systeme}

\newcommand{\polysv}{\leq_p}
\def\coNP{{\mathbf{\text{\textbf{co--}}\mathcal{NP}}}}
\newcommand{\NP}{\mathcal{NP}}
\renewcommand{\P}{\mathcal{P}}

\usepackage{algorithm}
%\usepackage{algpseudocode}
\usepackage[noend]{algpseudocode}

\newcommand{\prob}[1]{\mathbb{P}\left\{#1\right\}}
\newcommand{\expected}[1]{\mathbb{E}\left\{#1\right\}}

\begin{document}
	
	\thispagestyle{firstpage}
	
	\begin{center}
		\textbf{\Large{Алгоритмы и модели вычислений. \\ Домашнее задание № 2}}
	\end{center}


\begin{tasknum}{1}
	Докажите следующие свойства полиномиальной сводимости:
	
	($i$) Рефлексивность: $A\leq_p A$; транзитивность: если $A\leq_p B$ и $B\leq_p C$, то $A\leq_p C$;
	
	($ii$) Если $B\in\mathcal{P}$ и $A\leq_p B$, то $A\in\mathcal{P}$;
	
	($iii$) Если $B\in\mathcal{NP}$ и $A\leq_p B$, то $A\in\mathcal{NP}$.
\end{tasknum}

\begin{solution}
	Для начала запишем определение полиномиальной сводимости: $A\leq_p B \LRa \exists f$ ~---~ функция, вычислимая за полином от длины входа, такая что $x \in A \LRa f(x) \in B$
	\begin{enumerate}
		\item \begin{itemize}
			\item Рефлексивность очевидна: в качестве $f$ берём тождественную функцию $id$, тогда $x \in A \LRa id(x) = x \in A \Ra A \leq_p A$
			\item Транзитивность: если $A\leq_p B$ и $B\leq_p C$, то $\exists f, g$, вычислимые за полином, такие что $x \in A \LRa f(x) \in B \LRa g(f(x)) \in C$. Заметим, что функция $g(f(x))$ вычисляется за $poly(|f(x)|) = poly(poly(|x|)) = poly(|x|)$, так как полином от полинома также является полиномом. Таким образом $x \in A \LRa g(f(x)) \in C \Ra A \polysv C$
		\end{itemize}
		\item \begin{itemize}
			\item $A \polysv B \Ra \exists f$ --- функция, вычислимая за полином, такая что $x \in A \LRa f(x) \in B$; таким образом существует детерминированная МТ $M_1$, вычисляющая функцию $f(x)$ за $poly(|x|)$
			\item $B \in \P \Ra \exists M_2$ --- детерминированная МТ, распознающая язык $B$ за полином.
		\end{itemize}
		
		Построим теперь детерминированную МТ $M$, распознающую язык $A$ за полином: на входе $x$ сначала моделируется работа $M_1$, то есть за полином от длины входа вычисляется функция $f(x)$, а затем моделируется работа $M_2$ на входе $f(x)$
		
		Таким образом детерминированная МТ $M$ распознаёт язык $A$ за $poly(|x|) + poly(|f(x)|) = poly(|x|) + poly(poly(|x|)) = poly(|x|)$, так как полином от полинома также является полиномом и сумма полиномов есть полином. Таким образом язык $A$ распознаётся за полиномиальное от длины входа время детерминированной МТ $M$, следовательно $A \in \P$
		
		\item Доказательство аналогично предыдущему пункту с точностью до замены детерминированной МТ на недетерминированную МТ, приведём его для полноты:
		
		\begin{itemize}
			\item $A \polysv B \Ra \exists f$ --- функция, вычислимая за полином, такая что $x \in A \LRa f(x) \in B$; таким образом существует детерминированная МТ $M_1$, вычисляющая функцию $f(x)$ за $poly(|x|)$
			\item $B \in \P \Ra \exists M_2$ --- недетерминированная МТ, распознающая язык $B$ за полином.
		\end{itemize}
		
		Построим теперь недетерминированную МТ $M$, распознающую язык $A$ за полином: на входе $x$ сначала моделируется работа $M_1$, то есть за полином от длины входа вычисляется функция $f(x)$, а затем моделируется работа $M_2$ на входе $f(x)$
		
		Таким образом недетерминированная МТ $M$ распознаёт язык $A$ за $poly(|x|) + poly(|f(x)|) = poly(|x|) + poly(poly(|x|)) = poly(|x|)$, так как полином от полинома также является полиномом и сумма полиномов есть полином. Таким образом язык $A$ распознаётся за полиномиальное от длины входа время недетерминированной МТ $M$, следовательно $A \in \NP$
		
	\end{enumerate}

\end{solution}

\begin{tasknum}{2}
	Докажите, что следующие языки принадлежат классу $\mathcal{P}$. Считайте, что графы заданы матрицами смежности.
	
	($i$) Язык двудольных графов, содержащих не менее $2018$ треугольников (троек попарно смежных вершин);
	
	($ii$) Язык несвязных графов без циклов;
	
	($iii$) Язык квадратных $\{0; 1\}$-матриц порядка $n\geq 3000$, в которых есть квадратная подматрица порядка $n-2018$, заполненная одними единицами.
\end{tasknum}

\begin{solution}

	\begin{enumerate}
		\item Выберем произвольную тройку вершин. Исходный граф двудолен, по принципу Дирихле в одной из долей окажется по крайней мере 2 вершины. Они не будут смежны в силу двудольности исходного графа и, следовательно, тройка вершин не будет образовывать треугольник. Таким образом мы доказали, что в двудольном графе в принципе не может быть треугольников, а следовательно язык из условия задачи пуст: $L = \emptyset$. А пустое множество, очевидно, лежит в $\P$: машине Тьюринга достаточно просто на любом входе $x$ выдавать ответ $0$, что даже при условии полного прочтения входа $x$ (хотя это и необязательно), работает за $poly(|x|)$
		
		\item Построим полиномиальный алгоритм: для начала нам необходимо найти все компоненты связности исходного графа --- можем сделать это с помощью поиска в глубину: сначала запускаем $DFS$ из произвольной вершины, помечаем её, поиск обходит все достижимые (также помечаем их) и находит тем самым первую компоненту связности, затем запускаем $DFS$ из произвольной не помеченной раннее вершины и т.д, пока не будут найдены все компоненты связности исходного графа (если компонента связности всего одна, то граф связен и алгоритм останавливается, выводя $0$)
		
		Теперь необходимо проверить на ацикличность каждую из компонент связности графа: с этим также справляется $DFS$ --- в каждой из компонент связности запускаем серию обходов $DFS$ следующим образом: из каждой вершины, которая ещё никак не была помечена запускается поиск в глубину, который при входе в вершину красит её в красный цвет, а при выходе в синий. Таким образом, если в какой-то момент $DFS$ пытается пройти в красную вершину (которая при этом не является предком вершины, в которой мы находимся в данный момент времени), то цикл найден и алгоритм останавливается, выводя $0$. Если же оказалось, что все вершины помечены и не было найдено ни одного цикла, то граф является ацикличным, а так как алгоритм вообще дошёл до этой точки, то граф является ещё и несвязным, то есть алгоритм выводит $1$
		
		Таким образом МТ, реализующая построенный алгоритм, распознаёт язык из условия задачи. При этом алгоритм (а следовательно и МТ) работает за полиномиальное время, так как вся его работа заключается в использовании $O\left(|V|\right)$ поисков в глубину, которые сами по себе работают за линию. Следовательно язык из условия задачи $L \in \P$
		
		\item Построим полиномиальный алгоритм: с помощью динамического программирования найдём $k$ --- порядок максимальной подматрицы, заполненный одними единицами, а потом просто сравним это число с $n-2018$
		
		Пусть $A = (a_{ij}), i \in \ol{1\ldots n}, j \in \ol{1\ldots n}$ --- матрица порядка $n$, поданная на вход. Для простоты добавим ей $pad$ (обёртку) из нулей слева и сверху (то есть теперь $A$ --- матрица порядка $n+1$, причём $a_{i,j} = 0$ при $i = 1 \vee j=1$). Теперь будем преобразовывать её элементы следующим образом: 
		
		\begin{algorithm}
			\caption{Преобразование матрицы $A$}\label{alg}
			\begin{algorithmic}
				
				\For{$i \in \ol{2\ldots n+1}$}
				\For{$j \in \ol{2\ldots n+1}$}
				\If {$a_{i,j} \neq 0$}
				\State $a_{i,j} \leftarrow \min(a_{i-1,j}, \ \ a_{i,j-1}, \ \ a_{i-1,j-1}) + 1$
				\EndIf
				\EndFor
				\EndFor
				
			\end{algorithmic}
		\end{algorithm}
	
		После завершения алгоритма в элементе $a_{ij}$ матрицы $A$ содержится число, равное порядку наибольшей подматрицы из единичек, имеющей клетку $i, j$ своим правым нижним углом.
		
		Приведём пример работы алгоритма:
		
		\begin{equation*}
		\begin{matrix}
		\\
		0 & 1 & 1  \\
		1 & 1 & 1  \\
		0 & 0 & 1  \\
		\end{matrix}
		\ \ \ \ \xrightarrow{pad} \ \ \ \
		\begin{matrix}
		0 & 0 & 0 & 0 \\
		0 & 0 & 1 & 1 \\
		0 & 1 & 1 & 1  \\
		0 & 0 & 0 & 1  \\
		\end{matrix}
		\ \ \ \ \xrightarrow{\text{alg }\ref{alg}} \ \ \ \ 
		\begin{matrix}
		0 & 0 & 0 & 0 \\
		0 & 0 & 1 & 1 \\
		0 & 1 & 1 & 2  \\
		0 & 0 & 0 & 1  \\
		\end{matrix}
		\end{equation*}
		
		Таким образом, мы находим $k$ --- порядок максимальной подматрицы, заполненный одними единицами, а потом просто сравниваем это число с порядком $(n-1) - 2018$, ($n-1$ объясняется тем, что в ходе работы алгоритма мы увеличиваем порядок матрицы на $1$).
		
		Алгоритм работает полиномиальное время, следовательно $L$ --- язык из условия задачи лежит в $\P$
		
	\end{enumerate}
	
\end{solution}

\begin{tasknum}{3}
	Корректно ли следующее  рассуждение? Язык $\mathsf{3-COLOR}$ сводится к языку $\mathsf{2-COLOR}$ следующим образом: добавим новую вершину и соединим её со всеми вершинами исходного графа. Тогда новый граф можно окрасить в $3$ цвета тогда и только тогда, когда исходный можно было окрасить в $2$ цвета.
	
	При положительном ответе приведите обоснование записанной сводимости. В противном случае~--- укажите явное место ошибки.
\end{tasknum}

\begin{solution}
	
	Автор данного рассуждения пишет, что $\mathsf{3-COLOR}$ сводится к языку $\mathsf{2-COLOR}$, но для доказательства этого преобразовывает граф $G \in \mathsf{2-COLOR}$ так, чтобы $f(G) = \tilde{G}$ лежало в $\mathsf{3-COLOR}$, то есть на самом деле доказывает не $\mathsf{3-COLOR} \polysv \mathsf{2-COLOR}$, а $\mathsf{2-COLOR} \polysv \mathsf{3-COLOR}$
	
	Доказательство же того, что $\mathsf{2-COLOR} \polysv \mathsf{3-COLOR}$ не даёт нам ничего интересного, потому что мы сводим гораздо более простую задачу ($\mathsf{2-COLOR} \in \P$) к более сложной ($\mathsf{3-COLOR} \in \NP$).
	
	Заметим, что из этой сводимости и свойства, доказанного в третьем пункте первой задачи следует, что $\mathsf{2-COLOR} \in \NP$, однако мы это и так знали в силу того, что $\P \subseteq \NP$
	
\end{solution}

\begin{tasknum}{4}
	Докажите, что классы $\mathcal{P}$ и $\mathcal{NP}$ замкнуты относительно операции $*$~---~звезды Клини (была в ТРЯПе). Для языка $NP$ приведите также и сертификат принадлежности слова из $\Sigma^*$ языку $L^*$, где $L\in\mathcal{NP}$. 
\end{tasknum}

\begin{solution}
	\begin{itemize}
		\item
	$L \in \P \LRa L$ распознаётся детерминированной МТ, работающей за полиномиальное время от длины входа. Тогда, так как полиномы замкнуты относительно элементарных операций, то языки вида $L^n$ также распознаются детерминированной МТ, работающей за полиномиальное время от длины входа (т.е $L^n \in \P$), равно как и объединение языков, распознающихся за полином, распознаётся за полиномиальное время. Тогда имеем: $L^* =\displaystyle\bigcup_{n=0}^{\inf} L^n \in \P$
	
		\item
	Доказательство аналогично с точностью до замены детерминированной МТ на недетерминированную. Приведём сертификат принадлежности $s$ для $x \in \Sigma^*$ к $L^* \in \NP$. Пусть $s = w_1\#w_2\#\ldots\#w_n$, где $w_1w_2\ldots w_n = x, w_i \in L$ --- разложение слова $x$ по словам из языка $L$. Тогда мы можем за полином от $|x|$ проверить, что действительно $x = w_1w_2\ldots w_n$, следовательно $x \in L^n \subseteq L^* \in \NP$
	
	\end{itemize}
\end{solution}

\begin{tasknum}{5}
	Покажите, что язык разложения на множители $$L_{factor} = \{(N,M)\in\mathbb{Z}^2 \mbox{ }|\mbox{ } 1<M<N\mbox{ и } N \mbox{ имеет делитель } d, 1<d\leq M\}$$ лежит в пересечении $\mathcal{NP}\cap co-\mathcal{NP}$. 
\end{tasknum}

\begin{solution}
	\begin{itemize}
		\item Для начала покажем, что язык $L_{factor}$ лежит в $\NP$. Для этого предъявим сертификат. Пусть $x = (n, m) \in \mathbb{Z}^2$. Сертификатом будет $d \in \mathbb{Z}: 1 \le d\le m$ --- делитель числа $n$. Тогда мы с помощью деления в столбик за полиномиальное время от длины входа можем действительно убедиться в том, что $d \ | \ n$ и полиномиально проверить, что $1 \le d \le m$
		
		\item Теперь, чтобы показать, что язык $L_{factor}$ лежит в $co-\NP$ необходимо для любого входа $x = (n, m) \in \mathbb{Z}^2$ такого что  $x \notin L_{factor}$ подобрать такой сертификат $s$, чтобы мы могли полиномиально доказать $x \notin L_{factor}$. В качестве такого сертификата можно взять разложение $n$ на простые множители: $s = p_1\#p_2\#\ldots \#p_k$, где все $p_i$ различны (их кратности нас не интересуют), также заметим, что $k < n$. Теперь для доказательства того, что $x \notin L_{factor}$ необходимо и достаточно показать, что $\forall i \in \ol{1\ldots k} \vyp p_i > n$ --- что делается за полиномиальное от длины входа время
	\end{itemize}
	
\end{solution}

\begin{tasknum}{6}
	Язык ГП состоит из описаний графов, имеющих гамильтонов путь. Язык ГЦ состоит из описаний графов, имеющих  гамильтонов цикл (проходящий через все вершины, причем все вершины в этом цикле, кроме первой и последней, попарно различны). Постройте явные полиномиальные сводимости ГЦ к ГП и ГП к ГЦ.
\end{tasknum}

\begin{solution}
	\begin{itemize}
		\item ГЦ $\polysv$ ГП: пусть на вход подаётся некоторый граф $G = (V, E)$, будем строить $f(G) = \tilde{G} = (\tilde{V}, \tilde{E})$. Пусть $V = (v_1, \ldots, v_n)$.
		
		$\tilde{V} = V \cup \{v_{n+1}, v_{n+2}, v_{n+3}\}$; $\tilde{E} = E \cup E_{n+1} \cup E_{n+2} \cup E_{n+3}$, где 
		
		$E_{n+1} = \{(v_{n+1}, v_i) | (v_n, v_i) \in E\}$ --- по рёбрам это копия вершины $v_n$
		
		$E_{n+2} = \{(v_{n+2}, v_n)\}$; $E_{n+3} = \{(v_{n+3}, v_{n+1})\}$, таким образом $\deg (v_{n+2}) = \deg(v_{n+3}) = 1$
		
		Заметим, что $f(G)$ работает за полином, так как она лишь дополняет вход 3 вершинами и $2 + \deg(v_n) \le n+1$ рёбрами. Докажем теперь, что $G$ содержит ГЦ $\LRa$ $\tilde{G}$ содержит ГП
		
		\begin{itemize}
			\item \Ra: $G$ содержит ГЦ, который, не умаляя общности, проходит вершины в порядке $v_1 \ra v_2 \ra \ldots \ra v_n \ra v_1$. Тогда покажем, что $\tilde{G}$ содержит ГП: начинаем путь с вершины $v_{n+2}$, затем идём в вершину $v_{n}$ (такое ребро есть по построению), затем проходим по части ГЦ исходного графа $G$: $v_n \ra v_1 \ra \ldots \ra v_{n-1}$, после чего направляем $v_{n-1}$ в вершину $v_{n+1}$ (ребро $(v_{n-1}, v_n)$ есть в исходном графе, а следовательно $(v_{n-1}, v_{n+1}) \in \tilde{E}$ по построению), после чего идём по ребру $(v_{n+1}, v_{n+3})$, которое присутствует в $\tilde{G}$ по построению и тем самым завершаем построенный ГП: $v_{n+2} \ra v_n \ra v_1 \ra \ldots \ra v_{n-1} \ra v_{n+1} \ra v_{n+3}$
			
			\item \La: $\tilde{G}$ содержит ГП. Так как вершины $\deg (v_{n+2}) = \deg(v_{n+3}) = 1$, то эти вершины обязательно являются концами этого пути, и тогда при их удалении вместе с инцидентными рёбрами в графе останется ГП, который теперь путь будет иметь своими концами вершины $v_n$ и $v_{n+1}$ (единственные, с которыми были связаны удалённые $v_{n+2}$ и $v_{n+3}$). Допустим, что вершина $v_{n+1}$ в этом ГП соединена ребром с некоторой вершиной $u$. Тогда по построению этой вершины $v_{n+1}$ мы можем утверждать, что в исходном графе $G$ есть ребро $(v_n, u)$, после чего делаем финальный шаг: удаляем вершину $v_{n+1}$ со всеми инцидентными ей рёбрами (и получаем тем самым обратно наш исходный граф $G$), после чего получаем ГП, имеющий концами $v_n$ и $u$, добавляя ребро $(v_n, u) \in E$, получаем ГЦ в исходном графе $G$
		\end{itemize}
	
		\item ГП $\polysv$ ГЦ: пусть на вход подаётся некоторый граф $G = (V, E)$, будем строить $f(G) = \tilde{G} = (\tilde{V}, \tilde{E})$ следующим образом:
		
		$\tilde{V} = V \cup \{u\}$; $\tilde{E} = E \cup \{(u, v_i) | v_i \in V\}$ --- добавляем вершину, соединённую со всеми остальными
		
		Заметим, что $f(G)$ работает за полином, так как она лишь дополняет вход 1 вершиной и $|V| = n$ рёбрами. Докажем теперь, что $G$ содержит ГП $\LRa$ $\tilde{G}$ содержит ГЦ
		
		\begin{itemize}
			\item \Ra: $G$ содержит ГП, возьмём его концы и соединим рёбрами с вершиной $u$ --- получим ГЦ в $\tilde{G}$
			
			\item \La: $\tilde{G}$ содержит ГЦ, пусть в этом цикле вершина $u$ соединена с $v_p$ и $v_q$ --- разорвём цикл, удалив из него вершину $u$ --- получим ГП с концами $v_p$ и $v_q$ в исходном графе $G$
		\end{itemize}
		
	\end{itemize}
\end{solution}

\begin{tasknum}{7}
	Регулярный язык $L$ задан регулярным выражением. Постройте полиномиальный алгоритм проверки непринадлежности $w\notin L$. Вы должны определить, что вы понимаете под длиной входа, и выписать явную  оценку трудоёмкости алгоритма.
\end{tasknum}

\begin{solution}
	
	Под длиной входа будем понимать $|RegExpr|$ --- длину регулярного выражения. Для начала по алгоритму из ТРЯПа (который, естественно, забыт, но должен быть упомянут для важности) строим полный ДКА по РВ, делается это за $O(n^3)$, где $n = |RegExpr|$.
	
	Теперь будем считывать само слово $w$ и моделировать на МТ работу ДКА, таким образом, считав всё слово, мы либо оказываемся в принимающем состоянии и, соответственно, $w\in L$, либо же этого не происходит и тогда мы определённо знаем, что $w\notin L$.
	
	Оценим трудоёмкость алгоритма: переход из одного состояния ДКА в другое мы выполняем за 1 такт, так что на работу над входом тратится $O(|w|)$ тактов, построение ДКА по РВ, как было сказано ранее, выполняется за $O(n^3)$.
	
	 Таким образом, в ходе алгоритма совершается $O(n^3 + |w|)$ тактов работы МТ
	
\end{solution}

\begin{tasknum}{9}
	Для языка $L \subset \Sigma^*$ определим язык $\mathsf{AND}(L) = (L\#)^* = \{w\# \mid w \in L\}^* \subset (\Sigma \cup \{\#\})^*$, где символ $\# \notin \Sigma$~--- разделитель. 
	
	Верно ли, что если языки $L_1 \subset \Sigma_1^*$ и $L_2 \subset \Sigma_2^*$ таковы, что $L_1 \le_P L_2$, то $\mathsf{AND}(L_1) \le_P \mathsf{AND}(L_2)$?
\end{tasknum}

\begin{solution}
		 
	1) $L_1 \polysv L_2 \Ra \exists f: \Sigma^* \longrightarrow \Sigma^*$ такая, что $x \in L_1 \LRa f(x) \in L_2$, причём $f$ --- полиномиально вычислима от длины входа
	
	2) $x \in \mathsf{AND}(L_j) \LRa \exists n \in \mathbb{N}_0: x \in (L_j \#)^n \LRa \exists n \in \mathbb{N}_0: x = w_1 \# w_2 \# \ldots \# w_n \#, \forall i \in \ol{1\ldots n} \vyp w_i \in L_j$ (тут при $n=0 \vyp x = \epsilon$), то есть все слова, лежащие в языке $\mathsf{AND}(L_j)$, имеют вид $w_1 \# w_2 \# \ldots \# w_n \#, \forall i \in \ol{1\ldots n} \vyp w_i \in L_j$
	
	 Таким образом имеем:
	 
	$x = w_1 \# w_2 \# \ldots \# w_n \# \in \mathsf{AND}(L_1) \xLeftrightarrow{2)} w_i \in L_1 \xLeftrightarrow{1)} f(w_i) \in L_2 \xLeftrightarrow{2)} f(w_1) \# f(w_2) \# \ldots \# f(w_n) \# \in \mathsf{AND}(L_2) \xLeftrightarrow{3)} g(w_1 \# w_2 \# \ldots \# w_n \#) = f(w_1) \# f(w_2) \# \ldots \# f(w_n) \# \in \mathsf{AND}(L_2) \LRa g(x) \in \mathsf{AND}(L_2)$
	 
	 3) определим алгоритм, вычисляющий $g(x)$ следующим образом: сначала за линейное от длины входа время проверяем вход: если $x \neq \epsilon \  \wedge \ x \neq w_1 \# w_2 \# \ldots \# w_n \#$, то сразу останавливаем вычисление функции и выдаём, что $g(x) \notin \mathsf{AND}(L_2)$. Если же вход <<правильный>>, то для каждого $w_i$ вызываем функцию $f(w_i)$ и записываем в выход $f(w_i) \#$, получая в конечном счёте на выходе $g(x) = f(w_1) \# f(w_2) \# \ldots \# f(w_n) \# \in \mathsf{AND}(L_2) \LRa f(w_i) \in L_2 \LRa w_i \in L_1$. Заметим, что $g$ --- полиномиально вычислимая от длины входа функция, так как она сначала проверяет вход (что делается за линейное время), а затем $\le |x|$ раз вызывает полиномиально вычислимую от длины $w_i$ функцию $f$, что тоже делается за полиномиальное от длины входа время
	 
	
\end{solution}

\end{document}