\documentclass[12pt,a5paper,fleqn]{article}
\usepackage[utf8]{inputenc}
\usepackage{amssymb, amsmath, multicol}
\usepackage[russian]{babel}
\usepackage{graphicx}
\usepackage[shortcuts,cyremdash]{extdash}
\usepackage{wrapfig}
\usepackage{floatflt}
\usepackage{lipsum}
\usepackage{concmath}
\usepackage{euler}

\graphicspath{ {images/} }

\oddsidemargin=-17.9mm
\textwidth=133mm
\headheight=-35.4mm
\textheight=200mm
\parindent=0pt
\tolerance=100
\parskip=6pt
\pagestyle{empty}
\renewcommand{\tg}{\mathop{\mathrm{tg}}\nolimits}
\renewcommand{\ctg}{\mathop{\mathrm{ctg}}\nolimits}
\renewcommand{\arctan}{\mathop{\mathrm{arctg}}\nolimits}
\newcommand{\divisible}{\mathop{\raisebox{-2pt}{\vdots}}}
\renewcommand{\not}{\overline}
\newcommand{\modulo}{\mathop{\mathrm{mod}}\nolimits}


\RequirePackage{caption2}
\renewcommand\captionlabeldelim{}
\newcommand*{\hm}[1]{#1\nobreak\discretionary{}%
\newtheorem{Theorem}{Теорема}
{\hbox{$\mathsurround=0pt #1$}}{}}

\begin{document}


\begin{center}
{ \Large Задание на вторую неделю.}

\end{center}

{\bf 1.} Докажите следующие свойства полиномиальной сводимости:

($i$) Рефлексивность: $A\leq_p A$; транзитивность: если $A\leq_p B$ и $B\leq_p C$, то $A\leq_p C$;

($ii$) Если $B\in\mathcal{P}$ и $A\leq_p B$, то $A\in\mathcal{P}$;

($iii$) Если $B\in\mathcal{NP}$ и $A\leq_p B$, то $A\in\mathcal{NP}$.

\smallskip

{\bf 2.} Докажите, что следующие языки принадлежат классу $\mathcal{P}$. Считайте, что графы заданы матрицами смежности.

($i$) Язык двудольных графов, содержащих не менее $2018$ треугольников (троек попарно смежных вершин);

($ii$) Язык несвязных графов без циклов;

($iii$) Язык квадратных $\{0; 1\}$-матриц порядка $n\geq 3000$, в которых есть квадратная подматрица порядка $n-2018$, заполненная одними единицами.

\smallskip

{\bf 3.} Корректно ли следующее  рассуждение? Язык $\mathsf{3-COLOR}$ сводится к языку $\mathsf{2-COLOR}$ следующим образом: добавим новую вершину и соединим её со всеми вершинами исходного графа. Тогда новый граф можно окрасить в $3$ цвета тогда и только тогда, когда исходный можно было окрасить в $2$ цвета.

При положительном ответе приведите обоснование записанной сводимости. В противном случае~--- укажите явное место ошибки.

\smallskip

{\bf 4.} Докажите, что классы $\mathcal{P}$ и $\mathcal{NP}$ замкнуты относительно операции $*$~---~звезды Клини (была в ТРЯПе). Для языка $NP$ приведите также и сертификат принадлежности слова из $\Sigma^*$ языку $L^*$, где $L\in\mathcal{NP}$. 

\smallskip


{\bf 5.} Покажите, что язык разложения на множители $$L_{factor} = \{(N,M)\in\mathbb{Z}^2 \mbox{ }|\mbox{ } 1<M<N\mbox{ и } N \mbox{ имеет делитель } d, 1<d\leq M\}$$ лежит в пересечении $\mathcal{NP}\cap co-\mathcal{NP}$. 

\smallskip

{\bf 6.} Язык ГП состоит из описаний графов, имеющих гамильтонов путь. Язык ГЦ состоит из описаний графов, имеющих  гамильтонов цикл (проходящий через все вершины, причем все вершины в этом цикле, кроме первой и последней, попарно различны). Постройте явные полиномиальные сводимости ГЦ к ГП и ГП к ГЦ.

\smallskip

{\bf 7.} Регулярный язык $L$ задан регулярным выражением. Постройте полиномиальный алгоритм проверки непринадлежности $w\notin L$. Вы должны определить, что вы понимаете под длиной входа, и выписать явную  оценку трудоёмкости алгоритма.

\smallskip

{\bf 8 (Доп).} \textit{(оба пункта по 1 баллу)} Рассмотрим СЛУ $Ax=b$ с целыми коэффициентами. Пусть в этой системе $m$ уравнений и $n$ неизвестных, причем максимальный модуль элемента в матрице $A$ и столбце $b$ равен $h$. 

($i$) Оцените сверху числители и знаменатели чисел, которые могут возникнуть при непосредственном применении метода Гаусса. Приведите пример, в котором в процессе вычислений в промежуточных результатах длина возникающих чисел растёт быстрее, чем любой полином от длины записи системы в битовой арифметике.

($ii$) Оказывается, что если на каждом шаге эмулировать рациональную арифметику и сокращать дроби с помощью алгоритма Евклида, модифицированный таким образом метод Гаусса окажется полиномиальным по входу (по поводу этого факта будет выложен доп. файл). Оцените трудоемкость такого модифицированного метода по параметрам $m$, $n$ и $\log h$.

\smallskip

{\bf 9.} Для языка $L \subset \Sigma^*$ определим язык $\mathsf{AND}(L) = (L\#)^* = \{w\# \mid w \in L\}^* \subset (\Sigma \cup \{\#\})^*$, где символ $\# \notin \Sigma$~--- разделитель. 

Верно ли, что если языки $L_1 \subset \Sigma_1^*$ и $L_2 \subset \Sigma_2^*$ таковы, что $L_1 \le_P L_2$, то $\mathsf{AND}(L_1) \le_P \mathsf{AND}(L_2)$?


\smallskip

{\bf 10 (Доп).} Рассмотрим $n$ точек плоскости, заданных своими парами декартовых координат $(x, y)$. Требуется найти их выпуклую оболочку, т.~е. наименьшее по включению выпуклое множество, содержащее все эти $n$ точек. Выпуклой оболочкой будет некоторый многоугольник, причем все его вершины~---~некоторые из этих точек, а остальные лежат внутри. Вывести на экран нужно вершины этого многоугольника по порядку обхода периметра (начиная с любой из них).

Рассмотрим модель вычисления, в которой за $1$ такт можно делать одну из трёх операций: сравнивать два числа, складывать числа и возводить число в квадрат. Покажите, что в этой модели вычислений задача сортировки массива сводится к задаче построения выпуклой оболочки $n$ точек плоскости за линейное время.

\smallskip

\end{document}


